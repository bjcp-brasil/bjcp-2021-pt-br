\phantomsection
\subsection*{1A. American Light Lager}
\addcontentsline{toc}{subsection}{1A. American Light Lager}

\textbf{Impressões gerais}: Altamente carbonatada e de corpo muito
baixo, lager quase sem sabor, criada para ser consumida bem gelada.
Muito refrescante e para matar a sede.

\textbf{Aparência}: Cor de palha ao amarelo claro. Espuma branca não
muito persistente. Límpida.

\textbf{Aroma}: De baixo a nenhum aroma de malte, caso presente, pode
ser percebido como grãos, doce ou milho. Aroma de lúpulo de baixo a
nenhum, podendo ter notas de picante, floral ou herbal. Apesar de ser
desejável um caráter limpo de fermentação, uma leve característica de
levedura não é uma falha.

\textbf{Sabor}: Relativamente neutra ao palato com um final fresco e
seco. Sabor de grãos ou milho de baixo a muito baixo, que pode ser
percebido como doçura devido ao baixo amargor. Sabor de lúpulo de baixo
a nenhum, podendo ter perfil floral, picante ou herbal, embora seja
raramente forte para ser detectado. Amargor de lúpulo de baixo a muito
baixo. Equilíbrio pode variar de ligeiramente maltado ao ligeiramente
amargo, mas comumente equilibrada. A alta carbonatação pode realçar a
sensação de frescor e o final seco. Caráter limpo de fermentação lager.

\textbf{Sensação na Boca}: Corpo muito leve, às vezes aguado.
Carbonatação muito alta com carbonatação picante na língua.

\textbf{Comentários}: Desenvolvida para cativar a mais ampla gama de
pessoas possível. Sabores fortes significam falha na cerveja. Com pouco
sabor de malte ou lúpulo, a característica da levedura muitas vezes é o
que diferencia as marcas.

\textbf{História}: A Coors produziu uma Light Lager por alguns anos na
década de 1940. Versões modernas foram produzidas inicialmente por
Rheingold em 1967 para atender os consumidores que faziam dieta, mas
somente em 1973 se tornou popular, após a cervejaria Miller adquirir a
receita e fazer grande propaganda de \emph{marketing} entre os
esportistas com o slogan ``muito gosto, menos calorias''. As cervejas
deste estilo se tornaram as mais vendidas nos EUA na década de 1990.

\textbf{Ingredientes}: Cevada de duas ou seis fileiras com até 40\% de
adjuntos (arroz ou milho). Enzimas adicionais podem ser utilizadas para
reduzir o corpo e a quantidade de carboidratos. Levedura Lager. Pouco
uso de lúpulos.

\textbf{Comparação de Estilos}: Uma versão com menor corpo, menos álcool
e menos calorias do que uma American Lager. Menos caráter de lúpulo e
amargor do que na German Leitchbier.

\textbf{Estatísticas}: IBU: 8 - 12 SRM: 2 - 3 OG: 1.028 - 1.040 FG:
0.998 - 1.008 ABV: 2.8\% - 4.2\%

\textbf{Exemplos Comerciais}: Bud Light, Coors Light, Grain Belt Premium
Light American Lager, Michelob Light, Miller Lite, Old Milwaukee Light.

\textbf{Última Revisão}: American Light Lager (2015)

\textbf{Atributos do Estilo}: Balanço, baixa-fermentação, Lagered
(maturada), América do Norte, Cor pálida, família-pale-lager,
Intensidade session, estilo tradicional.