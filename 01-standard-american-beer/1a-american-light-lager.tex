\phantomsection
\subsection*{1A. American Light Lager}
\addcontentsline{toc}{subsection}{1A. American Light Lager}
\textbf{Impressões Gerais}: Uma lager altamente carbonatada, de corpo muito baixo e quase sem sabor, criada para ser consumida muito gelada. Muito refrescante e para matar a sede.

\textbf{Aroma}: Aroma de malte baixo é opcional, mas pode ser percebido como granulado, adocicado ou que remete a cereais, se presente. Aroma leve de lúpulo condimentado, floral e/ou herbal é opcional. Embora seja desejável um perfil de fermentação limpo, uma leve presença de caráter de levedura não é considerada um defeito.

\textbf{Aparência}: Cor de palha muito claro a amarelo claro. Colarinho branco, não muito persistente. Muito límpida.

\textbf{Sabor}: Sabor relativamente neutro, com um final nítido e seco, com um sabor que remete a cereais ou semelhante a milho de baixo a muito baixo que pode ser percebido como dulçor devido ao baixo amargor. Sabor de lúpulo floral, condimentado e/ou herbal de baixo a muito baixo é opcional, mas raramente é forte o suficiente para ser detectado. Amargor de baixo a muito baixo. Equilíbrio pode variar de ligeiramente maltado a ligeiramente amargo, mas geralmente é equilibrada. A alta carbonatação pode realçar a nitidez do final seco. Perfil limpo de fermentação.

\textbf{Sensação na Boca}: Corpo muito leve, às vezes aguado. Carbonatação muito alta, gera sensação de carbonatação picante na língua.

\textbf{Comentários}: Desenvolvida para atrair o maior público possível. Sabores fortes significam falha na cerveja. Com pouco sabor de malte ou lúpulo, o caráter de levedura muitas vezes é o que diferencia as diversas marcas.

\textbf{História}: A cervejaria Coors produziu brevemente uma cerveja Light Lager no início da década de 1940. As versões modernas foram inicialmente produzidas pela Rheingold em 1967 para atrair consumidores preocupados com a alimentação saudável, mas só se tornaram populares a partir de 1973, quando a Miller Brewing adquiriu a receita e promoveu a cerveja intensivamente para os fãs de esportes com a campanha “tastes great, less filling” \textit{(em português “saboroso, com menos calorias”)}. As cervejas desse gênero se tornaram as mais vendidas nos Estados Unidos na década de 1990.

\textbf{Ingredientes}: Cevada de duas ou seis fileiras com até 40\% de adjuntos (arroz ou milho). Enzimas adicionais podem reduzir ainda mais o corpo e a quantidade de carboidratos. Levedura Lager. Mínimo uso de lúpulos.

\textbf{Comparação de Estilos}: Uma versão com corpo mais leve, menos álcool e menos calorias que uma American Lager. Menos caráter de lúpulo e amargor do que uma German Leitchbier.

\begin{tabular}{@{}p{35mm}p{35mm}@{}}
  \textbf{Estatísticas}: & OG: 1,028 - 1,040 \\
  IBU: 8 - 12  & FG: 0,998 - 1,008 \\
  SRM: 2 - 3  & ABV: 2,8\% - 4,2\%
\end{tabular}

\textbf{Exemplos Comerciais}: Bud Light, Coors Light, Grain Belt Premium Light American Lager, Michelob Light, Miller Lite, Old Milwaukee Light.

\textbf{Última Revisão}: American Light Lager (2015)

\textbf{Atributos de Estilo}: balanced, bottom-fermented, lagered, north-america, pale-color, pale-lager-family, session-strength, traditional-style.