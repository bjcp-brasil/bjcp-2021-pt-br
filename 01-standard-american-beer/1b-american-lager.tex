\phantomsection
\subsection*{1B. American Lager}
\addcontentsline{toc}{subsection}{1B. American Lager}
\textbf{Impressões Gerais}: Uma cerveja lager muito clara, altamente carbonatada, de corpo leve, bem atenuada, com um sabor neutro e amargor baixo. Servida bem gelada, muito refrescante e para matar a sede.

\textbf{Aroma}: Aroma de malte baixo é opcional, mas pode ser percebido como cereais, adocicado ou semelhante a milho, se presente. Aroma leve de lúpulo condimentado ou floral é opcional. Embora seja desejável um perfil de fermentação limpo, uma leve presença de caráter de levedura não é considerada um defeito.

\textbf{Aparência}: Cor de palha a amarelo médio. Colarinho branco, não muito persistente. Límpida.

\textbf{Sabor}: Paladar relativamente neutro, com um final bem nítido e seco. Sabor de cereais ou milho de baixo a moderadamente baixo, que pode ser percebido como dulçor, devido ao baixo amargor. Sabor de lúpulo de baixo é opcional, podendo ter perfil floral, condimentado ou herbal, embora seja raramente forte o suficiente para ser detectado. Amargor de baixo a médio-baixo. Equilíbrio pode variar de ligeiramente maltado a ligeiramente amargo, mas é geralmente equilibrada. A carbonatação alta pode realçar a nitidez e o final seco. Perfil de fermentação limpo.

\textbf{Sensação na Boca}: Corpo de baixo a médio-baixo. Carbonatação muito alta, gera sensação de carbonatação picante na língua.

\textbf{Comentários}: Frequentemente, é o que os não-apreciadores de cervejas artesanais esperam ser servido quando pedem cerveja nos Estados Unidos. Pode ser comercializada como Pilsner fora da Europa, mas não deve ser confundida com exemplos tradicionais. Sabores fortes são considerados um defeito. Com pouco sabor de malte ou lúpulo, o caráter da levedura é o que mais frequentemente diferencia as marcas.

\textbf{História}: Originou-se da Pre-Prohibition Lager (ver Categoria 27) nos Estados Unidos, após a Lei Seca e a Segunda Guerra Mundial. As cervejarias sobreviventes se consolidaram, expandiram a distribuição e promoveram intensamente um estilo de cerveja que agradava grande parte da população. Tornou-se o estilo de cerveja dominante por muitas décadas e deu origem a muitos concorrentes internacionais que desenvolveriam produtos igualmente sem sabor para o mercado em massa, apoiados por uma intensa campanha publicitária.

\textbf{Ingredientes}: Malte de cevada de duas ou seis fileiras, com até 40\% de adjuntos (arroz ou milho). Levedura Lager. Uso leve de lúpulos.

\textbf{Comparação de Estilos}: Mais forte, com mais sabor e corpo que uma American Light Lager. Menos amargor e sabor que uma International Pale Lager. Significativamente menos sabor, lúpulo e amargor do que as tradicionais Pilsners europeias.

\begin{tabular}{@{}p{35mm}p{35mm}@{}}
  \textbf{Estatísticas}: & OG: 1,040 - 1,050 \\
  IBU: 8 - 18  & FG: 1,004 - 1,010 \\
  SRM: 2 - 3,5  & ABV: 4,2\% - 5,3\%
\end{tabular}

\textbf{Exemplos Comerciais}: Budweiser, Coors Original, Grain Belt Premium Lager, Miller High Life, Old Style, Pabst Blue Ribbon, Special Export.

\textbf{Última Revisão}: American Lager (2015)

\textbf{Atributos de Estilo}: balanced, bottom-fermented, lagered, north-america, pale-color, pale-lager-family, standard-strength, traditional-style
