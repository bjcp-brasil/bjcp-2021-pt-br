\phantomsection
\subsection*{1C. Cream Ale}
\addcontentsline{toc}{subsection}{1C. Cream Ale}
\textbf{Impressões Gerais}: Uma cerveja americana, \textit{"lawnmower"} (termo em inglês usado para definir uma cerveja leve, refrescante e fácil de beber, típica de atividades ao ar livre, como cortar a grama), limpa, bem atenuada e altamente carbonatada. Fácil de beber, suave e refrescante, com mais caráter do que as típicas American Lagers, mas ainda sutil e contida.

\textbf{Aroma}: Notas de malte de médio-baixo a baixo, com um aroma doce semelhante a milho. DMS em baixa intensidade é opcional. Aroma de lúpulo médio-baixo é opcional, podendo ser de qualquer variedade, ainda que notas florais, condimentadas e herbais sejam mais comuns. Em geral, tem um aroma sutil e equilibrado. Ésteres frutados de baixa intensidade são opcionais.

\textbf{Aparência}: Cor de palha claro a dourado claro, embora geralmente no lado mais claro. Colarinho de baixa a média formação, com carbonatação de média a alta. Limpidez brilhante e cristalina. Esfervecente.

\textbf{Sabor}: Amargor de lúpulo de baixo a médio-baixo. Dulçor de malte de baixo a médio-baixo, variando de acordo com a densidade original e atenuação. O malte é geralmente neutro, possivelmente com notas de grãos ou bolacha água-e-sal. Geralmente bem atenuada. Sabor equilibrado, com lúpulo apenas o suficiente para dar suporte ao malte. Um sabor leve de milho é comum, mas DMS em baixa intensidade é opcional. O final pode variar de seco, leve e bem definido, ao levemente doce. Perfil de fermentação limpo, mas ésteres frutados de baixa intensidade são opcionais. Sabor de lúpulo de baixo a médio-baixo de qualquer variedade, mas tipicamente floral, condimentado e/ou herbal. Sutil.

\textbf{Sensação na Boca}: Geralmente leve e bem definida, embora o corpo possa ser médio. Sensação na boca é suave, com atenuação de média a alta; níveis de atenuação mais elevados podem conferir uma qualidade \textit{saciante}. Alta carbonatação.

\textbf{Comentários}: A maioria dos exemplares comerciais possuem OG (densidade original) variando na faixa de 1,050-1,053 e o amargor raramente ultrapassa 20 IBU.

\textbf{História}: Uma ale de alta carbonatação e de consumo rápido da segunda metade do século XIX, que sobreviveu à Lei Seca. Uma ale feita para competir com as lagers produzidas no Canadá e nos estados do Nordeste, Médio Atlântico e Centro-Oeste dos Estados Unidos.

\textbf{Ingredientes}: Malte americano de seis fileiras ou uma combinação de malte de seis fileiras com malte Norte Americano de duas fileiras. Até 20\% de milho na mostura e até 20\% de açúcar na fervura. Qualquer variedade de lúpulo, frequentemente Americanos rústicos e/ou Continentais. Levedura de ale limpa ou uma mistura de cerveja ale e lager.

\textbf{Comparação de Estilos}: Semelhante ao estilo Standard American Lager, com mais caráter. Corpo mais leve, mais suave e mais carbonatada que uma Blonde Ale. Pode parecer uma Kölsch um pouco mais sutil.

\begin{tabular}{@{}p{35mm}p{35mm}@{}}
  \textbf{Estatísticas}: & OG: 1,042 - 1,055 \\
  IBU: 8 - 20  & FG: 1,006 - 1,012 \\
  SRM: 2 - 5  & ABV: 4,2\% - 5,6\%
\end{tabular}

\textbf{Exemplos Comerciais}: Genesee Cream Ale, Liebotschaner Cream Ale, Little Kings Cream Ale, Kiwanda Pre-Prohibition Cream Ale, Sleeman Cream Ale, Sun King Sunlight Cream Ale.

\textbf{Última Revisão}: Cream Ale (2015)

\textbf{Atributos de Estilos}: any-fermentation, balanced, north-america, pale-ale-family, pale-color, standard-strength, traditional-style

