\phantomsection
\subsection*{2A. International Pale Lager}
\addcontentsline{toc}{subsection}{2A. International Pale Lager}
\textbf{Impressão Geral}: Uma lager clara altamente atenuada, sem sabores fortes, geralmente bem equilibrada e altamente carbonatada. Servida gelada, é refrescante e mata a sede.

\textbf{Aroma}: Aroma de malte que remete a cereais ou levemente doce, como milho, de baixo a médio-baixo. Aroma de lúpulo condimentado, floral ou herbal de muito baixo a médio. Perfil de fermentação limpo.

\textbf{Aparência}: Cor de palha clara a dourada. Espuma branca e cremosa, que pode ter baixa retenção. Muito límpida.

\textbf{Sabor}: Sabor de malte remetendo a cereais de baixo a moderado, amargor de médio-baixo a médio, com final bem definido, seca e bem atenuada. O caráter de cereais pode ser um tanto neutro ou demonstrar uma leve qualidade de biscoito água-e-sal ou pão. Caráter moderado de milho ou dulçor de malte são opcionais. Sabor de lúpulo floral, condimentado e/ou herbal médio também é opcional. O equilíbrio pode variar de levemente maltado a levemente amargo, mas normalmente é mais próximo de ser equilibrado. Retrogosto neutro com malte leve e, às vezes, sabor de lúpulo.

\textbf{Sensação na Boca}: Corpo de leve a médio. Carbonatação de média-alta a alta. Pode apresentar uma leve picância de gás carbônico língua.

\textbf{Comentários}: Costuma ter menos adjuntos que as American Lagers. Pode ser puro malte, mas sabores fortes ainda são considerados falhas. Engloba uma ampla categoria de lagers internacionais do mercado de massa que vão das lagers americanas mais premium até as típicas cervejas internacionais “importadas” ou de “garrafas verdes” encontradas nos Estados Unidos e em muitos mercados de exportação. Muitas vezes rotulada de forma confusa como “Pilsner”. Qualquer skunky (aroma de gambá) nas cervejas comerciais é uma falha de manuseio e não uma característica do estilo.

\textbf{História}: Desenvolvida nos Estados Unidos como a versão premium da American lager padrão, com história semelhante. Fora dos Estados Unidos, desenvolvida como uma imitação das lagers de estilo americano ou como uma versão mais acessível (e frequentemente mais seca e menos amarga) de uma cerveja do tipo Pilsner. Geralmente amplamente comercializadas e exportadas por grandes cervejarias industriais ou multinacionais.

\textbf{Ingredientes}: Cevada de duas ou seis fileiras. Arroz, milho ou açúcar como adjuntos podem ser usados, mas geralmente são puro malte.

\textbf{Comparação de Estilos}: Geralmente mais amarga e encorpada que a American Lager. Menos lupulada e amarga que a German Pils. Menos corpo, sabor de malte e caráter de lúpulo do que a Czech Premium Pale Lager. Versões mais robustas podem se aproximar em sabor a Munich Helles, embora com mais caracaterística de adjuntos.

\textbf{Instruções para Inscrição}: Caso deseje, o participante pode especificar variações regionais (Mexican lager, Dutch lager, etc).

\begin{tabular}{@{}p{35mm}p{35mm}@{}}
  \textbf{Estatísticas}: & OG: 1,042 - 1,050 \\
  IBU: 18 - 25  & FG: 1,008 - 1,012  \\
  SRM: 2 - 6   & ABV: 4,5\% - 6\%
\end{tabular}

\textbf{Exemplos Comerciais}: Asahi Super Dry, Birra Moretti, Corona Extra, Devils Backbone Gold Leaf Lager, Full Sail Session Premium Lager, Heineken, Red Stripe, Singha.

\textbf{Última Revisão}: International Pale Lager (2015)

\textbf{Atributos de Estilo}: balanced, bottom-fermented, lagered, pale-color, pale-lager-family, standard-strength, traditional-style