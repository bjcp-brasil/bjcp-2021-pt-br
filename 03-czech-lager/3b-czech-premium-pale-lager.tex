\phantomsection
\subsection*{3B. Czech Premium Pale Lager}
\addcontentsline{toc}{subsection}{3B. Czech Premium Pale Lager}

\textbf{Impressão Geral}: Uma lager tcheca clara e refrescante com considerável caráter de malte e lúpulo e final prolongado. Os sabores de malte são complexos para uma cerveja do tipo Pilsner. O amargor é forte e limpo, mas sem aspereza, o que traz a impressão de um sabor arredondado e bem equilibrado, o que realça a facilidade em beber.

\textbf{Aroma}: Maltado rico de médio a médio-alto como pão e buquê de lúpulo floral, herbal e/ou picante/condimentado de médio-baixo a médio-alto. Embora o equilíbrio entre malte e lúpulo possa variar, a interação é rica e complexa. Leve diacetil ou ésteres frutados muito baixos são opcionais. Os ésteres tendem a aumentar com a densidade.

\textbf{Aparência}: Cor amarelo médio a dourado intenso. Claridade brilhante a muito límpida. Colarinho branco, cremoso, denso e de longa duração.

\textbf{Sabor}: Caráter de malte rico e complexo com notas de pão, combinado com um pronunciado, ainda sim suave e arredondado, amargor e sabor de lúpulo floral e/ou condimentado. Os sabores do malte e do lúpulo vão de médio a médio-alto. O malte pode conter uma leve impressão de caramelo. O amargor é proeminente, mas nunca áspero. O final longo pode ser equilibrado em direção ao lúpulo ou ao malte, mas nunca é agressivamente inclinado em qualquer destas direções. Diacetil de leve a moderadamente baixo e baixos ésteres derivados do lúpulo são aceitáveis, mas não precisam estar presentes.

\textbf{Sensação na Boca}: Corpo médio. Carbonatação de baixa a moderada.

\textbf{Comentários}: Geralmente um grupo de cervejas pivo Plzenského typu, ou do tipo Pilsner. Este estilo é uma combinação dos estilos tchecos svetlý ležák (11–12,9°P) e svetlé speciální pivo (13–14,9°P). Na República Tcheca somente a Pilsner Urquell e a Gambrinus são chamadas de Pilsner, apesar da enorme adoção deste nome em todo o mundo. Fora da República Tcheca, Czech Pilsner ou Bohemian Pilsner são às vezes utilizadas para diferenciar as cervejas de outras do tipo Pilsner. As versões Kvasnicové (“cerveja com levedura”) são populares na República Tcheca e podem receber kräusening com mosto inoculado, ou uma dose fresca de levedura pura após a fermentação. Estas cervejas, às vezes, são turvas, com caráter sutil de levedura e um caráter de lúpulo realçado. Os exemplares modernos variam no seu equilíbrio entre malte e lúpulo e muitos não são tão focadas no lúpulo quanto a Pilsner Urquell.

\textbf{História}: Comumente associada à Pilsner Urquell, que foi fabricada pela primeira vez em 1842 após a construção de uma nova cervejeira por cidadãos insatisfeitos com o padrão da cerveja produzida em Plzeň. O cervejeiro bávaro Josef Groll é creditado como o primeiro a fabricar esta cerveja, embora anteriormente possam ter existido cervejas claras na Bohemia. Tão importante quanto a levedura lager foi o uso de técnicas inglesas de malteação.

\textbf{Ingredientes}: Lúpulos tchecos tradicionais. Malte tcheco. Levedura lager tcheca. Água com baixo teor de sulfato e carbonato que fornece um perfil de lúpulo distintamente macio e arredondado, apesar das altas taxas de lupulagem. O nível de amargor de alguns exemplares mais comerciais caiu nos últimos anos, embora não tanto como em diversos exemplares alemães contemporâneos.

\textbf{Comparação de Estilo}: Mais cor, riqueza de malte e corpo do que uma German Pils, com um final mais cheio e com uma impressão mais limpa e suave. Mais forte que uma Czech Pale Lager.

\begin{tabular}{@{}p{35mm}p{35mm}@{}}
  \textbf{Estatísticas}: & OG: 1,044 - 1,060 \\
  IBU: 30 - 45  & FG: 1,013 - 1,017  \\
  SRM: 3,5 - 6  & ABV: 4,2\% - 5,8\%
\end{tabular}

\textbf{Exemplos Comerciais}: Bernard Svátecní ležák, Budvar 33 svetlý ležák, Pilsner Urquell, Pivovar Jihlava Ježek 11°, Primátor Premium, Radegast Ryze horká 12, Únetická 12°.

\textbf{Última Revisão}: Czech Premium Pale Lager (2015)

\textbf{Atributos de Estilo}: balanced, bottom-fermented, central-europe, hoppy, lagered, pale-color, pilsner-family, standard-strength, traditional-style
