\phantomsection
\subsection*{3C. Czech Amber Lager}
\addcontentsline{toc}{subsection}{3C. Czech Amber Lager}
\textbf{Impressão Geral}: Uma lager tcheca âmbar, maltada, com um caráter de lúpulo que pode variar de baixo a bastante significante. Os sabores de malte também podem variar, levando a diferentes interpretações e equilíbrios, variando de mais secos, como pão e leve biscoito a mais doces e com toques de caramelo.

\textbf{Aroma}: Rico aroma de malte de intensidade moderada que pode ser como pão e dominada por produtos de Maillard ou com leve dulçor como caramelo. Caráter de lúpulo condimentado, floral e/ou herbal pode ser de moderado a nenhum. Caráter limpo de lager, embora baixos ésteres frutados (frutas de caroço ou bagas) possam estar presentes. Diacetil baixo é opcional.

\textbf{Aparência}: Cor âmbar profundo a cobre. Clareza de limpa a brilhante. Colarinho de alta formação, quase branco e persistente.

\textbf{Sabor}: O sabor de malte complexo é dominante (de médio a médio-alto), embora sua natureza possa variar de seco e dominado por produtos de Maillard a caramelo e quase doce. Alguns exemplares tem o caráter semelhante a doce ou biscoito cracker. Sabor condimentado de lúpulo de baixo a moderado. Amargor de lúpulo proeminente, mas limpo, que proporciona um final equilibrado. Ésteres sutis como ameixa ou baga são opcionais. Baixo diacetil é opcional. Sem sabores de malte torrado. O final pode variar de seco e lupulado a relativamente doce.

\textbf{Sensação na Boca}: Corpo de médio a médio-cheio. Suave e arredondada, frequentemente com uma cremosidade delicada. Carbonatação de baixa a moderada.

\textbf{Comentários}: O nome tcheco deste estilo é polotmavé pivo, que se traduz como cerveja meio escura. Este estilo é a combinação dos estilos tchecos polotmavý ležák (11–12,9°P) e polotmavé speciální pivo (13–14,9°P). Algumas versões podem ser uma mistura de lagers claras e escuras.

\textbf{História}: Uma cerveja ao estilo Vienna Lager que continua a ser fabricada na República Tcheca. O ressurgimento e abertura de pequenas cervejarias na República Tcheca aumentou o número de exemplares deste estilo.

\textbf{Ingredientes}: Malte pilsner e caramelo, mas o malte Vienna e Munich também podem ser utilizados. Água com baixo teor mineral. Lúpulos tchecos tradicionais. Levedura lager tcheca.

\textbf{Comparação de Estilo}: Este estilo pode ser semelhante a uma Vienna Lager, mas com um caráter de lúpulos tchecos de adição tardia mais pronunciado, ou se aproxima a uma Britsh Bitter, mas com um caráter de malte significativamente mais rico e com um caráter de caramelo mais profundo. As versões das grandes cervejarias são geralmente semelhantes à Czech Premium Pale Lager com sabores leves de malte mais escuro e menos lúpulo, enquanto cervejarias menores frequentemente produzem versões com um considerável caráter de lúpulo, complexidade de malte ou dulçor residual.

\begin{tabular}{@{}p{35mm}p{35mm}@{}}
  \textbf{Estatísticas}: & OG: 1,044 - 1,060 \\
  IBU: 20 - 35  & FG: 1,013 - 1,017  \\
  SRM: 10 - 16   & ABV: 4,4\% - 5,8\%
\end{tabular}

\textbf{Exemplos Comerciais}: Bernard Jantarový ležák, Gambrinus Polotmavá 12°, Kozel Semi-Dark, Lobkowicz Démon 13, Primátor polotmavý 13°, Strakonický Dudák Klostermann polotmavý ležák 13°.

\textbf{Última Revisão}: Czech Amber Lager (2015)

\textbf{Atributos de Estilo}: amber-color, amber-lager-family, balanced, bottom-fermented, central-europe, lagered, standard-strength, traditional-style