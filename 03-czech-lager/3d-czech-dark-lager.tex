\phantomsection
\subsection*{3D. Czech Dark Lager}
\addcontentsline{toc}{subsection}{3D. Czech Dark Lager}
\textbf{Impressão Geral}: Uma lager tcheca maltada de caráter rico, escura, com perfil torrado que pode variar de quase nenhum até muito proeminente. Equilíbrio para o maltado e com um perfil de sabor interessante e complexo com vários níveis de lupulagem que permite um grande leque de interpretações possíveis.

\textbf{Aroma}: Maltado de intensidade média a médio alta com caráter rico, complexo, às vezes remetendo ao dulçor do malte, com notas opcionais tais como casca de pão, tostado, castanhas, noz de cola, fruta escura ou caramelo. Caráter de malte torrado tal como chocolate ou café adoçado pode variar de moderado a nenhum, mas não deve sobrepor o caráter de malte base. Aroma de lúpulo condimentado de baixo a moderado é opcional. Diacetil baixo e ésteres frutados (ameixas e bagas) podem estar presentes.

\textbf{Aparência}: Cor cobre escuro até quase preta, frequentemente com tons de vermelho ou granada. De límpida a brilhante. Colarinho persistente, volumoso, de quase branco a bege.

\textbf{Sabor}: Maltado profundo e complexo de médio a médio-alto domina, tipicamente com sabores provenientes de produtos de Maillard com dulçor residual de malte de baixo a moderado. Sabores de malte tais como caramelo, tostado, castanhas, alcaçuz, frutas passas escuras, chocolate ou café podem também estar presentes, com caráter torrado de muito baixo a moderado. Sabor de lúpulo condimentado de baixo a moderado. Amargor de médio a médio-baixo, mas deve ser perceptível. Equilíbrio pode variar entre maltado, relativamente equilibrado à ligeiramente lupulado. Diacetil de baixo a moderado e ésteres de ameixas ou bagas/berries podem estar presentes em níveis baixos.

\textbf{Sensação na Boca}: Corpo de médio a médio-cheio, com um preenchimento de boca considerável, sem ser pesada ou enjoativa. Textura moderadamente cremosa. Suave. Carbonatação moderada a baixa. Pode ter um leve aquecimento alcoólico nas versões mais fortes.

\textbf{Comentários}: Esse estilo é uma combinação entre os estilos tchecos tmavý ležák (11–12,9°P) e tmavé speciální pivo (13-14,9°P). Exemplos mais modernos são mais secos e têm um amargor mais alto, enquanto as versões tradicionais frequentemente têm IBU entre 18-20 e um equilíbrio mais para o adocicado.

\textbf{História}: A cervejaria U Fleku tem operado em Praga desde 1499 e produz a versão mais conhecida. Várias pequenas e novas cervejarias estão produzindo o estilo.

\textbf{Ingredientes}: Maltes Pilsner e caramelo escuro com adição de maltes torrados sem casca são mais comuns, mas adições de malte Munich e Vienna também são apropriadas. Água de baixo conteúdo mineral. Lúpulos tchecos tradicionais. Levedura lager tcheca.

\textbf{Comparação de Estilo}: A cerveja é a equivalente tcheca de uma lager escura com caráter entre a Munich Dunkel e a Schwarzbier, mas tipicamente com maior riqueza maltada, aroma de lúpulo, sabor e amargor.

\begin{tabular}{@{}p{35mm}p{35mm}@{}}
  \textbf{Estatísticas}: & OG: 1,044 - 1,060 \\
  IBU: 18 - 34  & FG: 1,013 - 1,017  \\
  SRM: 17 - 35   & ABV: 4,4\% - 5,8\%
\end{tabular}

\textbf{Exemplos Comerciais}: Bernard cerný ležák 12, Budvar tmavý ležák, Herold tmavé silné pivo 13°, Kozel Dark, Krušovice cerné, Primátor dark lager, U Fleku Flekovský tmavý 13° ležák.

\textbf{Última Revisão}: Czech Dark Lager (2015)

\textbf{Atributos de Estilo}: balanced, bottom-fermented, central-europe, dark-color, dark-lager-family, lagered, standard-strength, traditional-style