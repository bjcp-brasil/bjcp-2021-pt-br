\phantomsection
\subsection*{5A. German Leichtbier}
\addcontentsline{toc}{subsection}{5A. German Leichtbier}
\textbf{Impressão Geral}: Uma lager alemã clara, altamente atenuada, com corpo leve e com menos teor alcoólico e calorias que outras cervejas de teor alcoólico padrão. Moderadamente amarga com sabores perceptíveis de malte e lúpulo e, ainda sim, é uma cerveja interessante de beber.

\textbf{Aroma}: Aroma de lúpulo de baixo a médio com caráter condimentado, de herbal e/ou floral. Aroma de malte de baixo a médio-baixo adocicado de cereais ou ligeiramente como crackers. Perfil de fermentação limpo.

\textbf{Aparência}: Cor de palha claro a amarelo profundo. Limpidez brilhante. Colarinho branco moderado com persistência de média a média-baixa.

\textbf{Sabor}: Sabor inicial malte inicial adocicado de cereais de baixo a médio. Amargor de lúpulo médio. Sabor de lúpulo com característica condimentada, herbal e/ou floral de baixo a médio. Caráter de fermentação limpo, com maturação a frio bem feita. Final seco com um retrogosto levemente maltado e lupulado.

\textbf{Sensação na Boca}: Corpo de leve a muito leve. Carbonatação de média a alta. Suave, bem atenuada.

\textbf{Comentários}: Comercializada primariamente como uma cerveja orientada para dietas, com teor de carboidratos, álcool e calorias mais baixos. Pronuncia-se “LAIXT-beer.” Pode ser conhecida também como Diat Pils ou Helles, esse estilo está na classe de densidade \textit{schankbier}. Outras variações de cervejas da classe \textit{Leicht} (leve) podem ser feitas com Weissbier, Kölsch e Altbier; essas cervejas se enquadram melhor no estilo 34B Mixed-Style Beer.

\textbf{História}: As versões tradicionais existiam como bebidas para trabalhadores braçais em fábricas ou nos campos, mas as versões modernas são baseadas em produtos populares norte americanos da mesma classe e direcionados para consumidores fitness preocupados com saúde ou bem estar. Vem cada vez mais sendo substituída no mercado atual por cervejas não-alcoólicas e \textit{radlers}.

\textbf{Ingredientes}: Malte Pils da Europa continental. Levedura lager alemã. Lúpulos tradicionais alemães.

\textbf{Comparação de Estilo}: Como uma German Pils ou Munich Helles com teor alcoólico mais baixo, corpo mais leve e levemente menos agressiva. Mais amarga e mais saborosa que uma American Light Lager.

\begin{tabular}{@{}p{35mm}p{35mm}@{}}
  \textbf{Estatísticas}: & OG: 1,026 - 1,034 \\
  IBU: 15 - 28  & FG: 1,006 - 1,010  \\
  SRM: 1,5 - 4  & ABV: 2,4\% - 3,6\%
\end{tabular}

\textbf{Exemplos Comerciais}: Autenrieder Schlossbräu Leicht, Greif Bräu Leicht, Hohenthanner Tannen Hell Leicht, Müllerbrau Heimer Leicht, Schönramer Surtaler Schankbier, Waldhaus Sommer Bier.

\textbf{Última Revisão}: German Leichtbier (2015)

\textbf{Atributos de Estilo}: bitter, bottom-fermented, central-europe, hoppy, lagered, pale-color, pale-lager-family, session-strength, traditional-style
