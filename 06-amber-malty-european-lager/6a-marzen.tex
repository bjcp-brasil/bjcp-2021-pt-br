\phantomsection
\subsection*{6A. Märzen}
\addcontentsline{toc}{subsection}{6A. Märzen}
\textbf{Impressão Geral}: Uma cerveja lager alemã de cor âmbar com sabor de malte limpo, rico, tostado, como pão, amargor contido e um final bem atenuado. A impressão geral de malte é suave, elegante e complexa, com um retrogosto maltado rico que nunca é enjoativo ou pesado.

\textbf{Aroma}: Aroma maltado moderado, tipicamente rico, como pão, um pouco tostado e com leve notas de casca de pão. Perfil de fermentação lager limpo. Aroma de lúpulo muito baixo floral, herbal e/ou condimentado é opcional. Aromas de malte doce como caramelo, similar a biscoito seco ou torrado são inapropriados. Álcool muito leve pode ser percebido, mas nunca deve ser forte. O aroma primário deve ser de malte rico, limpo e elegante.

\textbf{Aparência}: Cor de âmbar alaranjado a cobre avermelhado profundo; não deve ser dourada. Limpidez brilhante com colarinho quase branco persistente.

\textbf{Sabor}: Sabor de malte rico de moderado a alto, que muitas vezes sugere um dulçor inicial, mas o final é de moderadamente seco a seco. Maltado característico e complexo, que muitas vezes inclui aspectos como pão e tostado. O amargor de lúpulo é moderado e o sabor de lúpulo floral, herbal e/ou condimentado é de baixo a inexistente. O lúpulo proporciona equilíbrio suficiente para que o sabor maltado e o final não pareçam doces. O retrogosto é maltado e persistente com os mesmos sabores de malte elegantes e ricos. Sabores aparentes de caramelo doce, biscoito seco ou torrado são inapropriados. Perfil de fermentação limpo.

\textbf{Sensação na Boca}: Corpo médio com texture suave e cremosa que muitas vezes sugere uma sensação na boca mais cheia. Carbonatação média. Completamente atenuada, sem impressão de dulçor ou enjoativa. Pode ter um leve aquecimento alcoólico, mas a força alcoólica deve estar relativamente escondida.

\textbf{Comentários}: As versões modernas domésticas da Oktoberfest alemã são douradas - veja o estilo Festbier para essa versão. Versões alemãs para exportação (para os Estados Unidos, pelo menos) geralmente são de cor laranja-âmbar, tem um caráter distinto de malte tostado e, muitas vezes, são rotuladas como Oktoberfest. Muitas versões artesanais de Oktoberfest são baseadas nesse estilo. Versões históricas da cerveja tendiam a ser mais escuras, mais para a faixa de cor marrom, mas houve muitos “tons” de Märzen (quando o nome era usado como teor alcoólico). A descrição desse estilo se refere especificamente à versão com maior teor alcoólico e de cor âmbar. A Festbier atual pode ser descrita como uma Märzen de corpo mais leve e mais clara.

\textbf{História}: Como o nome sugere, produzida em março como uma “cerveja de março” com maior teor alcoólico e com uma maturação longa em cavernas frias durante o verão. As versões modernas remontam à lager desenvolvida pela Spaten em 1841, contemporânea ao desenvolvimento da Vienna Lager. Entretanto, o nome Märzen é muito mais antigo que 1841 - as primeiras eram marrom-escuras e o nome implicava uma faixa de teor alcoólico (14°P), ao invés de um estilo. O estilo âmbar lager foi servido na Oktoberfest de 1872 até 1990, quando a Festbier, de cor dourada, foi adotada como a cerveja padrão do festival.

\textbf{Ingredientes}: O perfil de maltes varia, embora versões tradicionais alemãs enfatizem o malte Munich. A noção de elegância é proveniente de ingredientes da melhor qualidade, particularmente os maltes base. Mostura usando decocção é tradicional e realça o perfil rico de malte.

\textbf{Comparação de Estilo}: Não tão forte e rica como uma Dunkles Bock. Com mais profundidade e riqueza de malte do que uma Festbier, com um corpo mais cheio e um pouco menos de lúpulo. Menos lupulada, mas igualmente maltada que uma Czech Amber Lager, mas com um perfil de malte diferente.

\begin{tabular}{@{}p{35mm}p{35mm}@{}}
  \textbf{Estatísticas}: & OG: 1,054 - 1,060 \\
  IBU: 18 - 24  & FG: 1,010 - 1,014  \\
  SRM: 8 - 17  & ABV: 5,6\% - 6,3\%
\end{tabular}

\textbf{Exemplos Comerciais}: Hacker-Pschorr Oktoberfest Märzen, Hofmark Märzen, Paulaner Oktoberfest, Saalfelder Ur-Saalfelder, Weltenburg Kloster Anno 1050.

\textbf{Última Revisão}: Märzen (2015)

\textbf{Atributos de Estilo}: amber-color, amber-lager-family, bottom-fermented, central-europe, lagered, malty, standard-strength, traditional-style
