\phantomsection
\subsection*{8A. Munich Dunkel}
\addcontentsline{toc}{subsection}{8A. Munich Dunkel}
\textbf{Impressão Geral}: Uma lager marrom e maltada tradicional da Baviera. Sabores de malte profundamente tostados, como pão, sem qualquer sabor torrado ou queimado. Suave e rica com amargor contido e final relativamente seco que permite ser consumida em quantidade.

\textbf{Aroma}: Maltado rico de moderado a alto, como casca de pão tostada com notas de chocolate, nozes, caramelo e/ou \textit{toffee}. Versões tradicionais frescas frequentemente apresentam níveis mais altos de chocolate. O caráter de malte é mais um maltado rico do que um dulçor açucarado ou de caramelo. Perfil de fermentação limpo. Um aroma leve de lúpulo condimentado, floral e/ou herbal é opcional.

\textbf{Aparência}: De cobre profundo a marrom escuro, frequentemente com um tom vermelho ou granada. Colarinho cremoso de bege claro a bege médio. Geralmente límpida.

\textbf{Sabor}: Sabores ricos de malte similares ao aroma (com os mesmos descritores de malte) de médio a alto. Amargor contido, de médio-baixo a médio, que proporciona um equilíbrio geral maltado. Maltada e suave na boca, sem ser excessivamente doce, e um final médio seco com um retrogosto maltado. Sem sabores de malte torrado, queimado e/ou amargo, os sabores tostados não devem ter uma secura áspera de cereais e os sabores de caramelo não devem ser doces. Sabor de lúpulo condimentado, herbal e/ou floral de baixa intensidade é opcional. Caráter de fermentação limpo.

\textbf{Sensação na Boca}: Corpo de médio a médio-cheio, dando uma sensação suave e que lembra dextrinas, sem ser pesada ou enjoativa. Carbonatação moderada. Caráter de lager suave. Sem adstringência ápera ou agressiva. Sem aquecimento alcoólico.

\textbf{Comentários}: Um estilo tradicional de Munich, a companheira escura da Helles. Versões da Francônia são mais amargas do que as de Munich.

\textbf{História}: Desenvolvida na Spaten, em meados de 1830, depois do desenvolvimento do malte Munich e vista como a sucessora das cervejas escuras regionais da época. Ainda que sendo original de Munich, o estilo se tornou popular por toda a Baviera (especialmente na Francônia).

\textbf{Ingredientes}: Tradicionalmente, maltes Munich, mas Pils e Vienna também podem ser usados. Leve uso de maltes especiais para cor e profundidade. Mostura por decocção é tradicional. Lúpulos alemães e levedura lager.

\textbf{Comparação de Estilo}: Não é tão intensa em malte ou potente como uma Dunkles Bock. Sem os sabores mais torrados e muitas vezes o amargor de lúpulo de uma Schwarzbier. Mais rica, centrada em malte e menos lupulada que uma Czech Dark Lager.

\begin{tabular}{@{}p{35mm}p{35mm}@{}}
  \textbf{Estatísticas}: & OG: 1,048 - 1,056 \\
  IBU: 18 - 28 & FG: 1,010 - 1,016 \\
  SRM: 17 - 28 & ABV: 4,5\% - 5,6\%
\end{tabular}

\textbf{Exemplos Comerciais}: Ayinger Altbairisch Dunkel, Ettaler Kloster-Dunkel, Eittinger Urtyp Dunkel, Hacker-Pschorr Münchner Dunkel, Hofbräuhaus Dunkel, Weltenburger Kloster Barock-Dunkel.

\textbf{Última Revisão}: Munich Dunkel (2015)

\textbf{Atributos de Estilo}: bottom-fermented, central-europe, dark-color, dark-lager-family, lagered, malty, standard-strength, traditional-style
