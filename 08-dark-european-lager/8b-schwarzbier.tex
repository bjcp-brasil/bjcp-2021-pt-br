\phantomsection
\subsection*{8B. Schwarzbier}
\addcontentsline{toc}{subsection}{8B. Schwarzbier}
\textbf{Impressão Geral}: Uma lager alemã escura que equilibra sabores de malte torrado, mas ainda assim suaves, com amargor de lúpulo moderado. O corpo mais leve, a secura e a ausência de um retrogosto áspero, queimado ou pesado ajudam a fazer essa cerveja ser bastante fácil de beber.

\textbf{Aroma}: Aroma de malte de baixo a moderado, com dulçor maltado baixo e muitas vezes notas de malte torrado aparentes. O malte pode ser limpo e neutro ou moderadamente rico e com notas de pão, e pode ter um toque de caramelo escuro. O cartáter de torrado pode ser algo parecido com chocolate amargo ou café, mas não deve nunca ser queimado. Aroma de lúpulo condimentado, floral e/ou herbal de intensidade moderadamente baixa é opcional. Caráter limpo de levedura lager.

\textbf{Aparência}: De cor marrom médio a marrom muito escuro, frequentemente com reflexos de rubi profundo a granada, mas quase nunca é verdadeiramente preta. Muito límpida. Colarinho alto, persistente e de cor castanho claro.

\textbf{Sabor}: Sabor de malte de leve a moderado, podendo apresentar caráter de neutro e limpo até moderadamente rico com uma qualidade maltada lembrando pão. Sabor de malte torrado de leve a moderado pode proporcionar sabor de chocolate amargo que nunca tem sabor queimado. Amargor de médio-baixo a médio. Sabor de lúpulo com caráter condimentado, floral e/ou herbal de leve a moderado. Perfil lager limpo. Final seco. Algum dulçor residual é aceitável, mas não é tradicional. Em segundo plano, retrogosto sutil de amargor de lúpulo com torrado complementar de fundo.

\textbf{Sensação na Boca}: Corpo de médio-leve a médio. Carbonatação de moderada a moderadamente alta. Suave. Sem aspereza ou adstringência, mesmo com o uso de maltes torrados escuros.

\textbf{Comentários}: Em Alemão, significa literalmente cerveja preta. Embora às vezes seja chamada de "Pils preta", a cerveja raramente é tão escura quanto o preto ou tão intensamente lupulada e amarga quanto uma Pils. Sabores torrados intensos como os de uma Porter são uma falha.

\textbf{História}: Uma especialidade regional da Turingia, Saxônia e Francônia, na Alemanha. Serviu como inspiração para lagers pretas brassadas no Japão. A popularidade aumentou depois da reunificação alemã em 1990.

\textbf{Ingredientes}: Malte Munich e Pilsner alemães para a base, com maltes torrados escuros sem casca que adicionam sabores de torra sem sabores queimados. Variedades alemãs de lúpulo e levedura lager alemã de fermentação limpa são tradicionais.

\textbf{Comparação de Estilo}: Em comparação com uma Munich Dunkel, geralmente tem a cor mais escura, é mais seca na boca, apresenta corpo mais leve e perceptível (mas não alta) nota de malte torrado para equilibrar o malte base. Não deve ter o sabor de uma American Porter feita com levedura lager. Mais seca, menos maltada e com menos caráter de lúpulo que uma Czech Dark Lager.

\begin{tabular}{@{}p{35mm}p{35mm}@{}}
  \textbf{Estatísticas}: & OG: 1,046 - 1,052 \\
  IBU: 20 - 35 & FG: 1,010 - 1,016 \\
  SRM: 19 - 30 & ABV: 4,4\% - 5,4\%
\end{tabular}

\textbf{Exemplos Comerciais}: Chuckanut Schwarz Lager, Devils Backbone Schwartz Bier, Köstritzer Schwarzbier, Kulmbacher Mönchshof Schwarzbier, Mönchshof Schwarzbier, Neuzeller Original Badebier, pFriem Schwarzbier.

\textbf{Última Revisão}: Schwarzbier (2015)

\textbf{Atributos de Estilo}: balanced, bottom-fermented, central-europe, dark-color, dark-lager-family, lagered, standard-strength, traditional-style
