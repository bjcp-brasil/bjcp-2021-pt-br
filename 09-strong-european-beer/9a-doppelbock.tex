\phantomsection
\subsection*{9A. Doppelbock}
\addcontentsline{toc}{subsection}{9A. Doppelbock}
\textbf{Impressão Geral}: Uma lager alemã forte, rica e bem maltada que pode ter variantes tanto clara quanto escuras. As versões mais escuras têm sabores de malte mais intensamente desenvolvidos e profundos, enquanto as versões mais claras são levemente mais lupuladas e secas.

\textbf{Aroma}: Maltado muito intenso, possivelmente com leve nota de caramelo, e aroma de álcool de ausente a moderado. Virtualmente nenhum aroma de lúpulo. Versões escuras têm aroma significativo de produtos ricos de Maillard, malte profundamente tostado e possivelmente um leve aroma lembrando chocolate, mas que nunca deve ser torrado ou queimado. Frutas escuras de intensidade moderadamente-baixa, como ameixas, uvas escuras e couro de frutas são permitidas. Versões claras têm uma presença de malte rica e intensa, frequentemente tostada, possivelmente uma nota leve de lúpulo de caráter floral, picante e/ou herbal.

\textbf{Aparência}: Boa limpidez, com um colarinho volumoso, cremoso e persistente. Versões escuras apresentam cor de cobre a marrom escuro, frequentemente com tons rubis e colarinho quase branco. Versões claras têm cor de dourado profundo a âmbar claro e colarinho branco.

\textbf{Sabor}: Muito rica e maltada. Amargor de lúpulo varia de moderado a moderadamente-baixo mas sempre deixando o malte dominar o sabor. Um leve sabor de lúpulo é opcional. A maioria dos exemplares tem uma doçura de malte na boca moderada, mas devem ter uma impressão atenuada no final. A impressão de dulçor vem da baixa quantidade de lúpulo, não de fermentação incompleta. Perfil de fermentação limpo. Versões escuras tem sabores de malte e ésteres similares ao aroma (mesmos descritores e intensidades). Versões claras tem sabor intenso de malte lembrando pão e tostado, um leve sabor de lúpulo picante, floral e/ou herbal e final mais seco.

\textbf{Sensações de Boca}: Corpo de médio-cheio a cheio. Carbonatação de moderada a moderadamente-baixa. Muito suave, sem adstringência. Leve aquecimento alcoólico pode ser percebido, mas não pode queimar.

\textbf{Comentários}: Doppelbock significa bock dupla. A maioria das versões são escuras e podem apresentar reações de Maillard e caramelização provenientes de mostura por decocção, mas também existem excelentes versões claras. As versões claras não terão a mesma riqueza e os sabores de maltes e frutas mais escuros das versões escuras e poderão ser um pouco mais secas, lupuladas e amargas. Embora a maioria dos exemplos tradicionais esteja no limite inferior das faixas citadas, o estilo pode ser considerado como não tendo limite superior de densidade e teor alcoólico, desde que o equilíbrio permaneça o mesmo.

\textbf{História}: Uma especialidade da Baviera originária de Munique. Produzida pela primeira vez por monges de São Francisco de Paula no séc. XVIII. Versões históricas eram menos atenuadas do que as interpretações modernas e, portanto, mais doces e com teor alcoólico menor. Era chamada de "pão líquido" pelos monges e era consumida durante o jejum da quaresma. Cervejarias adotaram os nomes das cervejas terminando em "-ator" depois que uma corte do século XIX julgou que ninguém além da Paulaner poderia utilizar o nome Salvator. Cor tradicionalmente castanho escuro; versões mais claras são um desenvolvimento mais recente.

\textbf{Ingredientes}: Maltes Pilsen, Vienna e Munich. Ocasionalmente malte escuro para ajuste de cor. Lúpulos tradicionais alemães. Levedura lager alemã limpa. Mostura por decocção é tradicional.

\textbf{Comparação de Estilos}: Uma versão mais forte, rica e com corpo mais cheio do que a Dunkles Bock ou Helles Bock. Versões claras apresentarão uma atenuação mais alta e menos caráter de fruta escura do que as versões mais escuras.

\textbf{Instruções para Inscrição}: O participante deve especificar se a amostra é \textit{clara} ou \textit{escura}.

\begin{tabular}{@{}p{35mm}p{35mm}@{}}
  \textbf{Estatísticas}: & OG: 1,072 - 1,112 \\
  IBU: 16 - 26 & FG: 1,016 - 1,024 \\
  SRM: 6 - 25 & ABV: 7\% - 10\%
\end{tabular}

\textbf{Exemplos Comerciais}: Versões escuras - Andechs Doppelbock Dunkel, Ayinger Celebrator, Paulaner Salvator, Spaten Optimator, Tröegs Troegenator, Weihenstephaner Korbinian.\\
Versões claras - Eggenberg Urbock 23º, Meinel Doppelbock Hell, Plank Bavarian Heller Doppelbock, Riegele Auri

\textbf{Última Revisão}: Doppelbock (2015)

\textbf{Atributos de Estilo}: amber-color, bock-family, bottom-fermented, central-europe, high-strength, lagered, malty, pale-color, traditional-style
