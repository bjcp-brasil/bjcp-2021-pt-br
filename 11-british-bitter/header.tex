\section*{11. British Bitter}
\addcontentsline{toc}{section}{11. British Bitter}
\textit{Após o final do século XIX, a família de Bitters britânicas surgiu a partir de pale ales inglesas, como um produto de barril (chope). O uso de maltes crystal em bitters tornou-se mais difundido após a Primeira Guerra Mundial. Tradicionalmente servida muito fresca, sem pressão (somente por densidade ou bomba manual) em temperaturas de adega (ou seja, “real ale”). A maioria das versões engarrafadas ou em barril de bitters produzidas no Reino Unido são frequentemente versões com maior teor alcoólico e maior carbonatação de produtos de barril produzidos para exportação e têm um caráter e equilíbrio diferentes do que seus equivalentes na Grã-Bretanha (muitas vezes sendo mais doces e menos lupuladas do que versões de barril). Essas diretrizes refletem a versão “real ale” do estilo, não as receitas de produtos comerciais para exportação. Existem muitas variações regionais de bitters, desde versões mais escuras e doces servidas quase sem colarinho até versões mais brilhantes, lupuladas e claras com espuma generosa e tudo o que há entre elas. Os juízes não devem enfatizar demais o componente caramelo desses estilos. Bitters exportadas podem estar oxidadas, o que aumenta os sabores de caramelo (assim como sabores mais negativos). Não assuma que os sabores derivados da oxidação são tradicionais ou necessários para o estilo.}