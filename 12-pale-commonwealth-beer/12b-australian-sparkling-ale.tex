\phantomsection
\subsection*{12B. Australian Sparkling Ale}
\addcontentsline{toc}{subsection}{12B. Australian Sparkling Ale}
\textbf{Impressão Geral}: Uma cerveja bem equilibrada, clara, altamente carbonatada e refrescante, adequada para beber em climas quentes. Consideravelmente amarga, com um perfil moderado de lúpulo herbal condimentado e ésteres, como maçã e pera. Sabores de malte suaves e neutros, encorpada, com um final bem definido e altamente atenuada.

\textbf{Aroma}: Aroma bastante suave e limpo com uma mistura equilibrada de ésteres, lúpulo, malte e levedura - todos de intensidade de moderada a baixa. Os ésteres são frequentemente peras e maçãs, opcionalmente com um leve toque de banana. Os lúpulos são terrosos, herbáceos ou podem mostrar característica do lúpulo Pride of Ringwood, semelhante ao ferro. O malte pode variar de grãos neutros a moderadamente doce a levemente semelhante a pão; nenhum caramelo deve ser evidente. Exemplos muito frescos podem ter um aroma leve de levedura e sulfuroso.

\textbf{Aparência}: Cor de amarelo profundo a âmbar claro, geralmente dourado médio. Colarinho branco espesso, espumoso e persistente, com bolhas pequenas. Efervescência perceptível devido à alta carbonatação. Limpidez brilhante, quando decantada, mas normalmente servida com levedura para ter uma aparência turva. Normalmente não é turva, a menos que a levedura seja servida durante o serviço.

\textbf{Sabor}: Sabor arredondado de malte de cereais a pão, de baixo a médio, inicialmente suave a adocicado de malte, mas com um amargor de médio a médio-alto, que aumenta no meio do palato para equilibrar o malte. Sabores de caramelo normalmente ausentes. Altamente atenuada, dando um final seco e bem definido com um amargor persistente, embora o corpo dê uma impressão de cheio. Sabor de lúpulo de médio a médio-alto, um pouco terroso e possivelmente herbal, resinoso, apimentado ou semelhante ao ferro, mas não floral, permanecendo no retrogosto. Ésteres de médio-alto a médio-baixo, geralmente peras e maçãs. A banana é opcional, mas nunca deve dominar. Pode ser levemente mineral ou sulfuroso, especialmente se a levedura estiver presente. Não deve ser sem graça.

\textbf{Sensação na Boca}: Carbonatação de alta a muito alta, com bolhas que enchem a boca e uma espirituosa picada carbônica, com final bem definido. Corpo de médio a médio-alto, tendendo para o lado mais alto se servido com levedura. Suave, mas gasoso. Versões mais fortes podem ter um leve aquecimento alcoólico, mas não as versões com menos álcool. Muito bem atenuada; não deve ter nenhuma doçura residual.

\textbf{Comentários}: A Coopers fabrica seu carro-chefe Sparkling Ale desde 1862, embora a formulação tenha mudado ao longo dos anos. Atualmente, a cerveja terá uma limpidez brilhante quando decantada, mas nos pubs geralmente se despejam a maior parte da cerveja em um copo, em seguida agitam a garrafa e despejam toda levedura. Em alguns bares, a garrafa é rolada ao longo da bancada. Quando servido em chope, a cervejaria instrui os bares a inverter o barril para despertar o fermento. Uma aparência turva para o estilo parece ser uma preferência do consumidor moderno. Sempre naturalmente carbonatada, mesmo no barril. Uma ale atual, melhor apreciada fresca.

\textbf{História}: Registros cervejeiros mostram que a maioria das cervejas australianas fabricadas no século 19 era chope XXX (Mild) e \textit{porter}. A Ale em garrafa foi originalmente desenvolvida para competir com as \textit{pale ales engarrafadas importadas de cervejarias britânicas, como Bass e Wm Younger' Monk. No início do século 20, a pale ale} engarrafada saiu de moda e as cervejas lager “mais leves” estavam em voga. Muitas Australian Sparkling e Pale Ales foram rotuladas como ales, mas na verdade eram lagers de baixa fermentação com perfil de malte muito semelhante às ales que substituíram. Coopers de Adelaide, Sul da Australia é a única cervejaria sobrevivente que produz o estilo Sparkling Ale.

\textbf{Ingredientes}: Malte pale australiano de 2 fileiras, levemente torrado, variedades pilsen podem ser usadas. Pequenas quantidades de malte crystal apenas para ajuste de cor. Os exemplos modernos não usam adjuntos, açúcar de cana apenas para \textit{priming}. Exemplos históricos usando 45\% de malte de 2 fileiras e 30\% de malte mais protéico (6 fileiras), usariam cerca de 25\% de açúcar para diluir o teor de nitrogênio. Lúpulos australianos tradicionalmente usados, Cluster e Goldings até serem substituídos em meados da década de 1960 pelo Pride of Ringwood. Levedura do tipo Burton altamente atenuante (cepa australiana típica). Perfil de água variável, tipicamente com baixo teor de carbonato e sulfato moderado.

\textbf{Comparação de Estilos}: Superficialmente semelhante às English Pale Ales, embora muito mais carbonatadas, com menos caramelo, menos lúpulos tardios e apresentando a assinatura da levedura e variedade de lúpulo. Mais amargor do que os IBU podem sugerir devido à alta atenuação, baixa densidade final e lúpulo um tanto áspero.

\begin{tabular}{@{}p{35mm}p{35mm}@{}}
  \textbf{Estatísticas}: & OG: 1,038 - 1,050 \\
  IBU: 20 - 35  & FG: 1,004 - 1,006 \\
  SRM: 4 - 7  & ABV: 4,5\% - 6\%
\end{tabular}

\textbf{Exemplos Comerciais}: Coopers Sparkling Ale.

\textbf{Última Revisão}: Australian Sparkling Ale (2015)

\textbf{Atributos de Estilo}: bitter, pacific, pale-ale-family, pale-color, standard-strength, top-fermented, traditional-style
