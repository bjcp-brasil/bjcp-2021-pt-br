\phantomsection
\subsection*{13C. English Porter}
\addcontentsline{toc}{subsection}{13C. English Porter}
\textit{Chamada simplesmente de "Porter" na Grã-Bretanha, o nome "English Porter" é usado para diferenciá-la de outras porters derivadas desse estilo, descritas nesse guia.}

\textbf{Impressão Geral}: Uma cerveja inglesa marrom escura, de teor alcoólico moderado, com um caráter amargo e torrado contido. Pode ter uma variedade de sabores torrados, geralmente sem qualidades de queimado e muitas vezes tem um perfil maltado de chocolate e caramelo.

\textbf{Aroma}: Aroma de moderado a moderadamente baixo de malte como de pão, biscoito e tostado, com leve torrado, muitas vezes como chocolate. Complexidade adicional de malte pode estar presente como caramelo, de nozes e dulçor de \textit{toffee}. Pode ter um nível moderado de lúpulo floral ou terroso. Ésteres frutados moderados são opcionais, mas desejáveis. Diacetil baixo é opcional.

\textbf{Aparência}: Cor de marrom a marrom escuro, muitas vezes com reflexos rubis. Boa limpidez, embora possa ser opaca. Colarinho moderado, de quase branco a castanho claro, com retenção de boa a razoável.

\textbf{Sabor}: Sabor moderado de malte como de pão, biscoito e tostado, com uma torra, de leve a moderada, como chocolate, muitas vezes com um caráter significativo de caramelo, de nozes ou \textit{toffee}, possivelmente com níveis mais baixos de sabores como café ou alcaçuz. Não deve ser queimado ou com um torrado áspero, embora pequenas quantidades possam contribuir para a complexidade de um chocolate amargo. Sabor de lúpulo terroso ou floral moderado é opcional. Ésteres frutados de baixo a moderado. Amargor de médio-baixo a médio com variação no equilíbrio de levemente maltado a levemente amargo, com um final bastante seco a levemente adocicado. Diacetil moderadamente baixo é opcional.

\textbf{Sensação na Boca}: Corpo de médio-leve a médio. Carbonatação de moderadamente baixa a moderadamente alta. Textura cremosa de leve a moderada.

\textbf{Comentários}: Esta descrição de estilo descreve a versão moderna da English Porter, e não a todas as variações possíveis ao longo do tempo em todas as regiões onde existiu. As recriações históricas devem ser inscritas na categoria 27 Historical Beer, com uma descrição apropriada, indicando o perfil da cerveja. Os exemplos artesanais modernos no Reino Unido têm teor alcoólico mais alto e são mais lupulados.

\textbf{História}: Originária de Londres no início de 1700, a Porter evoluiu como uma versão mais lupulada e envelhecida (de guarda) da Brown Beer popular na época. Evoluiu muitas vezes com base em vários desenvolvimentos tecnológicos e de ingredientes (como a invenção do malte Black em 1817 e a fabricação industrial de cerveja em larga escala), bem como preferências do consumidor, guerras e política tributária. Tornou-se um estilo altamente popular e amplamente exportado no início de 1800, antes de declinar na década de 1870, quando mudou para uma cerveja não envelhecida e de menor densidade. Como as densidades continuaram a diminuir em todas as cervejas do Reino Unido na primeira metade da década de 1900, estilos deixaram de ser feitos (incluindo a \textit{porter}, que desapareceu na década 1950). A era da cerveja artesanal levou à sua reintrodução em 1978. Diz-se que o nome foi derivado de sua popularidade com a classe trabalhadora de Londres realizando várias tarefas diárias de transporte de carga. Progenitora várias interpretações regionais ao longo do tempo e precursora de todas as \textit{stouts} (que originalmente eram chamadas de \textit{stout porters}). Não há conexão histórica ou relação entre Mild e Porter.

\textbf{Ingredientes}: Os maltes variam, mas sempre está envolvido algum que produza uma cor escura. Chocolate ou outros maltes torrados, malte caramelo, açúcares cervejeiros e similares são comuns. Porters de Londres costumam usar malte Brown como sabor característico.

\textbf{Comparação de Estilos}: Difere da American Porter porque geralmente tem sabores mais suaves, mais doces e mais caramelados, densidades mais baixas e geralmente menos álcool; a American Porter também costuma ter mais caráter de lúpulo. Com mais peso e torra do que uma British Brown Ale. Maior em densidade do que um Dark Mild.

\begin{tabular}{@{}p{35mm}p{35mm}@{}}
  \textbf{Estatísticas}: & OG: 1,040 - 1,052 \\
  IBU: 18 - 35  & FG: 1,008 - 1,014  \\
  SRM: 20 - 30  & ABV: 4\% - 5,4\%
\end{tabular}

\textbf{Exemplos Comerciais}: Bateman’s Salem Porter, Burton Bridge Burton Porter, Fuller's London Porter, Nethergate Old Growler Porter, RCH Old Slug Porter, Samuel Smith Taddy Porter.

\textbf{Última Revisão}: English Porter (2015)

\textbf{Atributos de Estilo}: british-isles, dark-color, malty, porter-family, roasty, standard-strength, top-fermented, traditional-style
