\phantomsection
\subsection*{16C. Tropical Stout}
\addcontentsline{toc}{subsection}{16C. Tropical Stout}
\textbf{Impressão Geral}: Uma stout muito escura, doce, frutada, com teor alcoólico moderadamente alto, com sabores suaves e torrados, mas sem aspereza de queimado.

\textbf{Aroma}: Dulçor de moderada a alta intensidade é proeminente. Aroma moderado a alto de café, chocolate ou torrado, mas não queimado. Frutado médio a alto. Pode ter aroma de melaço, alcaçuz, açúcar queimado, frutas secas ou vínico. Versões com teor alcoólico mais alto podem ter um aroma sutil e limpo de álcool. Baixo aroma de lúpulo opcional. Diacetil baixo opcional.

\textbf{Aparência}: Cor marrom muito profundo a preto. Limpidez geralmente obscurecida pela cor escura. Límpida, se não for opaca. Colarinho volumoso castanho a marrom, com boa retenção.

\textbf{Sabor}: Bastante doce com sabores suaves de cereais escuros e amargor contido, médio-baixo a médio. Sabor suave e torrado, muitas vezes como café ou chocolate, embora moderado no equilíbrio pelo final doce. Sem sabor de malte queimado ou aspereza no final. Ésteres frutados moderados a altos. Pode ter uma qualidade doce, de rum escuro, melaço ou açúcar queimado. Baixo sabor de lúpulo opcional. Diacetil médio-baixo opcional.

\textbf{Sensação na Boca}: Corpo médio-cheio a cheio, muitas vezes com um caráter suave e cremoso. Pode ter um aquecimento alcoólico, mas não pode queimar. Carbonatação moderada a moderadamente alta.

\textbf{Comentários}: Surpreendentemente refrescante em um clima quente. Os níveis de doçura podem variar significativamente. Tropical implica que a cerveja se originou e é popular nos trópicos, não que tenha características de frutas tropicais oriundas de lúpulo ou de frutas.

\textbf{História}: Uma adaptação local das Foreign Extra Stouts fabricada com ingredientes e métodos regionais no Caribe e em outros mercados tropicais. Amargor mais baixo do que em \textit{stouts} tipo exportação, pois essas cervejas não precisam ser enviadas para o exterior e atendem às preferências do paladar local.

\textbf{Ingredientes}: Semelhante a uma Sweet Stout, mas com maior densidade. Malte pale, maltes torrados escuros e cereais. Lúpulo principalmente para amargor. Pode usar adjuntos e açúcar para aumentar a densidade. Normalmente feita com levedura lager fermentada a quente.

\textbf{Comparação de Estilos}: Tem gosto de uma Sweet Stout escalonada com maior frutado. Semelhante a algumas Imperial Stouts sem o alto amargor, torra forte ou queimado, e lúpulos tardios e com menor teor alcoólico. Muito mais doce e menos lupulada do que American Stouts. Muito mais doce e menos amarga do que as Foreign Extra Stouts de densidade semelhante.

\begin{tabular}{@{}p{35mm}p{35mm}@{}}
  \textbf{Estatísticas}: & OG: 1,056 - 1,075 \\
  IBU: 30 - 50  & FG: 1,010 - 1,018  \\
  SRM: 30 - 40  & ABV: 5,5\% - 8\%
\end{tabular}

\textbf{Exemplos Comerciais}: ABC Extra Stout, Bahamian Strong Back Stout, Dragon Stout, Jamaica Stout, Lion Stout, Royal Extra Stout.

\textbf{Última Revisão}: Tropical Stout (2015)

\textbf{Atributos de Estilo}: british-isles, dark-color, high-strength, malty, roasty, stout-family, sweet, top-fermented, traditional-style
