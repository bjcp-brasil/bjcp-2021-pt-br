\phantomsection
\subsection*{17A. British Strong Ale}
\addcontentsline{toc}{subsection}{17A. British Strong Ale}
\textbf{Impressão Geral}: Uma cerveja de teor alcoólico respeitável, tradicionalmente acondicionada em garrafa e guardada em adega. Pode ter uma ampla gama de interpretações, mas a maioria terá graus variados de riqueza maltada, lúpulo de adição tardia, amargor, ésteres frutados e aquecimento alcoólico. Os sabores e intensidade do malte e dos adjuntos podem variar muito, mas qualquer combinação deve resultar em uma experiência agradável ao paladar.

\textbf{Aroma}: Adocicado de malte com ésteres frutados, muitas vezes com uma mistura complexa de frutas secas, caramelo, de nozes, \textit{toffee} ou outros aromas de maltes especiais. Algumas notas de álcool são aceitáveis, mas não devem ser quentes ou remetendo a solvente. Os aromas de lúpulo podem variar muito, mas normalmente têm notas terrosas, resinosas, frutadas ou florais. O equilíbrio pode variar muito, mas a maioria dos exemplos terá uma mistura de malte, frutas, lúpulo e álcool em intensidades variadas.

\textbf{Aparência}: Cor de âmbar a marrom escuro avermelhado; muitas são bastante escuras. Geralmente límpida, embora as versões mais escuras possam ser quase opacas. Colarinho de baixo a moderado, com cor de creme a castanho claro, com retenção média.

\textbf{Sabor}: Caráter de malte de médio a alto, muitas vezes rico com sabores de nozes, \textit{toffee} ou caramelo. Notas leves de chocolate às vezes são encontradas em cervejas mais escuras. Pode ter uma complexidade de sabor interessante de açúcares cervejeiros. O equilíbrio geralmente é maltado, mas pode ser bem lupulado, o que afeta a impressão de malte. Ésteres frutados moderados são comuns, muitas vezes com um caráter de frutas escuras ou de frutas secas. O final pode variar de meio seco a um pouco doce. A força alcoólica deve ser evidente, mas não esmagadora. Diacetil baixo é opcional, mas geralmente não é desejável.

\textbf{Sensação na Boca}: Corpo de médio a cheio, denso (com dextrinas). O aquecimento alcoólico é muitas vezes evidente e sempre bem-vindo. Carbonatação de baixa a moderada. Textura suave.

\textbf{Comentários}: Mais uma categoria de inscrição do que um estilo; a força e o caráter dos exemplos podem variar muito. Encaixa-se no espaço de estilo entre cervejas de densidade normal e as Barley Wines. Pode incluir cervejas maltadas e lupuladas claras, cervejas inglesas mais alcoólicas de inverno, \textit{dark milds} fortes, Burton Ales fracas e outras cervejas únicas na faixa de densidade geral que não se encaixam em outras categorias. Os juízes devem permitir uma variação significativa no caráter, desde que a cerveja esteja dentro da faixa de teor alcoólico e tenha um caráter 'britânico' interessante, provavelmente se encaixa no estilo.

\textbf{História}: Uma coleção de estilos menores não relacionados, cada um com sua própria herança. Não use este agrupamento de categorias para inferir uma relação histórica entre exemplos – nenhuma está prevista. Esta é uma categoria de julgamento de especialidade britânica moderna, onde o atributo 'especial' é o nível de álcool.

\textbf{Ingredientes}: O perfil de malte varia, muitas vezes baseados em maltes pale com caramelo e maltes especiais. Alguns exemplos mais escuros sugerem um uso leve de maltes escuros (por exemplo, chocolate, malte black). Adjuntos açucarados e amiláceos (por exemplo, milho, cevada em flocos, trigo) são comuns. Os lúpulos de finalização são tradicionalmente ingleses.

\textbf{Comparação de Estilos}: Significativa sobreposição de densidade com Old Ale, mas sem um caráter envelhecido. Uma ampla gama de interpretações é possível. Não deve ser tão rico ou forte quanto uma English Barley Wine. Mais potente do que uma Strong Bitter, British Brown Ale e English Porter fortes do dia a dia. Mais malte especial ou caráter de açúcar do que American Strong Ale.

\begin{tabular}{@{}p{35mm}p{35mm}@{}}
  \textbf{Estatísticas}: & OG: 1,055 - 1,080 \\
  IBU: 30 - 60  & FG: 1,015 - 1,022  \\
  SRM: 8 - 22  & ABV: 5,5\% - 8\%
\end{tabular}

\textbf{Exemplos Comerciais}: Fuller’s 1845, Harvey’s Elizabethan Ale, J.W. Lees Moonraker, McEwan’s Champion, Samuel Smith’s Winter Welcome, Shepherd Neame 1698.

\textbf{Última Revisão}: British Strong Ale (2015)

\textbf{Atributos de Estilo}: amber-color, british-isles, high-strength, malty, strong-ale-family, top-fermented, traditional-style
