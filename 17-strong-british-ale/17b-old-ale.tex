\phantomsection
\subsection*{17B. Old Ale}
\addcontentsline{toc}{subsection}{17B. Old Ale}
\textbf{Impressão Geral}: Uma cerveja inglesa mais forte que a média, embora geralmente não tão forte ou rica quanto uma English Barley Wine, mas geralmente maltada. Com aquecimento alcoólico. Mostra os efeitos positivos da maturação de uma cerveja envelhecida bem guardada.

\textbf{Aroma}: Adocicado de malte com ésteres frutados, muitas vezes com uma mistura complexa de frutas secas, vínico, caramelo, melaço, \textit{toffee}, melaço claro (light treacle) ou outros aromas de maltes especiais. Algumas notas oxidadas de álcool e de nozes são aceitáveis, semelhantes às encontradas em vinhos Xerez, Porto ou Madeira. Aroma de lúpulo geralmente não está presente.

\textbf{Aparência}: Cor de âmbar profundo a marrom muito escuro avermelhado, mas a maioria é bastante escura. A idade e a oxidação podem escurecer ainda mais a cerveja. Límpida, mas pode ser quase opaca. Colarinho de baixo a moderado, com cor de creme a castanho claro e retenção de média a ruim.

\textbf{Sabor}: Caráter de malte de médio a alto, com uma complexidade de malte deliciosa, muitas vezes com sabores de nozes, caramelo ou melaço. Sabores leves de chocolate ou de malte torrado são opcionais, mas nunca devem ser proeminentes. Equilíbrio é frequentemente adociado do malte, mas pode ser bem lupulado; a impressão de amargor muitas vezes depende da quantidade de envelhecimento. Ésteres frutados de moderados a altos são comuns e podem assumir um caráter de frutas secas ou vínicas. O final pode variar de seco a um pouco doce. O envelhecimento prolongado pode contribuir com sabores oxidados semelhantes a um bom vinho Xerez, Porto ou Madeira. A força alcoólica deve ser evidente, embora não esmagadora. Diacetil baixo é opcional.

\textbf{Sensação na Boca}: Corpo de médio a cheio, denso (com dextrinas), embora exemplos mais antigos possam ter corpo mais baixo devido à atenuação contínua durante a maturação. O aquecimento alcoólico é muitas vezes evidente e sempre bem-vindo. Carbonatação de baixa a moderada, dependendo da idade e acondicionamento. Acidez leve pode estar presente, bem como algum tanino se envelhecido em madeira; ambos são opcionais.

\textbf{Comentários}: A força e o caráter variam muito. A qualidade definidora predominante para este estilo é a impressão de idade, que pode se manifestar de diferentes maneiras (complexidade, oxidação, couro, qualidades vínicas etc). Muitas dessas qualidades são, em outros momentos, falhas, mas se o caráter resultante da cerveja for agradavelmente fácil de beber e complexo, então essas características são aceitáveis. De forma alguma essas características permitidas devem ser interpretadas como aceitáveis para fazer uma cerveja com sabor desagradável e difícil de beber. Old Peculier é uma cerveja bem conhecida, mas bastante única, que é bem diferente de outras Old Ales.

\textbf{História}: Historicamente, uma cerveja envelhecida usada como cerveja de estoque para mistura ou apreciada com força total (envelhecida ou de estoque são as cervejas que foram envelhecidas ou armazenadas por um período significativo). Hoje, existem pelo menos dois tipos definidos na Grã-Bretanha, cervejas de barril fracas e não envelhecidas, que são semelhantes às Milds, de cerca de 4,5\%, e as envelhecidas mais fortes, que geralmente têm 6-8\% ou mais.

\textbf{Ingredientes}: A composição varia, embora geralmente semelhante a British Strong Ales. O caráter de idade é o maior direcionador do perfil de estilo final, que é mais tratamento do que de fabricação.

\textbf{Comparação de Estilos}: Aproximadamente sobrepondo a British Strong Ale e a extremidade inferior da English Barley Wine, mas sempre com uma qualidade envelhecida. A distinção entre uma Old Ale e uma Barley Wine é um tanto arbitrária acima de 7\% ABV e, geralmente, significa ter uma qualidade envelhecida mais significativa.

\begin{tabular}{@{}p{35mm}p{35mm}@{}}
  \textbf{Estatísticas}: & OG: 1,055 - 1,088 \\
  IBU: 30 - 60  & FG: 1,015 - 1,022  \\
  SRM: 10 - 22  & ABV: 5,5\% - 9\%
\end{tabular}

\textbf{Exemplos Comerciais}: Avery Old Jubilation, Berlina Old Ale, Burton Bridge Olde Expensive, Gale’s Prize Old Ale, Greene King Strong Suffolk Ale, Marston Owd Roger, Theakston Old Peculier.

\textbf{Última Revisão}: Old Ale (2015)

\textbf{Atributos de Estilo}: aged, amber-color, british-isles, high-strength, malty, strong-ale-family, top-fermented, traditional-style