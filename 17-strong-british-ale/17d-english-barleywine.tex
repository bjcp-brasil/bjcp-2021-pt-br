\phantomsection
\subsection*{17D. English Barley wine}
\addcontentsline{toc}{subsection}{17D. English Barley wine}
\textbf{Impressão Geral}: Uma ale forte e ricamente maltada com uma agradável profundidade frutada ou lupulada. Uma cerveja para ser degustada no inverno, com um corpo cheio, denso (com dextrinas), e com aquecimento alcoólico.

\textbf{Aroma}: Muito rico, fortemente maltado, muitas vezes com aroma de caramelo nas versões mais escuras ou um leve caráter de \textit{toffee} nas versões mais claras. Pode ter um caráter rico, incluindo notas de pão, tostado ou \textit{toffee}. Pode ter frutado de moderado a forte, muitas vezes com caráter de frutas escuras ou secas, principalmente nas versões escuras. O aroma de lúpulo pode variar de suave a assertivo e é tipicamente floral, terroso, semelhante a chá ou de marmelada. O aroma de álcool pode ser de baixo a moderado, mas é suave e arredondado. A intensidade aromática diminui com a idade e pode desenvolver uma qualidade como Xerez, vinho ou Porto.

\textbf{Aparência}: Cor variando de âmbar dourado a marrom escuro, muitas vezes com reflexos rubi e profundidade de cor significativa. Não deve ser preta ou opaca. Colarinho quase branco de baixo a moderado. Pode ter baixa retenção de colarinho. Limpidez brilhante, principalmente quando envelhecida, embora as versões mais jovens possam ter um pouco de turbidez. O alto teor alcoólico e a viscosidade podem ser visíveis como lágrimas.

\textbf{Sabor}: Dulçor de médio a altamente rico, maltada, muitas vezes complexa e com multicamadas, com sabores de pão, biscoito e malte caramelo (mais como \textit{toffee} em versões mais claras) e com um frutado de médio a alto (geralmente com aspectos de frutas escuras ou secas). Quando envelhecidos, esses componentes frutados se destacam mais, e as versões mais escuras terão um nível mais alto do que as mais claras. O aroma, sabor e amargor do lúpulo podem variar muito. Lúpulos de leves a fortes, com um caráter inglês (floral, terroso, de chá ou de marmelada) são comuns. O amargor pode ser de leve a bastante forte, desaparecendo com o tempo, então o equilíbrio pode ser de maltado a um pouco amargo. Versões com teor alcoólico mais alto terão um pouco de caráter alcoólico. O final e o retrogosto podem ser de moderadamente secos a moderadamente doces, muitas vezes dependendo da idade. As versões envelhecidas podem desenvolver sabores oxidados ou vínicos complexos em um nível perceptível, mas não proeminente. As versões claras geralmente parecem mais amargas, mais atenuadas e mais direcionadas para o lúpulo do que as versões mais escuras.

\textbf{Sensação na Boca}: Encorpada e densa (com dextrinas), com uma textura aveludada e deliciosa, diminuindo com a idade. Um aquecimento suave do álcool envelhecido deve estar presente, mas não deve queimar. A carbonatação pode ser de baixa a moderada, dependendo da idade e do acondicionamento.

\textbf{Comentários}: A mais rica e com maior teor alcoólico das cervejas inglesas modernas. Seu caráter pode mudar significativamente ao longo do tempo; tanto as versões jovens quanto as antigas devem ser apreciadas pelo que são. O perfil do malte pode variar muito; nem todos os exemplos terão todos os sabores ou aromas possíveis. Variedades mais claras não terão os sabores de caramelo e malte mais ricos, nem os de frutas secas mais escuras – não espere sabores e aromas que são impossíveis para uma cerveja dessa cor. Tipicamente escrito como “Barley Wine” no Reino Unido e “Barleywine” nos EUA.

\textbf{História}: Uma descendente moderna das Burton Ales com maior teor alcoólico. Bass No. 1 foi chamada pela primeira vez de Barley Wine em 1872. Tradicionalmente uma cerveja mais escura até Tennant (agora Whitbread) produzir pela primeira vez a Gold Label, uma versão dourada, em 1951. O estilo original que inspirou variações derivadas na Bélgica, Estados Unidos e em outros lugares do mundo.

\textbf{Ingredientes}: Maltes pale ale britânicos e crystal. Uso limitado de maltes escuros. Muitas vezes usa açúcares cervejeiros. Lúpulo inglês. Levedura britânica.

\textbf{Comparação de Estilos}: Menos lupulada e amarga, mais maltada e frutada que uma American Barleywine. Pode sobrepor uma Old Ale na extremidade inferior do intervalo, mas sem maiores sinais de idade. Não tão caramelada e muitas vezes não tão doce quanto uma Wee Heavy.

\begin{tabular}{@{}p{35mm}p{35mm}@{}}
  \textbf{Estatísticas}: & OG: 1,080 - 1,120 \\
  IBU: 35 - 70  & FG: 1,018 - 1,030  \\
  SRM: 8 - 22  & ABV: 8\% - 12\%
\end{tabular}

\textbf{Exemplos Comerciais}: Burton Bridge Thomas Sykes Old Ale, Coniston No. 9 Barley Wine, Fuller’s Golden Pride, Hogs Back A over T, J.W. Lee’s Vintage Harvest Ale, Robinson’s Old Tom.

\textbf{Última Revisão}: English Barley wine (2015)


\textbf{Atributos de Estilo}: amber-color, british-isles, malty, strong-ale-family, top-fermented, traditional-style, very-high-strength
