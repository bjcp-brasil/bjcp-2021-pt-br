\phantomsection
\subsection*{19C. American Brown Ale}
\addcontentsline{toc}{subsection}{19C. American Brown Ale}
\textbf{Impressão Geral}: Uma Ale americana com teor alcoólico padrão, maltada, mas lupulada, frequentemente com sabores de chocolate e caramelo. O sabor e o aroma do lúpulo complementam e realçam o malte ao invés de conflitar com ele.

\textbf{Aroma}: Aroma adocicado maltado a rico maltado, com qualidades de chocolate, caramelo, de nozes ou tostado. O aroma de lúpulo é tipicamente de baixo a moderado, de quase qualquer tipo que complementa o malte. Algumas interpretações do estilo podem opcionalmente apresentar um aroma de lúpulo mais forte, um caráter de lúpulo americano ou do Novo Mundo (cítrico, frutado, tropical etc) ou um aroma de \textit{dry-hopping}. Os ésteres frutados são de muito baixos a moderados. O caráter de malte escuro é mais robusto do que de outras cervejas marrons, mas não chega a ser demasiado como em uma Porter.

\textbf{Aparência}: Cor de marrom claro a muito escuro. Límpida. Colarinho de baixo a moderado, com cor de quase branco a castanho claro.

\textbf{Sabor}: Sabor adocicado ou rico em malte, de médio a moderadamente alto, como chocolate, caramelo, nozes ou complexidade de malte tostado; amargor de médio a médio-alto. Final de médio a médio-seco, com retrogosto de malte e lúpulo. Sabor de lúpulo de leve a moderado, às vezes cítrico, frutado ou tropical, embora qualquer sabor de lúpulo que complemente o malte seja aceitável. Ésteres frutados de muito baixos a moderados. O malte e o lúpulo são geralmente iguais em intensidade, mas o equilíbrio pode variar em qualquer direção. Não deve ter um caráter torrado sugestivo de uma \textit{Porter} ou \textit{Stout}.

\textbf{Sensação na Boca}: Corpo de médio a médio-cheio. Versões mais amargas podem ter uma impressão seca e resinosa. Carbonatação de moderada a moderadamente alta. Versões mais fortes podem ter leve aquecimento alcoólico.

\textbf{Comentários}: A maioria das American Browns comerciais não são tão agressivas quanto as versões originais feitas em casa e alguns exemplos modernos de cerveja artesanal. Este estilo reflete as ofertas comerciais atuais normalmente comercializadas como American Brown Ales, em vez das versões caseiras mais lupuladas e fortes dos primórdios da fabricação de cerveja caseira. Essas \textit{brown ales} com intensidade de IPA devem ser registradas como 21B Specialty IPA: Brown IPA.

\textbf{História}: Um estilo americano do início da era moderna da cerveja artesanal. Derivada das English Brown Ales, mas com mais lúpulo. Pete's Wicked Ale (1986) definiu o estilo, que foi julgado pela primeira vez no Great American Beer Festival em 1992.

\textbf{Ingredientes}: Malte pale, crystal e escuros (tipicamente chocolate). Lúpulos americanos são típicos, mas lúpulos continentais ou do Novo Mundo também podem ser usados.

\textbf{Comparação de Estilos}: Mais sabores de chocolate e caramelo do que American Pale ou Amber Ales, normalmente com menos amargor proeminente no equilíbrio. Menos amarga, álcool e caráter de lúpulo do que Brown IPAs. Mais amargas e geralmente mais lupuladas que as English Brown Ales, com uma presença de malte mais rica, geralmente mais alto teor alcoólico e caráter de lúpulo americano ou do Novo Mundo.

\begin{tabular}{@{}p{35mm}p{35mm}@{}}
  \textbf{Estatísticas}: & OG: 1,045 - 1,060 \\
  IBU: 20 - 30  & FG: 1,010 - 1,016 \\
  SRM: 18 - 35  & ABV: 4,3\% - 6,2\%
\end{tabular}

\textbf{Exemplos Comerciais}: Avery Ellie’s Brown Ale, Big Sky Moose Drool Brown Ale, Brooklyn Brown Ale, Bell’s Best Brown, Smuttynose Old Brown Dog Ale, Telluride Face Down Brown.

\textbf{Última Revisão}: American Brown Ale (2015)

\textbf{Atributos de Estilo}: balanced, brown-ale-family, craft-style, dark-color, hoppy, north-america, standard-strength, top-fermented
