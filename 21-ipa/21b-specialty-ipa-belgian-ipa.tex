\phantomsection
\subsection*{21B. Specialty IPA: Belgian IPA}
\addcontentsline{toc}{subsection}{21B. Specialty IPA: Belgian IPA}
\textbf{Impressão Geral}: Uma IPA seca e lupulada, com frutado e picância provenientes de levedura belga. Muitas vezes de cor mais clara e mais atenuada, semelhante a uma Belgian Tripel que foi preparada com mais lúpulo.

\textbf{Aroma}: Aroma de lúpulo de moderado a alto, frequentemente refletindo o caráter de lúpulos americanos ou do novo mundo (tropical, melão, frutas de caroço, cítricos, pinho, etc.) ou lúpulos da europa continental (condimentados, herbais, florais, etc.), possivelmente com uma leve nota oriunda de \textit{dry-hop}. Dulçor suave do malte, às vezes com um caráter açucarado ou de mel, mas raramente de caramelo. Ésteres de moderados a altos, geralmente de frutas cítricas, pde era, de maçã ou de banana. Leve aroma de especiarias, cravo ou pimenta, é opcional. Leve aroma de álcool também é opcional.

\textbf{Aparência}: Cor de dourada clara a âmbar. Colarinho quase branco, de moderado a volumoso, com boa retenção. De boa impidez a bastante turva.

\textbf{Sabor}: Sabores frutados e especiarias moderados, mesmos descritores do aroma. Sabor de lúpulo de moderado a alto, mesmos descritores do aroma. Sabor de malte de leve a relativamente neutro, como cereais, opcionalmente com níveis baixos de tosta, caramelo ou mel. Amargor de moderado a alto. Final de seco a meio seco, que muitas vezes acentua a percepção de amargor. O retrogosto tem um amargor persistente, mas não é desagradável.

\textbf{Sensação na Boca}: Corpo de leve a médio. Carbonatação de média a alta, que pode aliviar a impressão do corpo. Aquecimento alcoólico leve é opcional.

\textbf{Comentários}: A escolha da cepa de levedura e das variedades de lúpulo é complexa, pois muitas opções irão se colidir.

\textbf{História}: Um estilo relativamente moderno, datado de meados dos anos 2000. Cervejeiros caseiros e cervejarias artesanais, em suas receitas de American IPA, substituíram a levedura por uma belga. Normalmente, as cervejarias belgas adicionavam mais lúpulo às suas cervejas claras com teor alcoólico mais alto.

\textbf{Ingredientes}: Cepas de leveduras belgas usadas na fabricação de Belgian Tripels e Golden Strong Ales. Exemplos americanos tendem a usar lúpulos americanos ou do novo mundo, enquanto as versões belgas tendem a usar lúpulos europeus e apenas malte claro. Adjuntos de açúcar são comuns.

\textbf{Comparação de Estilos}: Um cruzamento entre uma American IPA ou Double IPA com uma Belgian Golden Strong Ale ou uma Belgian Tripel. Este estilo pode ser condimentado, com teor alcoólico mais alto, mais seco e mais frutado do que uma American IPA.

\begin{tabular}{@{}p{35mm}p{35mm}@{}}
  \textbf{Estatísticas}: & OG: 1,058 - 1,080 \\
  IBU: 50 - 100 & FG: 1,008 - 1,016 \\
  SRM: 5 - 8  & ABV: 6,2\% - 9,5\%
\end{tabular}

\textbf{Exemplos Comerciais}: Brewery Vivant Triomphe, Houblon Chouffe, Green Flash Le Freak, Urthel Hop It.

\textbf{Última Revisão}: Specialty IPA: Belgian IPA (2015)

\textbf{Atributos de Estilo}: bitter, craft-style, high-strength, hoppy, ipa-family, north-america, pale-color, specialty-family, top-fermented
