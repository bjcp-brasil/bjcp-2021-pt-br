\phantomsection
\subsection*{21B. Specialty IPA: Brown IPA}
\addcontentsline{toc}{subsection}{21B. Specialty IPA: Brown IPA}
\textbf{Impressão Geral}: Uma cerveja lupulada, amarga e moderadamente forte como uma American IPA, mas com caramelo escuro, chocolate, \textit{toffee} ou caráter de frutas escuras como em uma American Brown Ale. Mantendo o final seco e o corpo leve que torna as IPAs tão fáceis de beber; uma Brown IPA é um pouco mais saborosa e maltada do que uma American IPA, sem ser doce ou pesada.

\textbf{Aroma}: Aroma de lúpulo de moderado a moderadamente alto, muitas vezes com um caráter de frutas de caroço, de frutas tropicais, de frutas cítricas, resinoso, de pinho, de frutas vermelhas ou de melão. O aroma adocicado de malte médio-baixo a médio combina bem com a seleção de lúpulo e geralmente se apresenta como notas de chocolate ao leite, cacau, \textit{toffee}, nozes, biscoitos, caramelo escuro, pão tostado ou caráter de frutas escuras. Perfil de fermentação limpo. Ésteres leves são opcionais. Aroma leve de álcool é opcional.

\textbf{Aparência}: Cor variando de marrom avermelhado a marrom escuro, mas sem ser preto. Límpida, quando não for opaca. Leve turbidez é opcional. Colarinho de tamanho médio, de creme a castanho, com boa persistência.

\textbf{Sabor}: Sabor de lúpulo de médio a alto, mesmos descritores do aroma. Suporte maltado limpo, de médio-baixo a médio, com os mesmos descritores do aroma. As escolhas de malte e lúpulo não devem produzir conflitos no sabor. Amargor de médio-alto a alto, sem aspereza. Final de seco a médio, com um retrogosto amargo, lupulado e maltado. Ésteres baixos são opcionais. Sabor de álcool muito baixo é opcional. Sem sabores de malte altamente torrados ou queimados. O malte deve quase equilibrar o amargor e o sabor do lúpulo.

\textbf{Sensação na Boca}: Corpo de médio-leve a médio, com uma textura suave. Carbonatação de média a média-alta. Sem aspereza. Leve aquecimento alcoólico é opcional.

\textbf{Comentários}: Destacada do estilo American Brown Ale para melhor diferenciar os exemplos com maior teor alcoólico e altamente lupulados das cervejas mais equilibradas e de teor alcoólico padrão.

\textbf{História}: Veja American Brown Ale.

\textbf{Ingredientes}: Semelhante a uma American IPA, mas com maltes crystal médio ou escuro, maltes tipo chocolate, levemente torrados, ou outros maltes de cor intermediária. Pode usar açúcares como adjuntos, incluindo açúcar mascavo. Qualquer caráter de lúpulo americano ou do Novo Mundo é aceitável, mas os lúpulos e o perfil maltado não devem entrar em conflito.

\textbf{Comparação de Estilos}: Uma versão mais forte e amarga de uma American Brown Ale, com o equilíbrio seco de uma American IPA. Tem menos sabor torrado que a Black IPA, mas com mais sabores de chocolate que uma Red IPA.

\begin{tabular}{@{}p{35mm}p{35mm}@{}}
  \textbf{Estatísticas}: & OG: 1,056 - 1,070 \\
  IBU: 40 - 70  & FG: 1,008 - 1,016 \\
  SRM: 18 - 35  & ABV: 5,5\% - 7,5\%
\end{tabular}

\textbf{Exemplos Comerciais}: Dogfish Head Indian Brown Ale, Harpoon Brown IPA, Russian River Janet’s Brown Ale.

\textbf{Última Revisão}: Specialty IPA: Brown IPA (2015)

\textbf{Atributos de Estilo}: bitter, craft-style, dark-color, high-strength, hoppy, ipa-family, north-america, specialty-family, top-fermented
