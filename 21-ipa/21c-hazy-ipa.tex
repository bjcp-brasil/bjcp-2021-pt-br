\phantomsection
\subsection*{21C. Hazy IPA}
\addcontentsline{toc}{subsection}{21C. Hazy IPA}
\textbf{Impressão Geral}: Uma American IPA com sabores e aromas intensos de frutas, corpo macio, sensação suave na boca, muitas vezes opaca, com substancial turbidez. Menor percepção de amargor do que as IPAs tradicionais, mas sempre massivamente orientada para o lúpulo.

\textbf{Aroma}: Aroma intenso de lúpulo, como frutas de caroço, de frutas tropicais, cítricas ou outras qualidades de frutas; não dever ser gramíneo ou herbal. Em segundo plano, o malte neutro, limpo, como cereais ou levemente tostado; sem caramelo ou torrado. Ausência de qualquer caráter de malte é uma falha. Perfil de fermentação de neutro a frutado. Ésteres de levedura e lúpulo não devem entrar em conflito. Um aroma de creme de baunilha, amanteigado ou ácido é inadequado. Aroma leve de álcool é opcional.

\textbf{Aparência}: Cor variando de palha a âmbar muito claro, às vezes com um tom alaranjado. Turva, muitas vezes um pouco opaca; não deve ser totalmente opaca. A opacidade pode adicionar um “brilho” à cerveja e fazer com que a cor pareça mais escura. Qualquer partícula flutuante visível de lúpulo, aglomerados de levedura ou outras partículas é uma falha. Colarinho branco, consistência média a sólida, como um merengue, com retenção de alta a muito alta.

\textbf{Sabor}: Sabor frutado de lúpulo de alto a muito alto, com os mesmos descritores do aroma. Sabor de malte de baixo a médio, os mesmos descritores do aroma. Amargor percebido de baixo a médio-alto, muitas vezes mascarado pelo corpo mais cheio e macio, final de meio seco a médio. O perfil do lúpulo no retrogosto não deve ser acentuado ou áspero. Perfil de fermentação de neutro a frutado, servindo de suporte ao lúpulo. Não deve ser doce, embora altos níveis de ésteres e menor amargor possam às vezes dar essa impressão. Sabor de álcool, em segundo plano, é opcional.

\textbf{Sensação na Boca}: Corpo de médio a médio-cheio. Carbonatação média. Macia. Sem aspereza. Leve aquecimento alcoólico é opcional. A cerveja não deve ter na boca uma sensação cremosa, viscosa, uma acidez persistente ou uma textura de amido cru.

\textbf{Comentários}: Também conhecido como New England IPA ou NEIPA. A ênfase nos lúpulos de adição tardia, especialmente no \textit{dry-hopping}, com lúpulos com qualidades de frutas tropicais, confere o caráter de “suco” pelo qual este estilo é conhecido. Exemplos pesados, que sugerem \textit{milkshakes}, \textit{smoothies} de frutas ou sorvete de creme, estão fora desse estilo; As IPAs devem ser sempre fáceis de beber. A turbidez vem do \textit{dry-hopping}, não de levedura em suspensão, da opacidade causada pelo amido ou de outras técnicas; um brilho turvo é desejável, mas não uma confusão irregular opaca.

\textbf{História}: Um estilo moderno de cerveja artesanal originário na região de New England nos Estados Unidos, como uma variante da American IPA. Acredita-se que a Alchemist Heady Topper seja a inspiração original, quando o estilo cresceu em popularidade na década de 2010. O estilo continua a evoluir, incluindo a tendência de diminuir o amargor e usar o estilo como base para outras adições.

\textbf{Ingredientes}: Receita similar a uma American IPA, mas com mais cereais em flocos e menos caramelo ou maltes especiais. Lúpulos americanos ou do novo mundo com características frutadas. Levedura de neutra a esterificada. Equilíbrio para água rica em cloreto. Uso elevado de \textit{dry-hopping}, em parte durante a fermentação ativa, usando uma variedade de doses de lúpulo e temperaturas para enfatizar uma profundidade de aromas e sabores do lúpulo ao invés do amargor. A biotransformação dos óleos essenciais do lúpulo durante a fermentação aumenta a intensidade e a complexidade frutada.

\textbf{Comparação de Estilos}: Tem uma sensação na boca mais cheia e macia, uma expressão de lúpulo mais frutada, um equilíbrio do amargor percebido mais contido e uma aparência mais opaca do que a American IPA. Muitas American IPAs modernas são frutadas e um tanto turvas; exemplares com final seco e nítido, no máximo corpo médio e alto amargor percebido, devem ser inseridos como 21A American IPA. Adições notáveis de frutas, lactose, baunilha, etc. para aumentar o caráter frutado e suave devem ser inseridas nas categorias de especialidades definida pelos aditivos (por exemplo, 29A Fruit Beer, 29C Specialty Fruit Beer, 30D Specialty Spice Beer).

\begin{tabular}{@{}p{35mm}p{35mm}@{}}
  \textbf{Estatísticas}: & OG: 1,060 - 1,085 \\
  IBU: 25 - 60  & FG: 1,010 - 1,015 \\
  SRM: 3 - 7  & ABV: 6\% - 9\%
\end{tabular}

\textbf{Exemplos Comerciais}: Belching Beaver Hazers Gonna Haze, Hill Farmstead Susan, Other Half Green Diamonds Double IPA, Pinthouse Electric Jellyfish, Tree House Julius, Trillium Congress Street, WeldWerks Juicy Bits.

\textbf{Atributos de Estilo}: bitter, craft-style, high-strength, hoppy, ipa-family, north-america, pale-color, top-fermented
