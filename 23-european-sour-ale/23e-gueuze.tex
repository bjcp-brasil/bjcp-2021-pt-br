\phantomsection
\subsection*{23E. Gueuze}
\addcontentsline{toc}{subsection}{23E. Gueuze}
\textbf{Impressão Geral}: Uma cerveja de trigo, selvagem, belga, muito refrescante, altamente carbonatada, agradavelmente ácida, mas equilibrada. O caráter selvagem da cerveja pode ser complexo e variado, combinando sabores ácidos, \textit{funky} e frutados.

\textbf{Aroma}: Moderadamente ácida com caráter \textit{funky} complexo, mas equilibrada, acentuada por notas frutadas. O caráter \textit{funky} pode ser de moderado a forte e pode ser descrito como celeiro, couro, terra, cabra, feno, cavalo ou manta de cavalo. O sabor frutado é de leve a moderado, com frutas cítricas, casca de frutas cítricas, frutas de pomar ou ruibarbo. O malte é apenas para suporte e pode ter um leve caráter de pão, cereais, mel ou trigo, quando perceptível. Não deve apresentar defeitos entéricos, defumados, semelhantes a charutos ou de queijo. Sem lúpulo. Leve carvalho é aceitável. A complexidade do aroma é mais valorizada do que a intensidade, mas uma apresentação ácida equilibrada é desejável.

\textbf{Aparência}: Cor dourada, com excelente limpidez e uma espuma branca, espessa, densa como mousse, que parece durar para sempre. Efervescente.

\textbf{Sabor}: Ácido e \textit{funky} no paladar, com um caráter similar ao do aroma (mesmos descritores e intensidades se aplicam ao caráter \textit{funky} e frutas). Malte baixo como pão e cereais. Amargor de baixo a nenhum; a acidez fornece a maior parte do equilíbrio. Sem sabor de lúpulo. Final seco e bem definido, com um retrogosto ácido e \textit{funky}. Carvalho leve, baunilha e mel são aceitáveis. Não deve apresentar defeitos entéricos, defumados, semelhantes a charutos ou de queijo. A cerveja não deve ser ácida de maneira unidimensional; uma apresentação equilibrada e moderadamente ácida é clássica, com as notas frutadas e \textit{funky} proporcionando complexidade. Pode ser envelhecida.

\textbf{Sensação na Boca}: Corpo baixo a médio-baixo; não deve ser aguada. Tem uma acidez de baixa a alta, trazendo a capacidade de franzir o rosto sem ser acentuadamente adstringente. Algumas versões têm um caráter de aquecimento muito leve. Altamente carbonatada.

\textbf{Comentários}: A mistura de Lambics jovens e envelhecidas cria um produto mais complexo e muitas vezes reflete o gosto pessoal de quem as mistura. Um caráter perceptível avinagrado ou de cidra é considerado um defeito pelos cervejeiros belgas. Uma boa Gueuze não é a mais pungente, mas possui um buquê cheio e tentador, um aroma acentuado e uma textura macia e aveludada. Lambic é servida sem gás, enquanto Gueuze é servida efervescente. Os produtos marcados como \textit{oude} ou \textit{vieile} (“antigo”) são considerados os mais tradicionais.

\textbf{História}: Mesma história básica da Lambic, mas envolve mistura, que pode ser feita fora da cervejaria. Alguns dos melhores exemplares são produzidos por mestres em blendar, que fermentam, envelhecem, misturam e envasam o produto final. Alguns produtores modernos estão adoçando seus produtos após a fermentação para torná-los mais palatáveis para um público mais amplo. Estas diretrizes descrevem o produto tradicional.

\textbf{Ingredientes}: Os mesmos da Lambic, exceto que Lambics de um, dois e três anos são misturadas e depois armazenadas.

\textbf{Comparação de Estilos}: Mais complexa e carbonatada que uma Lambic. A acidez não é necessariamente mais forte, mas tende a ter um caráter selvagem mais bem desenvolvido.

\begin{tabular}{@{}p{35mm}p{35mm}@{}}
  \textbf{Estatísticas}: & OG: 1,040 - 1,054 \\
  IBU: 0 - 10  & FG: 1,000 - 1,006  \\
  SRM: 5 - 6  & ABV: 5\% - 8\%
\end{tabular}

\textbf{Exemplos Comerciais}: 3 Fonteinen Oud Gueuze, Cantillon Classic Gueuze 100\% Lambic, Girardin Gueuze 1882 (Black label), Hanssens Oude Gueuze, Lindemans Gueuze Cuvée René, Oude Gueuze Boon.

\textbf{Última Revisão}: Gueuze (2015)

\textbf{Atributos de Estilo}: aged, high-strength, pale-color, sour, traditional-style, western-europe, wheat-beer-family, wild-fermented
