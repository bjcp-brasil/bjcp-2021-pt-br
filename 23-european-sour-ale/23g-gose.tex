\phantomsection
\subsection*{23G. Gose}
\addcontentsline{toc}{subsection}{23G. Gose}
\textbf{Impressão Geral}: Uma cerveja ácida da Europa Central, levemente amarga, com um caráter de sal e semente de coentro distinto, porém contido. Muito refrescante, com um final seco, alta carbonatação e sabores vivos.

\textbf{Aroma}: Aroma de leve a moderadamente frutado de frutas de pomo. Leve acidez, ligeiramente acentuada. Semente de coentro perceptível, o qual pode ter uma qualidade aromática que remete a limão e uma intensidade que pode chegar a moderada. Leve pão, massa de pão, caráter de levedura como pão de fermentação natural antes de ser assado. A acidez e as sementes de coentro podem fornecer uma impressão viva, vibrante. O sal, quando de alguma forma aparente, pode ser percebido como uma brisa do mar muito leve ou apenas um frescor indefinido.

\textbf{Aparência}: Não filtrada, com turbidez de moderada a alta. Colarinho branco de moderado a alto com bolhas pequenas e boa retenção. Efervescente. De coloração amarela.

\textbf{Sabor}: Acidez perceptível, de média-baixa a média-alta. Sabor moderado de malte como pão ou massa de mão. De leve a moderado caráter frutado de frutas de pomo, frutas de caroço ou limões. Caráter de sal de leve a moderado, podendo chegar até o limiar do sabor; o sal deve ser perceptível (particularmente na prova inicial) mas não ter um gosto notadamente salgado. Amargor muito baixo. Sem sabor de lúpulo. Seca, final completamente atenuado, com a acidez, não  os lúpulos, equilibrando o malte. A acidez pode ser mais evidente no final e aprimorar a qualidade refrescante da cerveja. A acidez deve ser equilibrada, não protagonista (apesar das versões históricas serem muito ácidas). Sem THP.

\textbf{Sensação na Boca}: Carbonatação de alta a muito alta. Efervescente. Corpo de médio-baixo a médio-alto. O sal pode passar uma leve sensação de formigamento, uma qualidade de dar água na boca e uma sensação na boca mais arrendondada e mais espessa.

\textbf{Comentários}: Versões históricas podem ter sido mais ácidas do que os exemplares modernos devido a fermentação espontânea e podem ser misturadas com xaropes, assim como é feito com a Berliner Weisse, ou com licor de alcaravia. Exemplares modernos são inoculados com Lacto, são mais equilibrados e geralmente não exigem adoçantes. Pronuncia-se GÔ-za.

\textbf{História}: Pequeno estilo associado com Leipzig mas originado na Idade Média na cidade de Goslar sob o rio Gose. Documentada como tendo aparecido em Leipzig por volta de 1740. Leipzig foi descrita como tendo 80 casas de Gose em 1900. A produção teve um declínio significante depois da Segunda Guerra Mundial e cessou completamente em 1966. A produção moderna foi revivida nos anos 80 na Alemanha, mas a cerveja não era amplamente acessível. Ficou popular fora da Alemanha recentemente como um renascimento do estilo e é frequentemente utilizada como um estilo base pra cervejas ácidas com frutas e outras cervejas do tipo Specialty.

\textbf{Ingredientes}: Malte de trigo e Pilsen, uso restrito de sal e sementes de coentro, Lacto. As sementes de coentro devem ter uma nota viva, fresca, cítrica (limão ou laranja-azeda) e não vegetal, como salsão ou tipo presunto. O sal deve ter um caráter de sal marinho ou sal fresco, não uma nota metálica ou de iodo.

\textbf{Comparação de Estilos}: A acidez percebida não é tão intensa como na Berliner Weisse ou na Gueuze. Uso contido de sal, sementes de coentro e Lacto - não deve ter um gosto notadamente salgado. O aroma de sementes de coentro pode ser similar a uma Witbier. Turbidez similar a uma Weissbier.

\begin{tabular}{@{}p{35mm}p{35mm}@{}}
  \textbf{Estatísticas}: & OG: 1,036 - 1,056 \\
  IBU: 5 - 12  & FG: 1,006 - 1,010  \\
  SRM: 3 - 4  & ABV: 4,2\% - 4,8\%
\end{tabular}

\textbf{Exemplos Comerciais}: Anderson Valley Gose, Bayerisch Bahnhof Leipziger Gose, Original Ritterguts Gose, Westbrook Gose.

\textbf{Última Revisão}: Historical Beer: Gose (2015)

\textbf{Atributos de Estilo}: central-europe, historical-style, pale-color, sour, spice, standard-strength, top-fermented, wheat-beer-family
