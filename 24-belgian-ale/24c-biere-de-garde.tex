\phantomsection
\subsection*{24C. Bière de Garde}
\addcontentsline{toc}{subsection}{24C. Bière de Garde}
\textit{Três variações principais estão incluídas na estilo: a clara (blonde), a marrom (brune) e a mais tradicional, âmbar (ambrée).}

\textbf{Impressão Geral}: Uma família de cervejas artesanais francesas suaves, com teor alcoólico razoável, maltadas e maturadas a frio (\textit{lagering}); com uma variedade de sabores de malte apropriados para a cor dourada (\textit{blonde}), âmbar (\textit{ambrée}) ou marrom (\textit{brune}). Todas são maltadas, contudo, são secas e com sabores limpos. Versões mais escuras têm mais caráter de malte, enquanto as versões mais claras podem ter mais lúpulo, embora ainda sejam cervejas focadas no malte.

\textbf{Aroma}: Proeminente riqueza de malte, muitas vezes com intensidade complexa, de leve a moderada, com caráter tostado e de pão. Ésteres de baixos a moderados. Lúpulos pouco condimentados, apimentados ou herbáceos são opcionais. Geralmente bastante limpa, embora as versões com teor alcoólico mais alto possam ter uma leve nota de álcool picante à medida que aquece. As versões mais claras ainda são maltadas, mas carecem de aromas mais ricos e profundos e podem ter um pouco mais de lúpulo.

\textbf{Aparência}: Existem variações de dourado, âmbar e marrom, com variação de cor: de dourado claro a bronze avermelhado, até marrom acastanhado. Limpidez é brilhante a regular, mas alguma turbidez é permitida. Colarinho bem formado, geralmente branco a quase branco (variando com a cor da cerveja), média persistência.

\textbf{Sabor}: Riqueza maltada de média a alta, muitas vezes com caráter de tostado, de biscoito, \textit{toffee} ou leve caramelo. Ésteres e sabores alcoólicos de baixos a moderados. Amargor de lúpulo médio-baixo, trazendo um equilíbrio maltado ao paladar e retrogosto. Final de meio seco a seco, não pode ser doce, enjoativo ou pesado. Sabor de lúpulo picante, condimentado e/ou herbal é opcional. Sabor, profundidade, riqueza, intensidade e complexidade do malte aumentam com a cor da cerveja. As versões mais escuras terão uma impressão inicial mais rica de malte do que as versões mais claras, mas não devem parecer torradas. Versões mais claras podem ter sabor de lúpulo ligeiramente maior.

\textbf{Sensação na Boca}: Corpo médio a médio-leve, geralmente com um caráter suave, sedoso e cremoso. Carbonatação moderada a alta. Aquecimento alcoólico moderado, mas nunca deve ser quente.

\textbf{Comentários}: Adega, mofo, bolor ou caráter rústico frequentemente mencionados na literatura são sinais de amostras importadas em condições não favoráveis e não de produtos frescos e autênticos. A idade e a oxidação também podem aumentar o sabor frutado e os sabores de caramelo, mas aumentam a aspereza. Embora caramelo e frutas possam fazer parte do estilo, não se deve confundir esse caráter de oxidação com a cerveja base propriamente.

\textbf{História}: O nome significa, de um modo geral, cerveja para guardar. Uma tradicional cerveja artesanal de áreas rurais (\textit{farmhouse}) ao redor de Lille, no norte da França. Historicamente, produzida no início da primavera e mantida em adegas frias para consumo em dias mais quentes. Embora documentado como existente em 1800, Jenlain é a versão prototípica da cerveja âmbar moderna engarrafada pela primeira vez na década de 1940.

\textbf{Ingredientes}: Os maltes base variam de acordo com a cor da cerveja, mas geralmente incluem os tipos \textit{pale}, Vienna e Munich. Maltes do tipo crystal de cores variadas. Adjuntos de açúcar podem ser usados. Leveduras lager ou ale fermentadas em temperaturas baixas, seguidas de longo condicionamento a frio. Lúpulo continental.

\textbf{Comparação de Estilos}: Chamar esta cerveja de \textit{farmhouse} convida a comparações com a Saison, que tem um equilíbrio completamente diferente - a Bière de Garde é maltada e suave, enquanto a Saison é lupulada e amarga. Na verdade, tem mais semelhança no perfil de malte com uma Bock.

\textbf{Instruções para Inscrição}: O participante deve especificar Bière de Garde clara (blonde), âmbar (ambrée) ou escura (brune). Se nenhuma cor for especificada, o juiz deve tentar julgar com base na observação inicial, esperando um sabor e equilíbrio de malte que correspondam à cor.

\begin{tabular}{@{}p{35mm}p{35mm}@{}}
  \textbf{Estatísticas}: & OG: 1,060 - 1,080 \\
  IBU: 18 - 28  & FG: 1,008 - 1,016  \\
  SRM: 6 - 19  & ABV: 6\% - 8,5\%
\end{tabular}

\textbf{Exemplos Comerciais}: Ch’Ti Blonde, Jenlain Ambrée, La Choulette Brune, Russian River Perdition, Saint Sylvestre 3 Monts Blonde, Two Brothers Domaine Dupage.

\textbf{Última Revisão}: Bière de Garde (2015)

\textbf{Atributos de Estilo}: amber-ale-family, amber-color, any-fermentation, high-strength, lagered, malty, pale-color, traditional-style, western-europe
