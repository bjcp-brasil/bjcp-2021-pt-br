\phantomsection
\subsection*{25B. Saison}
\addcontentsline{toc}{subsection}{25B. Saison}
\textbf{Impressão Geral}: Uma família de \textit{ales} belgas refrescantes, altamente atenuadas, lupuladas e razoavelmente amargas, com final muito seco e carbonatação alta. Caracterizadas por um perfil de fermentação frutado, condimentado e às vezes fenólico, com uso de cereais e às vezes condimentos/especiarias para complexidade. Existe muita variação de teor alcoólico e de cor.

\textbf{Aparência}: Cor de dourado claro a âmbar profundo, algumas vezes laranja claro. Colarinho de longa duração, denso, com cor de branco a marfim. Rendado belga. A cerveja não é filtrada então a limpidez é variável (de baixa a boa) e pode ser turva. Efervescente.
Versões mais escuras podem ter cor de cobre a marrom escura. Versões com maior teor alcoólico podem ser de cor mais escura.

\textbf{Aroma}: Uma agradável combinação de frutado-condimentado de levedura e lúpulos. Os ésteres frutados têm intensidade de moderada a alta e muitas vezes têm um frutado cítrico com caráter de frutas de pomo ou drupas. Nota condimenta de baixa a moderadamente-alta, muitas vezes como pimenta-do-reino, não como cravo. Lúpulo de baixo a moderado com caráter continental (condimentado, floral, terroso ou frutado). O malte é muitas vezes ofuscado, mas se for detectado, é leve como cereais. Condimentos/especiarias e ervas são opcionais, mas não devem dominar. Acidez é opcional (veja os Comentários).
Versões mais alcoólicas podem ter mais intensidade de aroma e podem apresentar um leve caráter alcoólico e caráter de malte moderado. Versões com teor alcoólico leve são menos intensas e não têm um caráter alcoólico. Versões mais escuras têm mais caráter de malte associado com cereais escuros.

\textbf{Sabor}: Equilíbrio entre frutado e condimentado de levedura, amargor lupulado e maltado como cereais, com um amargor de moderado a alto e um final muito seco. O aspecto frutado e condimentado é de médio-baixo a médio-alto e o sabor de lúpulo é de baixo a médio, ambos com características similares ao aroma (os mesmos descritores se aplicam aqui). O malte é de baixo a médio com um sabor suave de cereais. Atenuação muito alta, nunca com um final doce ou pesado. Retrogosto amargo e condimentado. Condimento/especiarias e ervas são opcionais, mas se usados devem estar em harmonia com o caráter de levedura. Acidez é opcional (vejas os Comentários).
Versões mais escuras vão ter mais caráter de malte, inclusive sabores dos maltes escuros. Versões com teor alcoólico mais alto apresentam intensidade de malte maior e uma leve nota alcoólica.

\textbf{Sensação na Boca}: Corpo de leve a médio-cheio. Carbonatação muito alta. Efervescente. Leve aquecimento alcoólico é opcional. Acidez é rara, mas opcional (veja os Comentários).

\textbf{Comentários}: O estilo geralmente descreve a versão clara de teor alcoólico padrão seguido pelas diferentes variações de teor alcoólico e cor. Versões mais escuras tendem a ter mais caráter de malte e menos amargor de lúpulo aparente, gerando uma apresentação mais equilibrada. Versões com teor alcoólico mais alto muitas vezes têm mais sabor e caráter rico de malte, aquecimento e corpo devido a sua densidade mais alta. Não há correlação entre teor alcoólico e cor.
Acidez é totalmente opcional, e se presente em níveis de baixo a moderado, pode substituir de alguma forma o amargor no equilíbrio. Uma Saison não deve ser simultaneamente ácida e amarga. A atenuação alta pode fazer a cerveja parecer mais amarga do que sugerido pelo IBU. As versões claras muitas vezes são mais amargas e lupuladas do que as versões escuras. A seleção de levedura muitas vezes é responsável pelas notas frutadas e condimentadas e pode mudar significativamente o caráter; permita uma variedade de interpretações.
Muitas vezes chamadas de \textit{ales} Farmhouse nos Estados Unidos, este termo não é comum na Europa onde elas simplesmente fazem parte de um grupo maior de \textit{ales} artesanais. Brettanomyces não é típico para esse estilo; Saisons com Brett devem ser inscritas no estilo 28A Brett Beer. Uma Grisette é um tipo bem conhecido de Saison, popular entre mineradores; inscreva-a como 25B Saison, Teor alcoólico leve, Comentário: Grisette com trigo como cereal característico.

\textbf{História}: Uma \textit{provision ale} (termo usado para descrever cervejas que podiam ser guardadas por um longo tempo) da Valônia, região da Belgica em que o idioma é o francês. Originalmente era um produto de baixo teor alcoólico que visava não debilitar agricultores e trabalhadores do campo, ainda assim existiam produtos de tavernas (de teor alcoólico mais alto). Saison Dupont, a saison moderna mais conhecida, foi produzida pela primeira vez na década de 1920. A Super Saison da Dupont foi produzida pela primeira vez em 1954 e sua versão escura em meados de 1980. A Fantôme começou a produzir suas saisons “sazonais” em 1988. Ainda que o estilo mantenha sua imagem rústica, ele é na sua grande maioria feito em grandes cervejarias.

\textbf{Ingredientes}: Malte base pale. Cereais como trigo, aveia, espelta ou centeio. Pode ter adjuntos a base de açúcar. Lúpulos continentais. Levedura belga condimentada-frutada para saison. Condimentos/especiarias e ervas não são comuns, mas são permitidos se não forem dominantes.

\textbf{Comparação de Estilos}: A versão clara, com teor alcoólico padrão é como uma Belgian Blond Ale mais altamente atenuada, lupulada e amarga, com um caráter de levedura mais forte. Com teor alcoólico super e cor clara, é similar a uma Belgian Tripel, mas muitas vezes com uma qualidade mais rústica e de cereais e às vezes com um caráter de levedura mais condimentado.

\textbf{Instruções para Inscrição}: O participante deve especificar o teor alcoólico (leve, padrão, super) e a cor (clara, escura). O participante pode identificar os cereais usados.

\textbf{Estatísticas}:\\
IBU: 20 - 35\\
SRM: 5 – 14 (clara)
\begin{adjustwidth}{25pt}{0pt}
15 – 22 (escura)
\end{adjustwidth}
OG: 1,048 – 1,065 (densidade padrão)\\
FG: 1,002 – 1,008 (densidade padrão)\\
ABV: 3,5 – 5,0\% (teor alcoólico leve)
\begin{adjustwidth}{25pt}{0pt}
5,0 – 7,0\% (teor alcoólico padrão)\\
7,0 – 9,5\% (teor alcoólico super)
\end{adjustwidth}

\leftskip0mm\textbf{Exemplos Comerciais}: Ellezelloise Saison 2000, Lefebvre Saison 1900, Saison Dupont, Saison de Pipaix, Saison Voisin, Boulevard Tank 7 Farmhouse Ale. \\
\textbf{Última Revisão}: Saison (2015)

\textbf{Atributos de Estilo}: bitter, pale-color, standard-strength, top-fermented, traditional-style, western-europe
