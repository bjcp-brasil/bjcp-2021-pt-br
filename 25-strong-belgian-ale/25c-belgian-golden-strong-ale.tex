\phantomsection
\subsection*{25C. Belgian Golden Strong Ale}
\addcontentsline{toc}{subsection}{25C. Belgian Golden Strong Ale}
\textbf{Impressão Geral}: Uma cerveja belga com teor alcoólico alto, muito clara, altamente atenuada, que é mais frutada e lupulada do que condimentada. Complexa e delicada, o final seco, o corpo baixo e a alta carbonatação acentuam o caráter do lúpulo e da levedura. Carbonatação espumante e efervescente, formando um colarinho firme branco.

\textbf{Aroma}: Um complexo buquê de ésteres frutados, lúpulos herbais e elevado álcool apimentado sobre uma base de malte quase neutra. Os ésteres são de moderados a altos, muitas vezes remetendo a maçã, pera e marmelo, especialmente pera. Lúpulos de baixos a moderados de perfil herbal, floral ou condimentado. Álcool e fenóis são de baixos a moderados e, muitas vezes, têm uma qualidade perfumada ou apimentada. A percepção do álcool deve ser suave, não pode ser quente ou remetendo a solvente. O malte deve ser quase neutro, possivelmente com leve adocicado de grãos.

\textbf{Aparência}: Cor de amarelo claro a dourado. Boa limpidez. Efervescente. Colarinho branco, maciço, de longa duração, firme, resultando no característico 'rendado belga' (\textit{Belgian lace}) na borda do copo na medida em que desaparece.

\textbf{Sabor}: Perfil semelhante ao do aroma (mesmos descritores e intensidades) para ésteres, lúpulos, fenóis e álcool. Os ésteres como pera, álcool apimentado, lúpulos herbais e sabores suaves de malte perduram na boca por um longo tempo até o retrogosto, tendo um final seco. O amargor é de médio a alto, acentuado pelo final seco e pela alta carbonatação, que permanece no retrogosto.

\textbf{Sensação na Boca}: Carbonatação muito alta. Efervescente. Corpo de leve a médio, porém mais leve do que a densidade pode sugerir. A carbonatação acentua a percepção de leveza. O aquecimento alcoólico é suave, mas perceptível, sem que seja quente ou que remeta a solvente.

\textbf{Comentários}: Os nomes para diversos exemplares comerciais do estilo incluem referências ao diabo, referindo-se ao seu alto teor alcoólico e como tributo ao exemplar original (Duvel). Tradicionalmente refermentada em garrafa.

\textbf{História}: Desenvolvida pela cervejaria Moortgat depois da I Guerra Mundial como uma resposta à crescente popularidade das cervejas Pilsen. Originariamente uma cerveja mais escura, assumiu a sua forma moderna na década de 1970

\textbf{Ingredientes}: Malte Pilsen com substanciais adjuntos açucarados. Lúpulos continentais. Levedura belga frutada. Água bastante mole. Não é tradicional adicionar especiarias.

\textbf{Comparação de Estilos}: Muitas vezes confundida com a Belgian Tripel, mas geralmente é mais clara, com corpo mais leve, com final mais bem definido e seca. Tende a se usar levedura que favorece o desenvolvimento de ésteres (particularmente de maçã, pera e marmelo) ao invés de condimentos no equilíbrio e tem mais caráter de lúpulo com adição tardia.

\begin{tabular}{@{}p{35mm}p{35mm}@{}}
  \textbf{Estatísticas}: & OG: 1,070 - 1,095  \\
  IBU: 22 - 35  & FG: 1,005 - 1,016   \\
  SRM: 3 - 6 & ABV: 7,5\% - 10,5\%
\end{tabular}

\textbf{Exemplos Comerciais}: Brigand, Delirium Tremens, Duvel, Judas, Lucifer, Russian River Damnation.

\textbf{Última revisão}: Belgian Golden Strong Ale (2015)

\textbf{Atributos de Estilo}: bitter, pale-color, top-fermented, traditional-style, very-high-strength, western-europe
