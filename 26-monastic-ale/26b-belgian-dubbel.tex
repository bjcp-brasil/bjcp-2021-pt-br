\phantomsection
\subsection*{26B. Belgian Dubbel}
\addcontentsline{toc}{subsection}{26B. Belgian Dubbel}
\textbf{Impressão Geral}: Uma \textit{ale} belga complexa de cor cobre avermelhada, teor alcoólico moderado, maltada, com sabores maltados ricos, ésteres que remetem a frutas secas ou escuras e uma nota alcoólica combinada a um perfil maltado com final bem seco.

\textbf{Aroma}: Aroma maltado rico que varia de moderado a moderadamente forte com notas de chocolate, açúcar caramelizado ou tostado. Nunca torrado ou queimado. Ésteres frutados moderados, muitas vezes remetendo a frutas secas ou escuras, especialmente passas e ameixas, por vezes frutas de caroço e banana. Fenólico de baixo a moderado, com caráter apimentado, condimentado. Tipicamente os lúpulos são ausentes, mas quando presentes, têm caráter condimentado, herbal ou floral de baixa intensidade. No equilíbrio, o malte é o mais forte, com ésteres e condimentados adicionando complexidade. Perfume alcoólico de baixa intensidade e suave é opcional.

\textbf{Aparência}: Cor de âmbar escuro a cobre, com uma atraente profundidade avermelhada. Geralmente límpida. Colarinho cremoso com longa duração, de cor quase branco, espesso e denso.

\textbf{Sabor}: Perfil similar ao aroma (são aplicados os mesmos descritores e intensidades) para malte, ésteres, fenólicos, álcool e lúpulos. Amargor de médio-baixo a médio, mas o malte sempre será mais proeminente no equilíbrio. Os ésteres e fenóis adicionam complexidade e interesse para o malte, ao passo que o álcool tipicamente não é percebido. Maltado rico, por vezes com sabor adocicado, com final moderadamente seco e sabor maltado acentuado no retrogosto pelos ésteres e fenóis provenientes da levedura.

\textbf{Sensação na Boca}: Macia, com corpo de médio a médio-cheio. Carbonatação média-alta, que pode influenciar na percepção do corpo. Aquecimento alcoólico baixo é opcional, porém nunca deve ser quente ou remeter a solvente.

\textbf{Comentários}: A maioria dos exemplos comerciais estão na faixa de 6,5 a 7,0\% de ABV. Por causa do amargor contido, o sabor pode ser um pouco adocicado, no entanto são cervejas bastante secas.

\textbf{História}: Ainda que cervejas fortes e escuras fossem produzidas muito antes, a Dubbel moderna remonta à \textit{double brown} ou à \textit{strong beer} produzida pela Westmalle em 1922, quando a cervejaria foi restabelecida após a I Guerra Mundial. Outros exemplares datam de período posterior à II Guerra Mundial.

\textbf{Ingredientes}: Levedura belga com caráter condimentado-esterificado. Impressão de uma variedade de grãos complexa, embora muitas versões sejam bem simples, com xarope de açúcar caramelizado ou açúcares não refinados, sendo a levedura responsável por grande parte da complexidade. Lúpulos continentais. Especiarias não são típicas, mas se presentes, devem ser sutis.

\textbf{Comparação de Estilos}: Talvez semelhante a uma Dunkles Bock, mas com caráter de levedura belga e açúcar. Similar em força e equilíbrio a uma Belgian Blond Ale, mas com um perfil mais rico de malte e éster. Menor teor alcoólico e menos intensa do que uma Belgian Dark Strong Ale.

\begin{tabular}{@{}p{35mm}p{35mm}@{}}
  \textbf{Estatísticas}: & OG: 1,062 - 1,075 \\
  IBU: 15 - 25  & FG: 1,008 - 1,018  \\
  SRM: 10 - 17  & ABV: 6\% - 7,6\%
\end{tabular}

\textbf{Exemplos Comerciais}: Chimay Red, Corsendonk Bruin, La Trappe Dubbel, Rochefort 6, St. Bernardus Pater 6, Westmalle Dubbel.

\textbf{Última Revisão}: Belgian Dubbel (2015)

\textbf{Atributos de Estilo}: amber-color, high-strength, malty, top-fermented, traditional-style, western-europe
