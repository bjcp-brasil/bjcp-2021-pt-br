\phantomsection
\subsection*{27A. Historical Beer: Kentucky Common}
\addcontentsline{toc}{subsection}{27A. Historical Beer: Kentucky Common}
\textbf{Impressão Geral}: Uma cerveja escura, limpa, seca, refrescante, ligeiramente maltada e com alta carbonatação. De sabor suave, com leve tostado e sabores de caramelo, servida bem fresca como uma cerveja leve, de salão.

\textbf{Aroma}: Maltado de baixo a médio como cereais, semelhante a milho ou dulçor de malte, com um leve tostado, cereal tipo biscoito, pão ou caramelo. Aroma de lúpulo de médio a moderadamente baixo, geralmente com caráter floral ou condimentado. Perfil limpo de fermentação, com possível éster leve que remete a fruto de baga. Baixos níveis de DMS são opcionais. Sem acidez. O equilíbrio tende ao malte.

\textbf{Aparência}: Cor de âmbar alaranjado a marrom. Normalmente límpida, mas pode ter uma leve turbidez. A formação de espuma pode não ser duradoura e geralmente é de cor branca a bege.

\textbf{Sabor}: Dulçor de malte como cereais moderado com notas de caramelo, \textit{toffee}, pão ou biscoito de baixo a médio-baixo. Geralmente com sabores leves no palato, típico de cerveja com adjuntos; baixo dulçor de cereais semelhante a milho é comum. Sabor de lúpulo floral ou condimentado de médio a baixo. Amargor de médio a baixo, sem retrogosto áspero ou agressivo. Pode apresentar leve frutado. O equilíbrio no final é em direção ao malte, possivelmente com um sabor levemente amargo ou mineral-sulfatado. O final é bastante seco. Sem acidez.

\textbf{Sensação na Boca}: Corpo de médio a médio-baixo, com sensação na boca relativamente macia. Altamente carbonatada. Pode ter uma textura cremosa.

\textbf{Comentários}: Relatos modernos do estilo geralmente mencionam acidez láctica ou uso da técnica \textit{sour mash}, mas os registros cervejeiros de meados de 1900 em cervejarias maiores não têm indicação de longos descansos ácidos, \textit{sour mash} ou envelhecimento prolongado. Essas histórias são provavelmente invenções modernas de cervejeiros caseiros, teorizando que, como os destiladores locais de Bourbon usavam \textit{sour mash}, os cervejeiros também deveriam usá-lo. Nenhum registro indica uso de \textit{sour mash} ou mesmo um perfil ácido na cerveja; pelo contrário, foi fabricada como uma cerveja barata e consumo rápido. Inscreva as versões ácidas em 28B Mixed-Fermentation Sour Beer.

\textbf{História}: Kentucky Common, um estilo originalmente americano, foi quase exclusivamente produzido e vendido ao redor de Louisville, Kentucky, de algum período depois da Guerra Civil até a Lei Seca. Era barato e produzido rapidamente, colocado em barris enquanto fermentava ativamente e bem fechado para permitir a carbonatação na adega do salão. Antes da morte do estilo, ele representava cerca de 75\% das vendas em torno de Louisville. Alguns especularam que era uma variante escura da Cream Ale, criada por cervejeiros imigrantes germânicos que adicionaram grãos mais escuros para ajudar a acidificar a água local rica em carbonatos.

\textbf{Ingredientes}: Malte de cevada de seis fileiras. \textit{Grits} (canjiquinha) de milho. Malte caramelo e \textit{black}. Lúpulo americano rústico de amargor. Lúpulos continentais importados para acabamento. Água com alto teor de carbonato. Levedura \textit{ale}.

\textbf{Comparação de Estilos}: Como uma Cream Ale de cor mais escura, enfatizando o milho, mas com um leve caráter de malte no sabor. Sabores e equilíbrio de malte são provavelmente mais próximos das \textit{lagers} modernas com adjuntos como a International Amber ou Dark Lagers, Irish Red Ales ou Belgian Pale Ales.

\begin{tabular}{@{}p{35mm}p{35mm}@{}}
  \textbf{Estatísticas}: & OG: 1,044 - 1,055  \\
  IBU: 15 - 30  & FG: 1,010 - 1,018  \\
  SRM: 11 - 20 & ABV: 4\% - 5,5\%
\end{tabular}

\textbf{Exemplos Comerciais}: Apocalypse Brew Works Ortel's 1912.

\textbf{Última Revisão}: Historical Beer: Kentucky Common (2015)

\textbf{Atributos de Estilo}: amber-color, balanced, historical-style, north-america, standard-strength, top-fermented
