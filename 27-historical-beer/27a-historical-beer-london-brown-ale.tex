\phantomsection
\subsection*{27A. Historical Beer: London Brown Ale}
\addcontentsline{toc}{subsection}{27A. Historical Beer: London Brown Ale}
\textbf{Impressão Geral}: Uma ale marrom de baixo teor alcoólico, deliciosa, doce e maltada, com complexidade de malte como caramelo e \textit{toffee} e um final de sabor doce.

\textbf{Aroma}: Aroma moderado de dulçor de malte, muitas vezes com um rico caráter de caramelo ou \textit{toffee}. Ésteres frutados de baixos a médios, geralmente frutas escuras, como ameixas. Muito baixo aroma de lúpulo terroso ou floral é opcional.

\textbf{Aparência}: Cor de marrom médio a muito escuro, mas pode ser quase preta. Quase opaca, embora deva ser relativamente límpida se visível. Espuma de quase branca a castanha, com formação de baixa a moderada.

\textbf{Sabor}: Profundo sabor doce de malte tipo caramelo ou \textit{toffee} no paladar, perdurando até o final, frequentemente com notas de biscoito e café. Alguns ésteres de frutas escuras podem estar presentes; perfil de fermentação relativamente limpo para uma cerveja inglesa. Baixo amargor. Baixo sabor de lúpulo terroso ou floral é opcional, mas é raro. Sabor moderadamente baixo de malte torrado ou malte \textit{black} mais pungente é opcional. Final moderadamente doce com retrogosto suave e maltado. Pode ter um sabor doce açucarado.

\textbf{Sensação na Boca}: Corpo médio, mas a doçura residual pode dar uma impressão mais encorpada. Carbonatação de média-baixa a média. Bastante cremosa e de textura suave, particularmente pela sua densidade.

\textbf{Comentários}: Cada vez mais rara, a Mann's tem mais de 90\% de participação de mercado na Grã-Bretanha, mas em um segmento cada vez menor. Sempre engarrafada. Frequentemente usada como misturas doces entre \textit{cask ales} leves e amargas nos \textit{pubs}. As versões comerciais podem ser pasteurizadas e adoçadas, o que dá um sabor mais açucarado.

\textbf{História}: Desenvolvida pela Mann's como um produto engarrafado em 1902. Elencada na época como “a cerveja mais doce de Londres”. As versões anteriores à Primeira Guerra Mundial tinham cerca de 5\% ABV, mas com o mesmo equilíbrio geral. Diminuiu em popularidade na segunda metade do século 20 e agora está quase extinta.

\textbf{Ingredientes}: Malte \textit{pale ale} inglês como base, com uma grande proporção de malte caramelo mais escuros e frequentemente algum malte \textit{black} e de trigo (este é o grão tradicional da Mann – outras podem contar com açúcares escuros para cor e sabor). Água com teor de carbonato de moderado a alto. Lúpulo inglês. Adoçada pós-fermentação com lactose ou adoçantes artificiais, ou sacarose, se pasteurizada.

\textbf{Comparação de Estilos}: Pode parecer uma versão menos torrada de uma Sweet Stout (e de menor densidade, pelo menos para os exemplos de Sweet Stout dos EUA) ou uma versão doce de uma Dark Mild.

\begin{tabular}{@{}p{35mm}p{35mm}@{}}
  \textbf{Estatísticas}: & OG: 1,033 - 1,038  \\
  IBU: 15 - 20  & FG: 1,012 - 1,015  \\
  SRM: 22 - 35 & ABV: 2,8\% - 3,6\%
\end{tabular}

\textbf{Exemplos Comerciais}: Harveys Bloomsbury Brown Ale, Mann's Brown Ale.

\textbf{Última Revisão}: Historical Beer: London Brown Ale (2015)

\textbf{Atributos de Estilo}: british-isles, brown-ale-family, dark-color, historical-style, malty, session-strength, sweet, top-fermented
