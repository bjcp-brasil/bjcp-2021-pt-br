\phantomsection
\subsection*{28B. Mixed-Fermentation Sour Beer}
\addcontentsline{toc}{subsection}{28B. Mixed-Fermentation Sour Beer}
\textit{Destinada a cervejas fermentadas com qualquer combinação de Sacch, Lacto, Pedio e Brett (ou adicionais bactérias), com ou sem envelhecimento em carvalho (exceto se a cerveja se enquadrar em 28A ou 28D).}

\textbf{Impressão Geral}: Uma versão ácida e \textit{funky} de um estilo base de cerveja.

\textbf{Aroma}: Varia com o estilo base. A contribuição de micróbios que são \textit{Saccharomyces} deve ser perceptível a forte e frequentemente contribui com uma nota selvagem, ácida e \textit{funky}. Os melhores exemplares apresentam uma amplitude de aromáticos ao invés de uma característica única e dominante. O aroma deve ser convidativo, não agressivo e desagradável.

\textbf{Aparência}: Varia com o estilo base. Limpidez pode variar; turbidez baixa não é necessariamente uma falha. Retenção de espuma pode ser ruim.

\textbf{Sabor}: Varia com o estilo base. Procure um equilíbrio agradável entre a cerveja base e o caráter da fermentação. Uma variedade de resultados é possível, desde acidez e \textit{funky} bastante altos até uma cerveja sutil, agradável e harmoniosa. Os melhores exemplos são prazerosos de beber com os ésteres e fenóis complementando o malte ou lúpulo. O caráter selvagem pode ser proeminente, mas não precisa ser dominante em um estilo com um perfil intenso de malte ou lúpulo. A acidez deve ser firme, mas agradável, variando de limpa a complexa, mas não deve ser agressiva ou com caráter de vinagre; ácido acético proeminente ou ofensivo é uma falha. O amargor tende a ser baixo, especialmente conforme a acidez aumenta.

\textbf{Sensação na Boca}: Varia com o estilo base. Geralmente tem um corpo baixo, quase sempre mais leve do que o esperado para o estilo base. Carbonatação usualmente moderada a alta, embora muitas vezes menor em exemplos de alto teor alcoólico.

\textbf{Comentários}: O estilo base é menos relevante neste estilo porque as leveduras e bactérias tendem a dominar o perfil. Amargor normalmente é contido, uma vez que acidez e amargor são sensações conflitantes no palato. Características inapropriadas incluem diacetil, solvente, textura viscosa e oxidação intensa.

\textbf{História}: Interpretações modernas artesanais ou experimentações inspiradas em Belgian sour ales .

\textbf{Ingredientes}: Virtualmente qualquer estilo de cerveja. Frequentemente fermentada por uma combinação de \textit{Lactobacillus}, \textit{Pediococcus}, \textit{Saccharomyces} e \textit{Brettanomyces}. Pode também ser uma mistura de estilos. Madeiras ou envelhecimento em barricas é bastante comum, mas não obrigatório; caso presente não deve ser um sabor primário ou dominante.

\textbf{Comparação de Estilos}: Uma versão ácida e \textit{funky} de um estilo base de cerveja, mas não necessariamente é tão ácida ou \textit{funky} quanto alguns exemplos europeus de cervejas ácidas.

\textbf{Instruções para Inscrição}: O participante do concurso deve especificar uma descrição da cerveja, identificando leveduras ou bactérias e um estilo base ou ingredientes e características alvo para a cerveja.

\textbf{Estatísticas}: Varia com o estilo base.

\textbf{Exemplos Comerciais}: Boulevard Love Child, Jester King Le Petit Prince, Jolly Pumpkin Oro de Calabaza, Lost Abbey Ghosts in the Forest, New Belgium Le Terroir, Russian River Temptation.

\textbf{Atributos de Estilo}: craft-style, north-america, sour, specialty-beer, wild-fermentation
