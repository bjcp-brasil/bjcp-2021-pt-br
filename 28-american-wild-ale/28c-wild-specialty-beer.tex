\phantomsection
\subsection*{28C. Wild Specialty Beer}
\addcontentsline{toc}{subsection}{28C. Wild Specialty Beer}
\textit{Destinada para variações de cervejas do estilo 28A, 28B ou 28D. Essas variações podem incluir a adição de um ou mais ingredientes especiais; envelhecimento em variedades de madeira não tradicionais que confiram um caráter significativo e identificável de madeira (por exemplo, Cedro, Amburana); ou envelhecimento em barris que contém outra bebida alcoólica (por exemplo, destilados, vinho, sidra).}

\textbf{Impressão Geral}: Uma American Wild Ale com frutas, ervas, especiarias ou outros ingredientes especiais.

\textbf{Aroma}: Varia com o estilo base. O ingrediente especial deve ser evidente assim como as características de uma fermentação selvagem. Os melhores exemplares apresentam uma mistura de aromáticos de fermentação e de ingredientes especiais, criando um aroma que pode ser difícil de atribuir precisamente.

\textbf{Aparência}: Varia com o estilo base, geralmente apresentando cor, tonalidade ou nuance de um ingrediente especial (principalmente no caso de uso de frutas) tanto na cerveja quanto na espuma. Limpidez variável; leve turbidez não é uma falha. Retenção de espuma normalmente é baixa.

\textbf{Sabor}: Varia com o estilo base. O ingrediente especial deve ser evidente assim como as características de uma fermentação selvagem. Se frutas foram fermentadas, o dulçor geralmente não aparece, restando apenas os ésteres frutados. Frutas e outros ingredientes especiais podem adicionar acidez; caso positivo a acidez pode ser proeminente, mas não deve ser intensa demais. Tanto a acidez quanto taninos provenientes de frutas e outros ingredientes especiais podem aumentar a percepção de secura portanto cuidados devem ser observados com o equilíbrio da cerveja. A acidez deve realçar a percepção de sabores de frutas, mas não prejudicar. Notas de madeira, caso presentes, agregam sabor, mas devem ser equilibradas.

\textbf{Sensação na Boca}: Varia com o estilo base. Geralmente tem um corpo baixo, mais leve do que o esperado para o estilo base. Carbonatação usualmente moderada a alta; a carbonatação deve equilibrar o estilo base caso seja declarado. A presença de taninos de ingredientes especiais (frequentemente frutas ou madeiras) pode contribuir com leve adstringência, aumentar o corpo ou fazer a cerveja ter uma percepção mais seca do que realmente é.

\textbf{Comentários}: Este estilo destina-se a versões com frutas (e outros ingredientes especiais) de outros estilos da categoria 28, não variações de estilos clássicos europeus ácidos ou selvagens. Versões de Lambics com adição de frutas devem ser inscritas em 23F Fruit Lambic. Versões de outros estilos clássicos ácidos com frutas (por exemplo, Flanders Red, Oud Bruin, Gose, Berliner Weisse) devem ser inseridas em 29A Fruit Beer. Cervejas com açúcares e frutas não fermentadas adicionadas após a fermentação devem ser inseridas em 29C Specialty Fruit Beer.

\textbf{História}: Interpretações modernas artesanais ou experimentações inspiradas em Belgian wild ales.

\textbf{Ingredientes}: Virtualmente qualquer estilo de cerveja. Qualquer combinação de Sacch, Brett, Lacto, Pedio ou outro microrganismo similar. Pode ser ainda uma mistura de estilos. Enquanto cerejas, framboesas e pêssegos são mais comuns, outras frutas também podem ser utilizadas. Vegetais com características de frutas (como pimentas, ruibarbo, abóbora) também podem ser utilizados. Envelhecimento em madeira ou em barricas é bastante comum, mas não obrigatório. É permitido o uso de madeiras com características únicas ou não usuais ou madeiras com contato prévio com outros tipos de álcool.

\textbf{Comparação de Estilos}: Como uma cerveja com frutas, ervas, condimentos ou madeira, mas ácida ou \textit{funky}.

\textbf{Instruções para Inscrição}: O participante do concurso deve especificar quaisquer ingredientes especiais utilizados (por exemplo, frutas, especiarias, ervas ou madeiras). O participante deve fornecer uma descrição da amostra, identificando leveduras ou bactérias utilizadas e estilo base ou ingredientes, ou caráter alvo da cerveja. Uma descrição geral da característica especial da cerveja pode cobrir todos os itens exigidos.

\textbf{Estatísticas}: Varia com o estilo base.

\textbf{Exemplos Comerciais}: Cascade Bourbonic Plague, Jester King Atrial Rubicite, New Belgium Dominga Mimosa Sour, New Glarus Wisconsin Belgian Red, Russian River Supplication, The Lost Abbey Cuvee de Tomme.

\textbf{Atributos de Estilo}: craft-style, fruit, north-america, sour, specialty-beer, wild-fermentation
