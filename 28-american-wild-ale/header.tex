\section*{28. American Wild Ale}
\addcontentsline{toc}{section}{28. American Wild Ale}

O nome American Wild Ale é comumente usado por cervejeiros artesanais e homebrewers. No entanto, a palavra \textit{Wild} não implica que essas cervejas sejam necessariamente fermentadas espontaneamente; em vez disso, indica que elas são influenciadas por outra microbiota além das leveduras cervejeiras tradicionais ou talvez sejam cervejas de fermentação mista. O uso da palavra \textit{American} não significa que a cerveja deva ser baseada em um estilo clássico de cerveja americana ou que os métodos sejam praticados exclusivamente nos Estados Unidos. Estilos base nesta categoria não precisam ser estilos clássicos (embora possam ser); algo como “Blond Ale, 7\%” seria bom, já que o estilo base é frequentemente perdido devido ao caráter de fermentação. Esta categoria destina-se a uma ampla gama de cervejas que não se enquadram nos estilos tradicionais europeus ácidos, selvagens ou de fermentação espontânea. Todos os estilos nesta categoria são cervejas especiais onde muitas interpretações criativas são possíveis e os estilos são definidos apenas pelo uso de ingredientes e perfis de fermentação específicos. Como estilos especiais, a descrição obrigatória fornecida pelo participante é de extrema importância para o juiz. Os estilos nesta categoria são diferenciados pelos tipos de leveduras e bactérias que são utilizados – veja o preâmbulo de cada estilo para mais informações. Usamos os termos abreviados de conversação utilizados na indústria cervejeira: \textit{Brett} para Brettanomyces, \textit{Sacch} para Saccharomyces, \textit{Lacto} para Lactobacillus e \textit{Pedio} para Pediococcus. Consulte o glossário para obter informações adicionais. O estilo Wild Specialty Beer é para cervejas de outros estilos dentro desta categoria quando ingredientes especiais são adicionados. Notas de segundo plano de carvalho podem ser usados em todos os estilos dentro desta categoria, mas cervejas envelhecidas em outras madeiras com sabores únicos ou barris que continham outros produtos alcoólicos devem ser inseridas no estilo Wild Specialty Beer.
