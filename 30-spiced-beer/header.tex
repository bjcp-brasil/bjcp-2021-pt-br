\section*{30. Spiced Beer}
\addcontentsline{toc}{section}{30. Spiced Beer}
\textit{Foram utilizadas as definições comuns ou culinárias de especiarias, ervas ou vegetais, e não as definições botânicas ou científicas. De maneira geral, especiarias são as sementes secas, vagens de sementes, frutas, raízes, cascas, etc. de plantas utilizadas para dar sabor à comida. Ervas são plantas ou partes de plantas (folhas, flores, pétalas, talos) utilizadas para dar sabor à comida. Vegetais são plantas comestíveis salgadas ou menos doces, usados principalmente para cozinhar e, às vezes, para serem comidos crus. Vegetais podem incluir algumas frutas botânicas. Esta categoria explicitamente inclui todos as especiarias, ervas e vegetais culinários, como amêndoas (ou qualquer coisa que tenha nut no nome em inglês, incluindo coco), pimenta, café, chocolate, pontas de abeto, rosa mosqueta, hibisco, cascas e raspas de frutas (porém, não o suco), ruibarbo, e similares. Não estão incluídos aqui frutas ou grãos. Açúcares e xaropes fermentáveis com sabor (por exemplo, néctar de agave, xarope de bordo, melaço, sorgo, mel) ou adoçantes (por exemplo, lactose) podem ser incluídos somente em combinação com outros ingredientes permitidos e não devem ter um caráter dominante.}

\textit{Veja a categoria 29 para a definição e exemplos de frutas. Veja a introdução da seção Cerveja de Especialidade para comentários adicionais, particularmente relacionados à avaliação do equilíbrio dos ingredientes adicionados com a cerveja base.}
