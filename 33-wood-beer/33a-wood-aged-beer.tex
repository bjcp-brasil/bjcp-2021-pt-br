\phantomsection
\subsection*{33A. Wood-Aged Beer}
\addcontentsline{toc}{subsection}{33A. Wood-Aged Beer}
\textit{Este estilo é destinado às cervejas envelhecidas em madeira sem a adição do caráter alcoólico proveniente do uso prévio do barril. Cervejas envelhecidas em barris de Bourbon ou com características de outras bebidas alcoólicas devem ser categorizadas no estilo 33B Spectialty Wood-Aged Beer.}

\textbf{Impressão Geral}: Um aprimoramento agradável do estilo base da cerveja, com as características do envelhecimento em contato com a madeira. Os melhores exemplos serão suaves, saborosos, bem equilibrados e bem envelhecidos.

\textbf{Aroma}: Varia de acordo com o estilo base. Um aroma amadeirado, de baixo a moderado, geralmente está presente; algumas variedades podem ter um caráter mais forte ou distinto. A madeira fresca pode, ocasionalmente, transmitir cheiro de madeira crua e recém-cortada, embora esse caráter nunca deva ser muito forte. Se a madeira for tostada ou carbonizada, pode haver de baixo a moderado caráter de baunilha, caramelo, toffee, tosta ou cacau.

\textbf{Aparência}: Varia de acordo com o estilo base. Frequentemente uma cerveja mais escura do que o estilo base inalterado, particularmente se forem usados barris tostados ou carbonizados.

\textbf{Sabor}: Varia de acordo com o estilo base. A madeira geralmente contribui com um sabor amadeirado e, possivelmente, com um perfil característico da variedade utilizada. A madeira nova pode adicionar uma impressão de madeira crua e recém cortada. A madeira tostada ou carbonizada pode adicionar notas de baunilha, caramelo, butterscotch, pão tostado, nozes tostadas, café, chocolate ou cacau, dependendo da variedade de madeira e do nível de queima usada. Os sabores derivados da madeira devem ser equilibrados, sustentados e perceptíveis, sem serem excessivos em relação ao estilo base da cerveja.

\textbf{Sensação na Boca}: Varia de acordo com o estilo base. Os taninos da madeira podem aumentar a percepção de corpo, além de potencializar a secura no final; alguma adstringência dos taninos da madeira é permitida. As características azedas ou ácidas devem ser baixas ou inexistentes.

\textbf{Comentários}: Grande parte do caráter depende do tipo de madeira usada e de como ela complementa e aprimora o estilo base. O caráter de envelhecimento é permitido, mas oxidação ou acidez excessivas são falhas. Álcool perceptível não é uma falha em estilos base mais fortes. Esta categoria não deve ser usada para estilos base onde o envelhecimento em madeira é um requisito fundamental (por exemplo, Flanders Red, Lambic). Cervejas feitas com envelhecimento limitado em madeira ou produtos que fornecem apenas um caráter sutil podem ser inseridas nas categorias de estilo base de cerveja, desde que o caráter de madeira não seja destacado.

\textbf{História}: Um método de produção tradicional, que raramente é usado pelas principais cervejarias e, geralmente, apenas para produtos especiais. Mais popular entre as cervejarias artesanais modernas, que procuram produtos novos e diferenciados. Os barris e tonéis de carvalho são tradicionais, embora outras madeiras estejam se tornando mais populares.

\textbf{Ingredientes}: Varia de acordo com o estilo base. Envelhecido em barris ou tonéis de madeira ou, ainda, usando aditivos à base de madeira (por exemplo, lascas, ripas, espirais, cubos). Estilos base mais encorpados e de maior densidade geralmente são usados, pois podem resistir melhor aos sabores adicionais, embora experimentações sejam incentivadas.

\textbf{Instruções para Inscrição}: O participante \textbf{deve} especificar o \underline{tipo de madeira} utilizada e o \underline{nível de tosta ou carbonização} (se utilizada). Caso seja usada uma variedade incomum de madeira, o participante \textbf{deve} fornecer uma breve descrição dos aspectos sensoriais que ela agrega à cerveja. O participante \textbf{deve} fornecer a descrição da cerveja, identificando \textbf{ou} um Estilo Base \textbf{ou} os ingredientes, especificações e/ou o caráter desejado para a cerveja. Uma descrição geral da natureza especial da cerveja pode abranger todos os itens necessários.

\textbf{Estatísticas}:

IBU: varia com o estilo base\\
SRM: varia com o estilo base, geralmente mais escura do que a base inadulterada\\
OG: varia com o estilo base, geralmente acima da média\\
FG: varia com o estilo base\\
ABV: varia com o estilo base, geralmente acima da média
\textbf{Exemplos Comerciais}: Bush Prestige, Cigar City Spanish Cedar Jai Alai, Firestone Walker Double Barrel Ale, Midnight Sun Arctic Devil, Petrus Aged Pale, Samuel Smith Yorkshire Stingo.

\textbf{Última Revisão}: Wood-Aged Beer (2015)

\textbf{Atributos de Estilo}: specialty-beer, wood
