\phantomsection
\subsection*{33B. Specialty Wood-Aged Beer}
\addcontentsline{toc}{subsection}{33B. Specialty Wood-Aged Beer}
\textit{Este estilo é destinado às cervejas envelhecidas em madeira com a adição do perfil e caráter alcoólico proveniente do uso prévio do barril. Cervejas envelhecidas em barris de Bourbon ou com características de outras bebidas alcoólicas devem ser categorizadas neste estilo.}

\textbf{Impressão Geral}: Um aprimoramento do estilo base da cerveja, com as características provenientes do envelhecimento em contato com a madeira, incluindo as que estiveram previamente em contato com bebidas alcoólicas. Os melhores exemplos serão suaves, saborosos, bem equilibrados e bem envelhecidos.

\textbf{Aroma}: Varia de acordo com o estilo base. Um aroma amadeirado, de baixo a moderado, geralmente está presente; algumas variedades podem ter um caráter mais forte ou distinto. Se a madeira for tostada ou carbonizada, pode haver de baixo a moderado caráter de baunilha, caramelo, toffee, tosta ou cacau. Os aromas associados ao álcool previamente armazenado na madeira (por exemplo, de destilados, vinho) devem ser perceptíveis, mas equilibrados.

\textbf{Aparência}: Varia de acordo com o estilo base. Frequentemente uma cerveja mais escura do que o estilo base inalterado, particularmente se forem usados barris carbonizados. Cervejas envelhecidas em barris de vinho ou outras bebidas com cores distintas também podem conferir cor à cerveja finalizada.

\textbf{Sabor}: Varia de acordo com o estilo base. A madeira geralmente contribui com um sabor amadeirado e, possivelmente, com um perfil característico da variedade utilizada. A madeira tostada ou carbonizada pode adicionar notas de baunilha, caramelo, butterscotch, pão tostado, nozes tostadas, café, chocolate ou cacau, dependendo da variedade de madeira e do nível de queima usada. Os sabores derivados da madeira e do álcool previamente armazenado na madeira devem ser equilibrados, mutuamente sustentados e perceptíveis, sem serem excessivos em relação ao estilo base da cerveja ou um ao outro.

\textbf{Sensação na Boca}: Varia de acordo com o estilo base. Os taninos da madeira podem aumentar a percepção de corpo, além de potencializar a secura no final; alguma adstringência dos taninos da madeira é permitida. Geralmente apresenta aquecimento alcoólico adicional, mas que não deve ser quente ou áspero. As características azedas ou ácidas devem ser baixas ou inexistentes.

\textbf{Comentários}: O sucesso neste estilo depende do quanto os caráteres amadeirado e alcoólico complementam e aprimoram a cerveja base, bem como se estão bem integrados com o perfil geral de sabor. O caráter de envelhecimento é permitido, mas oxidação ou acidez excessivas são falhas. Cervejas ácidas envelhecidas em madeira devem ser inscritas em 28C Wild Specialty Beer.

\textbf{História}: Igual a 33A Wood-Aged Beer.

\textbf{Ingredientes}: Varia de acordo com o estilo base. Envelhecidas em tonéis ou barris de madeira anteriormente utilizados para armazenar bebidas alcoólicas (por exemplo, uísque, bourbon, rum, gin, tequila, porto, xerez, vinho Madeira, vinho). Estilos base mais encorpados e de maior densidade geralmente são usados, pois podem resistir melhor aos sabores adicionais, embora experimentações sejam incentivadas.

\textbf{Instruções para Inscrição}: O participante \textbf{deve} especificar o caráter alcoólico adicional, com informações sobre o barril, caso sejam relevantes para o perfil final do sabor. Se uma variedade incomum de madeira foi usada, o participante \textbf{deve} fornecer uma breve descrição dos aspectos sensoriais que ela agrega à cerveja. O participante \textbf{deve} fornecer a descrição da cerveja, identificando \textbf{ou} um Estilo Base \textbf{ou} os ingredientes, especificações e/ou o caráter desejado para a cerveja. Uma descrição geral da natureza especial da cerveja pode abranger todos os itens necessários.

\textbf{Estatísticas}:

IBU: varia com o estilo base\\
SRM: varia com o estilo base, geralmente mais escura do que a base inadulterada\\
OG: varia com o estilo base, geralmente acima da média\\
FG: varia com o estilo base\\
ABV: varia com o estilo base, geralmente acima da média

\textbf{Exemplos Comerciais}: AleSmith Barrel-Aged Old Numbskull, Founders Kentucky Breakfast Stout, Firestone Walker Parabola, Goose Island Bourbon County Stout, Great Divide Barrel Aged Yeti, The Lost Abbey Angel’s Share Ale.

\textbf{Última Revisão}: Specialty Wood-Aged Beer (2015)

\textbf{Atributos de Estilo}: specialty-beer, wood
