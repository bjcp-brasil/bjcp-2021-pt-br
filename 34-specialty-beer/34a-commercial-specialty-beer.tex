\phantomsection
\subsection*{34A. Commercial Specialty Beer}
\addcontentsline{toc}{subsection}{34A. Commercial Specialty Beer}
\textit{Este estilo se destina a reproduções ou interpretações de cervejas comerciais específicas e que não se enquadram em outros estilos definidos. As cervejas nele inscritas não precisam ser cópias exatas. A cerveja deve ser julgada de acordo com o quão bem ela se encaixa no estilo mais amplo, representado pela cerveja exemplo, e não com o quão uma cópia exata de um produto comercial específico ela é. Se uma Commercial Specialty Beer se encaixa em outro estilo definido, não a inscreva aqui.}

\textbf{Impressão Geral}: De acordo com a cerveja declarada.

\textbf{Aroma}: De acordo com a cerveja declarada.

\textbf{Aparência}: De acordo com a cerveja declarada.

\textbf{Sabor}: De acordo com a cerveja declarada.

\textbf{Sensação na Boca}: De acordo com a cerveja declarada.

\textbf{Comentários}: Destinada a ser uma subcategoria que engloba cervejas específicas, baseadas em exemplares comerciais únicos, e que não se encaixam nos demais estilos existentes. Versões anteriores do Guia de Estilos incluíam o chamado Belgian Specialty Ale; este estilo se enquadra no mesmo propósito, permitindo cervejas não-belgas e com objetivo similar.

\textbf{Instruções para Inscrição}: O participante deve especificar o nome da cerveja comercial, as especificações (estatísticas) para a cerveja e uma breve descrição sensorial e/ou a lista de ingredientes usados. Sem essas informações, os juízes que não estão familiarizados com a cerveja comercial podem não ter base para comparação.

\textbf{Estatísticas}: OG, FG, IBU, SRM e ABV vão variar de acordo com a cerveja base declarada.

\textbf{Exemplos Comerciais}: Orval, La Chouffe.

\textbf{Última Revisão}: Clone Beer (2015)

\textbf{Atributos de Estilo}: specialty-beer

