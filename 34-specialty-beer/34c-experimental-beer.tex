\phantomsection
\subsection*{34C. Experimental Beer}
\addcontentsline{toc}{subsection}{34C. Experimental Beer}
\textit{Esta é, explicitamente, uma subcategoria abrangente, para qualquer cerveja que não se encaixe na descrição de nenhum \textbf{Estilo Existente}. Aqui, nenhuma cerveja está “fora de estilo”, a menos que possa ser incluída em outro. Este é o último recurso para qualquer cerveja inscrita em uma competição. Com a ampla definição anterior de estilos, este raramente deve ser usado.}

\textbf{Impressão Geral}: Varia, mas deve ser uma experiência única.

\textbf{Aroma}: Varia.

\textbf{Aparência}: Varia.

\textbf{Sabor}: Varia.

\textbf{Sensação na Boca}: Varia.

\textbf{Comentários}: O estilo não pode representar uma cerveja comercial reconhecida (caso contrário, ela deveria ser uma Commercial Specialty Beer) e não pode se encaixar em nenhum outro estilo de Cerveja de Especialidade (incluindo aqueles dentro dessa categoria).

\textbf{Instruções para Inscrição}: O participante \textbf{deve} especificar a natureza especial da cerveja experimental, incluindo os ingredientes e/ou os processos especiais que fazem com que ela não se encaixe em nenhum outro lugar deste guia. O participante \textbf{deve} fornecer as estatísticas da cerveja \textbf{e} ou uma breve descrição sensorial ou a lista de ingredientes usados. \textit{Sem essas informações, os juízes não terão base para a avaliação}.

\textbf{Estatísticas}: OG, FG, IBU, SRM e ABV vão variar de acordo com a cerveja declarada.

\textbf{Exemplos Comerciais}: Nenhum

\textbf{Última Revisão}: Experimental Beer (2015)

\textbf{Atributos de Estilo}: specialty-beer
