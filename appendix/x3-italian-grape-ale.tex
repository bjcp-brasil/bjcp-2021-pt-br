\phantomsection
\subsection*{X3. Italian Grape Ale}
\addcontentsline{toc}{subsection}{X3. Italian Grape Ale}

\textit{Para uso fora da Itália, consulte 29D Grape Ale}
\textbf{Impressão Geral}: Uma cerveja italiana às vezes refrescante, às vezes mais complexa, caracterizada por diferentes variedades de uvas.

\textbf{Aroma}: As características aromáticas de uma determinada uva devem ser perceptíveis, mas não devem se sobrepor aos outros aromas. O caráter da uva deve ser agradável e não deve apresentar defeitos, como oxidação. O caráter de malte é geralmente contido e não deve exibir um perfil torrado. O aroma de lúpulo (floral, terroso) pode variar de médio-baixo a ausente. Alguns exemplares podem ter um leve perfil selvagem, descrito como curral, terroso, caprino, mas não deve ser tão intenso como em uma Lambic ou Fruit Lambic. Sem diacetil.

\textbf{Aparência}: A cor pode variar de dourado claro a cobre, mas alguns exemplares podem ser marrons. A cor avermelhada ou rubi se deve, geralmente, à utilização de variedades de uva tinta. Espuma de branca a avermelhada, geralmente com retenção média-baixa. Geralmente com boa limpidez, mas pode apresentar alguma turbidez.

\textbf{Sabor}: Assim como no aroma, o caráter de uva (similar a mosto ou vinho) deve estar presente e sua intensidade pode variar de média-baixa a média-alta. Variedades de uva podem contribuir de diferentes formas no perfil de sabor: em geral, sabores de frutas tropicais ou frutas de caroço (pêssego, pera, damasco e abacaxi) podem vir de uvas brancas, e sabores de frutas vermelhas (p.ex., cereja e morango) de variedades de uvas tintas. Além disso, o caráter frutado oriundo da fermentação também é comum. Diferentes tipos de maltes especiais podem ser usados, mas devem ser equilibrados e para dar suporte, não tão proeminentes a ponto de ofuscar a cerveja base. Um forte caráter torrado e/ou de chocolate é inapropriado. Leves notas ácidas, devido ao uso de uva, são comuns e podem ajudar a melhorar a facilidade em se beber, mas não devem ser proeminentes como em uma Sour Ale, Lambic ou similares. Sabores de carvalho, junto com algumas notas de curral, terrosas ou caprinas podem estar presentes, mas não devem ser predominantes. Amargor e sabores de lúpulo são baixos. Diacetil ausente.

\textbf{Sensação na Boca}: A carbonatação média-alta melhora a percepção do aroma. O corpo é geralmente de baixo a médio e alguma acidez pode contribuir para aumentar a percepção de secura. Exemplares fortes podem apresentar algum aquecimento, mas sem serem quentes ou como solventes.

\textbf{História}: Inicialmente produzinda em Birrificio Montegioco e Birrificio Barley, em 2006-2007, a Italian Grape Ale (IGA) agora é produzida por muitas cervejarias artesanais italianas. Também está se tornando popular nos EUA e em outros países vinicultores. Representa uma união entre a cerveja e o vinho, promovida pela grande disponibilidade local de diferentes variedades de uvas em todo o país. Podem ser uma expressão do território, da biodiversidade e da criatividade do cervejeiro. Normalmente vista como uma cerveja especial dentre os produtos da cervejaria. As cervejarias chamam de “Wild IGA” ou “Sour IGA” qualquer versão selvagem ou ácida do estilo.

\textbf{Ingredientes}: Malte Pils, na maioria dos casos, ou malte base Pale com alguns maltes especiais (se houver). O teor de uva pode representar até 40\% do perfil de maltes. A uva ou o mosto de uvas, às vezes fervidos por longo tempo antes da utilização, podem ser utilizados em diferentes fases: durante a fervura ou, mais frequentemente, durante a fermentação primária ou secundária. A levedura pode apresentar um caráter neutro (mais comum) ou um perfil frutado ou condimentado (cepas inglesas e belgas). Levedura de vinho também pode ser usada em conjunto com outras leveduras. As variedades de lúpulos do velho mundo, principalmente alemãs ou inglesas, são utilizadas em baixas quantidades para não caracterizar excessivamente a cerveja.

\textbf{Comparação de Estilos}: Semelhante à Fruit Beer, mas evoluiu como um estilo autônomo devido à abundância de variedades de uvas na Itália.

\begin{tabular}{@{}p{35mm}p{35mm}@{}}
  \textbf{Estatísticas}: & OG: 1,055 – 1,100 \\
  IBU: 6 - 30 & 1,008 – 1,015 \\
  SRM: 4 - 25 & ABV: 4,5 – 12\%
\end{tabular}

\textbf{Exemplos Comerciais}: Montegioco Open Mind, Birrificio Barley BB5-10, Birrificio del Forte Il Tralcio, Viess Beer al mosto di gewurtztraminer, CRAK IGA Cabernet, Birrificio Apuano Ninkasi, Luppolajo Mons Rubus

\textbf{Atributos de Estilo}: Specialty-beer, Fruit
