\phantomsection
\subsection*{X4. Catharina Sour}
\addcontentsline{toc}{subsection}{X4. Catharina Sour}

\textit{Estilo sugerido para inscrição: Categoria 29 (Fruit Beer)}

\textbf{Impressão Geral}: Uma cerveja refrescante de trigo, ácida e com frutas, possui um caráter vívido de frutas e uma acidez lática limpa. A graduação alcoólica contida, o corpo leve, a carbonatação elevada e amargor abaixo da percepção fazem com que a fruta fresca seja o destaque. A fruta não precisa ser de caráter tropical, mas, normalmente, apresenta este perfil.

\textbf{Aroma}: Caráter de fruta de médio a alto, reconhecível e identificável de forma imediata. Acidez lática limpa, de intensidade baixa à média, que complementa a fruta. Malte tipicamente neutro, mas pode apresentar notas de pão e cereais em caráter de apoio. Fermentação limpa, sem caráter de levedura selvagem ou \textit{funky}. Sem aroma de lúpulo. Sem álcool agressivo. Especiarias, ervas e vegetais devem complementar a fruta, se estiverem presentes.

\textbf{Aparência}: Coloração tipicamente bastante clara, de amarela-palha a dourada. Espuma branca de média à alta formação e média à boa retenção. A coloração da espuma e da cerveja podem ser alteradas e ficar com a coloração da fruta. Pode ser de bastante límpida à turva. Efervescente.

\textbf{Sabor}: Sabor de fruta fresca dominante, de médio a alto, e com acidez lática limpa, de baixa à média-alta, de forma complementar, mas notável. A fruta deve ter um caráter fresco, sem parecer cozida, parecida com geléia ou artificial. O malte é normalmente ausente; se presente, pode ter um caráter baixo de cereais ou pão, mas não deve nunca competir com a fruta ou a acidez. Amargor do lúpulo abaixo do limiar de percepção. Final seco, com um retrogosto limpo, ácido e frutado. Sem sabor de lúpulo, notas acéticas, diacetil, ou sabores \textit{funky}, oriundos de Brettanomyces. Especiarias, ervas e vegetais são opcionais e em caráter complementar à fruta.

\textbf{Sensação na Boca}: Corpo de baixo a médio-baixo. Carbonatação de média a alta. Sem aquecimento alcoólico. Acidez de baixa a média-baixa, sem ser agressivamente ácida ou adstringente.

\textbf{Comentários}: Melhor servida fresca. A acidez pode fazer com que a cerveja pareça mais seca e com um corpo menor do que a densidade final sugere. Uma Berliner Weisse com adição de frutas deve ser inscrita na categoria 29A Fruit Beer.

\textbf{História}: Os exemplos individuais existiam com nomes diferentes, anteriormente, no Brasil, mas o estilo se tornou popular com esse nome depois que foi definido formalmente, em 2015, durante uma reunião entre cervejeiros profissionais e caseiros em Santa Catarina. Utilizando ingredientes locais, adequado para um clima quente, o estilo se espalhou para outros estados do Brasil e além, sendo um estilo muito popular na América do Sul – tanto em competições comerciais como caseiras.

\textbf{Ingredientes}: Malte Pilsen com malte de trigo ou trigo não maltado. A técnica de Kettle Sour com o uso de Lactobacillus é a mais comum de ser utilizada, seguida por uma fermentação com uma levedura ale neutra. A fruta é tipicamente adicionada nos estágios finais da fermentação. Frutas da estação, frescas, comumente tropicais. Especiarias, ervas e vegetais são opcionais, mas devem sempre estar em caráter de apoio e elevar a percepção da fruta.

\textbf{Comparação de Estilo}: Como uma Berliner Weisse mais forte, mas com fruta fresca e sem Brett. Menos ácida do que Lambic e Gueuze e sem o caráter da Brett. A partir do guia de estilos 2021, cervejas semelhantes podem ser inscritas no estilo mais amplo 28C Wild Specialty Beer Style.

\begin{tabular}{@{}p{35mm}p{35mm}@{}}
  \textbf{Estatísticas}: & OG: 1.039 – 1.048\\
  IBU: 2 - 8 & FG: 1.004 - 1.012 \\
  SRM: 2 - 6  & ABV: 4.0 - 5.5\%
\end{tabular}

\textbf{Exemplos Comerciais}: Armada Daenerys, Blumenau Catharina Sour Pêssego, Istepô Goiabêra, Itajahy Catharina Araçá Sour, Liffey Coroa Real, UNIKA Tangerina Clemenules

\textbf{Atributos de Estilo}: craft-style, south-america, fruit, sour, specialty-beer
