\subsection*{Formato da descrição de estilos}
\addcontentsline{toc}{subsection}{Formato da descrição de estilos}
Nós usamos um formato padrão para descrever os estilos de cerveja. As sessões dentro deste modelo/template têm significado específico, e devem ser compreendidas para que não sejam mal utilizadas:
\begin{itemize}
\item \textbf{Impressão Geral:} Esta sessão descreve a essência do estilo - aqueles pontos que o distinguem dos demais estilos e o tornam único. Também pode ser pensado como uma descrição expandida, a nível de consumidores, útil para descrever e diferenciar o estilo para alguém que não é um \textit{beer geek} ou juiz. Essa sessão reconhece os diversos usos fora de julgamentos e possibilita a outros descreverem uma cerveja de forma simples, sem precisar de todos detalhes necessários aos juízes.
\item \textbf{Aparência, Aroma, Sabor, Sensação na Boca:} Estas quatro sessões são os blocos sensoriais básicos que constroem e definem o estilo e são os padrões pelos quais a cerveja é julgada durante uma competição. Estas sessões focam nas percepções sensoriais provenientes dos ingredientes, não nos ingredientes ou nos processos em si. Por exemplo, dizer que uma Munich Helles possui um gosto como o de malte Pilsen Continental é uma ótima maneira de descrever resumidamente o que é percebido; exceto, claro, caso você não tenha ideia de qual o gosto do malte Pilsen Continental. Nossas diretrizes de guia de estilos são escritas de forma que um juiz treinado que não tenha provado exemplares de um determinado estilo possa efetuar um bom trabalho julgando-o, usando um modelo de avaliação estruturado junto do guia de estilos como referência.
\item \textbf{Comentários:} Esta sessão contém trivialidades interessantes ou notas adicionais sobre o estilo que não afetam sua avaliação sensorial. Nem todo estilo terá comentários extensos; alguns são bem simples.
\item \textbf{História:} O BJCP não é uma organização com foco em pesquisa histórica. Valemos-nos nas informações disponíveis, revisando frequentemente nossos resumos conforme novos fatos são publicados. Nossas histórias consistem de sumários abreviados, que trazem os pontos mais importantes do desenvolvimento do estilo. Por favor, não entenda que essas notas são a história completa dos estilos.
\item \textbf{Ingredientes:} Identificamos os ingredientes ou processos típicos e comuns que trazem o caráter distinto ao estilo. Por favor, não trate essas notas como receitas ou requisitos - cerveja pode ser feita de diversas maneiras.
\item \textbf{Comparação de Estilos:} Como alguns podem entender melhor um estilo não conhecido ao ser descrito em comparação com outros estilos conhecidos, fornecemos notas nos pontos chave que distinguem um estilo de outro similar ou relacionado. Nem toda comparação de estilo possível é listada no guia.
\item \textbf{Instruções para Inscrição:} Esta sessão identifica as informações necessárias para que juízes possam avaliar a inscrição em uma competição. Essa informação deve sempre ser fornecida por quem inscrever uma amostra, aceita pelo software utilizado na competição e providenciada aos juízes. Participantes devem conseguir suprir comentários adicionais sobre suas inscrições, sujeito a revisão pelos organizadores da competição.
\item \textbf{Estatísticas Vitais:}
Original Gravity (OG) = Densidade Original ou Extrato Primário
Final Gravity (FG) = Densidade Final ou Extrato Final
Alcohol-by-Volume (ABV) = Álcool por Volume (%)
International Bitterness Units (IBUs) = Unidade Internacional de Amargor
Para aqueles de fora dos Estados Unidos que usam a escala de cor do \textit{European Brewing Convention} (EBC), saiba que o valor EBC é praticamente o dobro do valor equivalente no SRM. Para aqueles familiarizados com o sistema Lovibond, Lovibond é praticamente equivalente ao SRM para cores de cervejas, excluindo as mais escuras. Para os puritanos presentes, estamos tratando do que é distinguível para um juiz usando os olhos, não químicos utilizando equipamentos analíticos dentro de um laboratório.
Algumas categorias incluem múltiplos estilos que apresentam certa continuidade, como, por exemplo, a \textit{English Bitter} ou \textit{Scottish Ale}. Quando fornecemos uma linha divisória entre estes estilos, tipicamente usamos um único número para representar a faixa superior de um estilo e a faixa inferior do próximo. Isso não implica que a cerveja que esteja no limite de um parâmetro (por exemplo, ABV ou OG) deve ser inscrita em ambos estilos. Nenhuma sobreposição é intencional. Nesses casos, trate o limite superior como “terminando logo antes” e o limite inferior como “começando aqui” dos números apresentados.
Tenha em mente que essas estatísticas vitais ainda são apenas diretrizes, não dados absolutos. Exemplos Comerciais fora destes parâmetros com certeza existem, mas essas estatísticas são feitas para descrever onde a maior parte dos exemplares está. Elas ajudam a determinar a ordem de julgamento, não se um exemplo deve ser desclassificado.
\item \textbf{Exemplos Comerciais:} Incluímos uma seleção dos exemplares comerciais atuais que acreditamos serem representantes do estilo no período em que publicamos o guia. Podemos publicar exemplares adicionais no site do BJCP no futuro. Nós não podemos garantir que as cervejarias continuarão produzindo estes exemplares, que os nomes permaneçam os mesmos, que as receitas não mudarão ou que estejam sempre disponíveis para compra em seu local. Alguns são sazonais, rotativos, apenas encontrados em \textbf{brewpub} ou mesmo difíceis de encontrar fora de festivais, competições e mercados locais.
Não faça presunções de significados adicionais sobre a ordem em que listamos os exemplares. Não assuma que todo exemplo comercial listado obteria a nota máxima quando avaliado seguindo as descrições do estilo. Não é apenas porque um exemplar comercial é referenciado para o estilo que todo exemplar será sempre de classe mundial. Algumas cervejas podem ser mal manipuladas e algumas mudam com o passar do tempo.
\item \textbf{Atributos de Estilo:} Para facilitar a organização dos estilos em agrupamentos alternativos, utilizamos \textbf{tags} para evidenciar atributos ou informações sobre o estilo. As \textbf{tags} não estão em nenhuma ordem em particular e não devem ser utilizadas para aferir um significado mais profundo.
\end{itemize}
