\subsection*{Linguagem de descrição dos estilos}
\addcontentsline{toc}{subsection}{Linguagem de descrição dos estilos}
O guia de estilos é um conjunto de documentos longos e algumas descrições de estilos são complexas. Para evitar que a linguagem usada seja excessivamente entediante, sinônimos (palavras ou frases significando exatamente a mesma coisa ou tendo um sentido muito próximo) são frequentemente utilizados. Não busque achar mais significado nos sinônimos do que o esperado. No passado, algumas pessoas questionaram a diferença entre baixo e leve, médio e moderado, escuro e profundo e diversos outros exemplos. A resposta é, não existe diferença entre estas palavras neste contexto. Elas tendem a possuir o mesmo significado (normalmente, intensidades relativas de percepções). Entenda essas palavras com o seu significado básico. Caso você se pegue analisando o guia de estilos como se estivesse à procura de uma mensagem secreta dentro de uma música tocada de trás pra frente, você está se esforçando demais.

Quando usamos diversas palavras para expressar coisas similares, estamos apenas tentando ser versados e usar um vocabulário razoavelmente instruído. Não buscamos ser a polícia da linguagem e dizer que um sinônimo está sempre certo e os demais sempre errados. Portanto não busque por inconsistências no uso de termos, nem tente adicionar distinções com nuances em diferentes palavras usadas para expressar o mesmo contexto. Não exija que as palavras usadas no guia de estilos sejam exatamente as palavras usadas nas súmulas ou exames. Preocupe-se mais com o conceito que está sendo passado e menos com a expressão específica do conceito.

Formatamos as listas utilizando a vírgula de Oxford, que é uma construção gramatical menos ambígua. Ao descrever listas de características, \textit{\textbf{“ou”} significa algum ou todos os itens podem estar presentes, \textbf{“todos”} significa que todos os itens devem estar presentes, \textbf{“qualquer um/ou”} significa que apenas um pode estar presente, \textbf{“nenhum”} significa que nenhum dos itens podem estar presentes.} O uso anterior de \textbf{“e/ou”} foi substituído por apenas \textbf{“ou”}, que tem o mesmo significado lógico.

Quando usamos nomes de estilos em Letras Maiúsculas, a intenção é que seja uma referência cruzada aos estilos contidos neste guia. Nomes de estilos sem letras maiúsculas representam uma referência mais generalista.

Esteja atento aos modificadores usados nas descrições dos estilos. Busque por orientação na magnitude e qualidade de cada característica. Note que diversas características são opcionais. Cervejas que não trazem estes elementos não necessários não devem ser rebaixadas. Se uma intensidade é usada juntamente com um indicador opcional, significa que qualquer intensidade, desde nenhuma até a listada, é aceitável, mas esta característica não é obrigatória.

Frases como \textit{“pode ter”}, \textit{“pode conter”}, \textit{“pode demonstrar”}, \textit{“é aceitável”}, \textit{“é apropriado”}, \textit{“é típico”}, \textit{“opcionalmente”}, etc. são indicadores de elementos opcionais. Elementos necessários são normalmente escritos com frases declaratórias ou usam palavras como \textit{“deve”} ou \textit{“precisa”}. Elementos que não devem estar presentes normalmente usam frases como \textit{“inapropriado”}, \textit{“sem”} ou \textit{“não apresenta”}. Novamente, entenda essas palavras no seu sentido bruto.

Não foque exageradamente em palavras soltas ou frases dentro das descrições dos estilos a ponto de perder o propósito maior da descrição. Entenda a impressão geral do estilo, o equilíbrio geral e como o estilo difere de estilos relacionados ou similares. Não dê importância desproporcional a frases especificas se elas forem mudar a impressão geral, equilíbrio, o propósito do estilo ou caso isso faça com que ela seja desqualificada ou rebaixada a problemática para o estilo.