\subsection*{Nomenclatura dos Estilos e Categorias}
\addcontentsline{toc}{subsection}{Nomenclatura dos Estilos e Categorias}

Escolhemos nomes e títulos para melhor representar estilos e agrupamentos em nosso sistema de categorização.  Não deixe com que estes nomes interfiram no seu entendimento real das descrições do estilo – este é o ponto principal do guia de estilos. Nós não estamos dizendo para as cervejarias como elas devem chamar seus produtos, estamos apenas fornecendo nomes que podem ser usados para uma fácil referência.

Entendemos que muitos dos estilos estabelecidos por nós podem ter nomes alternativos, ou serem chamados por nomes diferentes em outras (e às vezes nas mesmas) partes do mundo. No passado, usamos diversos nomes em títulos de estilos para evitar demonstrar alguma preferência, mas isso levava ao uso equivocado de todas versões do nome de forma simultânea. Então, decidimos escolher um nome não ambíguo para cada estilo.

Adicionamos o país ou região de origem à alguns títulos de estilos para diferenciar entre estilos usando um nome em comum (como Porter). Selecionamos os títulos com a intenção que sejam únicos e descritivos, mas não necessariamente são a forma como eles são chamados em seus mercados locais. Não estamos buscando nenhuma apropriação ou domínio baseado nos nomes que selecionamos, também pedimos desculpas por qualquer ofensa não intencional que possa ocorrer na esfera política, ética ou social.

Alguns nomes que usamos são protegidos por marcas registradas ou apelações de origem controlada. Não estamos dizendo que estes não devem ser respeitados, ou que cervejarias comerciais podem usar estes nomes. Pelo contrário, estas são as formas mais apropriadas de utilizar os nomes quando tratando de alguns estilos. Caso este conceito seja difícil de entender, apenas assuma que existe uma designação implícita de “-estilo” em todo nome de estilo. Nós não queríamos utilizar “-estilo” na forma como nomeamos visto que este é um guia de estilos, portanto tudo é um estilo.

\textit{(Nota do tradutor: O significado se perde um pouco na tradução. O importante é entender que alguns estilos são protegidos por denominação de origem controlada ou podem conter algum registro sobre a nomenclatura. Por exemplo cervejas Lambic ou Trapistas. Cervejas não podem utilizar estes nomes seus rótulos sem permissão. O “-estilo” funciona como “tipo”, por exemplo cerveja “tipo” Lambic.)}