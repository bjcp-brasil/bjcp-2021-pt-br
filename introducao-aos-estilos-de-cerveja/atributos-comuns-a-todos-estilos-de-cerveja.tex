\subsection*{Atributos Comuns a Todos Estilos de Cerveja}
\addcontentsline{toc}{subsection}{Atributos Comuns a Todos Estilos de Cerveja}
Presume-se que os atributos sensoriais discutidos nesta seção estejam presentes em todas as descrições de estilo de cerveja, a menos que indicado de outra forma. Não é necessário repetir todas características para todas descrições de estilos. \textit{Caso uma característica não esteja mencionada em uma descrição de estilo, como diacetil, por exemplo, não assuma que ela é de alguma forma aceitável.}\\
\textbf{A menos que explicitamente escrito em uma descrição de estilo individual}, assumimos que todos os estilos de cerveja possuem fermentação limpa e estão livres de falhas técnicas, incluindo acetaldeído, adstringência, clorofenol, diacetil, DMS, álcoois superiores e fenóis. Também é presumido que todos estilos de cerveja estão livres de problemas de envase e guarda, incluindo oxidação, fotodegradação/\textit{light-struck}, acidez e características de mofo.

Na sensação na boca, presumimos que todas cervejas estão livres de adstringência, não possuam cremosidade ou outras sensações palatáveis, a menos que descritas. Presumimos que cervejas com uma graduação alcoólica de 6% ou menos não tenham sabor ou a sensação de aquecimento provinda do álcool, a menos que esteja descrito. Cervejas mais alcoólicas e que possuem uma presença perceptível de álcool não devem ser agressivas, quentes, como solvente ou queimar. O caráter do álcool, a menos que descrito, deve ser limpo e livre de álcoois superiores.

Lagers tendem a ser mais leves, limpas e livre de ésteres. Lagers mais fortes e escuras podem conter ésteres leves que complementam o sabor. Lagers claras, especialmente as muito frescas, podem ter notas sulfurosas leves, porém agradáveis, provenientes da levedura. Algumas notas sulfurosas podem ser voláteis. Estas notas sulfurosas são aceitáveis, mas aromas sulfurosos desagradáveis (ovos podres, gás de esgoto etc.) são defeitos.

Estilos produzidos com grandes quantidades de malte Pilsen podem ter baixas notas de DMS. Isto não é uma falha, mas também não é um caráter obrigatório, a menos que assim descrito. Em ambos casos, uma pequena quantidade de DMS ou enxofre não significa que quantidades proeminentes são de alguma forma desejáveis – elas não são. Apenas esteja ciente que o uso de alguns ingredientes tradicionais frequentemente deixa pequenas indicações sensoriais de sua presença e que estas poderiam ser consideradas falhas em outros contextos. Isso é perfeitamente aceitável, ainda que não necessário.

A menos que assim descrito, presuma que todas lagers não possuam nenhum aspecto frutado (ésteres). Ales tendem a ser menos suaves que lagers, então, ao menos que descrito, assuma que todas ales podem conter certos ésteres (não é necessário, porém, não é uma falha).
