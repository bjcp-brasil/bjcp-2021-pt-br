\subsection*{Categorização Básica}
\addcontentsline{toc}{subsection}{Categorização Básica}
A categorização mais generalista de estilos de cerveja pelo tipo de levedura é um fenômeno moderno oriundo da cerveja artesanal. Cervejeiros americanos e a maioria dos outros cervejeiros artesanais chamam suas cervejas de ales caso usem uma levedura de alta fermentação (ale) e lagers caso utilizem leveduras de baixa fermentação (lager). A maioria dos sistemas de categorização permitirão uma terceira classificação, frequentemente chamada de fermentação espontânea por causa do método usado, no entanto fermentação mista ou selvagem são provavelmente os termos mais utilizados pelas cervejarias artesanais modernas para cervejas fermentadas com bactérias ou leveduras não-Saccharomyces. O termo selvagem neste contexto não deve ser interpretado para implicar fermentação espontânea; a maioria é inoculada diretamente com as cepas de fermentação desejada.

Na Alemanha e outros centros cervejeiros do velho mundo a terminologia normalmente utilizada para diferenciar as cervejas é se referir a elas como alta-fermentação ou baixa-fermentação. Alemães pensam em ale como um tipo de cerveja inglesa e lager como um método de condicionamento/maturação a frio da cerveja (lagering). Portanto, alemães consideram de forma normal uma Kölsch como uma cerveja lager de alta fermentação, ao invés de uma ale.

Cervejeiros ingleses, principalmente quando lidando no contexto histórico costumam separar ales de porters e stouts como tipos de cerveja (apesar de na sequência, falarem que não existe diferença entre porters e stouts). Eles podem ir ainda além e descrever ales como historicamente destintas de cervejas que foram lupuladas (ou ainda mais lupuladas). Essas notas históricas são importantes para entender receitas e textos antigos, mas possuem pouca relevância atualmente nos termos comuns usados para descrever cerveja.

Este guia de estilos busca utilizar as definições modernas de cerveja artesanal de ale, lager e selvagem como os maiores agrupamentos de estilos de cerveja, mas menciona como são descritos em contextos locais ou regionais, quando possível.