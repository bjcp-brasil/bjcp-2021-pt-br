\subsection*{Categorização Básica}
\addcontentsline{toc}{subsection}{Categorização Básica}
A categorização mais generalista de estilos de cerveja pelo tipo de levedura é um fenômeno moderno, oriundo da cerveja artesanal. Cervejeiros americanos e a maioria dos outros cervejeiros artesanais chamam suas cervejas de \textit{ales}, caso usem uma levedura de alta fermentação (Ale), e \textit{lagers}, caso utilizem leveduras de baixa fermentação (Lager). A maioria dos sistemas de categorização permitirão uma terceira classificação, frequentemente chamada de fermentação espontânea por causa do método usado; no entanto, \textit{fermentação mista} ou \textit{selvagem} são, provavelmente, os termos mais utilizados pelas cervejarias artesanais modernas para cervejas fermentadas com bactérias ou com leveduras não-Saccharomyces. O termo selvagem, neste contexto, não deve ser interpretado para implicar fermentação espontânea, já que a maioria é inoculada diretamente com as cepas de fermentação desejada.

Na Alemanha e outros centros cervejeiros do Velho Mundo, a terminologia normalmente utilizada para diferenciar as cervejas é se referir a elas como de alta-fermentação ou baixa-fermentação. Os alemães pensam em ale como um tipo de cerveja inglesa e lager como um método de condicionamento/maturação a frio da cerveja (\textit{lagering}). Portanto, os alemães normalmente consideram uma Kölsch como uma cerveja lager de alta fermentação ao invés de uma ale.

Cervejeiros ingleses, principalmente quando lidando no contexto histórico, costumam separar ales de porters e stouts como tipos de cerveja (apesar de, em seguida, falarem que não existe diferença entre porters e stouts). Eles podem ir ainda além e descrever ales como historicamente distintas de cervejas que foram lupuladas (ou altamente lupuladas). Essas notas históricas são importantes para entender receitas e textos antigos, mas possuem pouca relevância atualmente nos termos comuns usados para descrever cervejas.

Este guia de estilos busca utilizar as definições modernas de cerveja artesanal de \textit{ale}, \textit{lager} e \textit{selvagem} como os maiores agrupamentos de estilos de cerveja, mas menciona como são descritos em contextos locais ou regionais, quando possível.
