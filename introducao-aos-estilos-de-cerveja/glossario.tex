\subsection*{Glossário}
\addcontentsline{toc}{subsection}{Glossário}
\textit{Algumas terminologias utilizadas no guia de estilos podem ser desconhecidas para alguns leitores. Ao invés de incluir um dicionário completo, destacamos alguns termos que podem não ser bem entendidos ou que têm um significado especifico implícito dentro do guia. Às vezes nomes de ingredientes são usados como abreviações do caráter que trazem à cerveja. Quando juízes utilizam estes termos, eles não necessariamente estão implicando que estes ingredientes especificamente foram utilizados, apenas que a característica percebida combina com a comumente proveniente dos ingredientes mencionados.}
\subsubsection*{Termos Referentes à Lúpulo}
\addcontentsline{toc}{subsubsection}{Termos Referentes à Lúpulo}
\textbf{Lúpulos Americanos} - lúpulos americanos da era das cervejas artesanais, normalmente cítricos, resinosos, pinho ou de características similares. Os lúpulos mais modernos podem ter um escopo de características maiores, como por exemplo drupas, bagas/\textit{berries}, frutas tropicais e melão.

\textbf{Lúpulos Continentais, Lúpulos do Velho Mundo} - lúpulos europeus tradicionais, incluindo lúpulos alemães e checos tradicionais, lúpulos ingleses e outras variedades da Europa continental. Normalmente descritos como florais, condimentados, herbais ou terrosos. Normalmente menos intensos que muitos dos lúpulos do novo mundo.

\textbf{Dry-Hopping} - uma adição de lúpulo realizada após a fervura que traz à cerveja um aroma fresco e vívido de lúpulo. Uma cerveja com dry hopping é mais robusta, vívida e intensa que uma mesma cerveja sem o \textit{dry hopping}. Este método pode alterar o equilíbrio da cerveja para ser mais focada nos lúpulos sem aumentar seu amargor. Não deve ser gramíneo, vegetal, oxidado, parecido com queijo ou ter um caráter de velho. Fresco e jovem, não cozido.

\textbf{Juicy} - um termo moderno e na moda utilizado para descrever lúpulos que possuem característica lembrando suco de frutas frescas, especialmente de frutas tropicais. Possui outros significados como por exemplo, “dar água na boca” ou “úmido” que não se aplica à cerveja.

\textbf{Lúpulos do Novo Mundo} - lúpulos americanos, incluindo os da Austrália, Nova Zelândia e outros locais fora os do velho mundo. Podem ter todos atributos dos lúpulos americanos clássicos, assim como frutas tropicais, drupas, uva branca e outras características aromáticas interessantes.

\textbf{Lúpulos Alemães ou Checos tradicionais} - também chamados de lúpulos nobres ou autóctones (\textit{landrace}), considerados por muito tempo por ter o caráter mais fino e refinado para lagers europeias tradicionais. Comumente tendo um caráter sutil, leve floral, condimentado ou herbal. Tradicional implica que estes são variedades clássicas, ao invés de lúpulos modernos e mais agressivos.
\subsubsection*{Termos de Mosturação e Malte}
\addcontentsline{toc}{subsubsection}{Termos de Mosturação e Malte}
\textbf{Biscoito} - seco, cereais tostados, farinha de trigo ou sabor de massa lembrando biscoitos ou cookies digestivos ingleses. Para cerveja, um sabor normalmente associado com o malte Biscuit (biscoito) e alguns maltes ingleses tradicionais.

\textbf{Produtos de Maillard} - um grupo de compostos produzidos por uma interação complexa entre açúcares e aminoácidos em altas temperaturas resultando em cores marrons e compostos ricos, maltados e às vezes até de certa forma que remetem à carne. Nas versões anteriores do guia, tratada como melanoidina, que são um subgrupo de produtos de Maillard responsáveis pelas cores vermelhas-marrons (e de acordo com Kunze, “intensas em aroma”). Em algumas literaturas cervejeiras, os termos melanoidina e produtos de Maillard são utilizados sem distinção. A química e o sabor característico dos produtos de Maillard não são bem compreendidos, então juízes e cervejeiros devem evitar discussões excessivamente pedantes sobre o tema. O ponto central é que nos referimos aos sabores de malte ricos e precisamos de algum jeito abreviado de discutir eles.

\textbf{Malte Munique} - pode trazer pão, caráter de malte rico que amplia o perfil do malte na cerveja sem adicionar dulçor residual, ainda que alguns confundam caráter maltado com dulçor. Maltes Munique mais escuros podem adicionar um caráter intenso de malte tostado, similar à casca de pão tostada.

\textbf{Malte Pilsner ou Pilsen} - Malte Pilsner continental é bastante distinto e possui um caráter levemente adocicado, lembrando cereais leves, levemente tostado e semelhante à mel. Possui mais percursores de DMS que outros maltes e seu uso pode às vezes resultar em um baixo sabor de milho do DMS.

\textbf{Malte Vienna} - Pode trazer um caráter e malte de pão com tosta, mas não espere que as notas tostadas sejam extremas - elas são mais como a crosta de um pão recém assado do que pão tostado.
\subsubsection*{Termos de Levedura ou Fermentação}
\addcontentsline{toc}{subsubsection}{Termos de Levedura ou Fermentação}
\textbf{Goma de mascar (bubblegum)} - refere-se ao perfil de sabor do chiclete Bazooka original, uma goma de mascar rosa; um sabor que lembra um mix de frutas doces dominado por banana e morango e com sabor de ponche de frutas.

\textit{(Nota do tradutor: Bubbaloo tutti-frutti é a opção local neste perfil.)}\\
\textbf{Perfil de fermentação limpa} - um caráter contendo de muito baixo a nenhum subproduto de fermentação na cerveja final, normalmente implica que não há ésteres, diacetil, acetaldeído ou compostos similares, exceto quando especificamente informado. Um jeito abreviado de dizer que uma lista longa de possíveis subprodutos de fermentação não estão presentes em quantidade perceptíveis (quantias praticamente imperceptíveis próximas ao limiar de percepção são normalmente aceitáveis, no entanto).

\textbf{Kveik} - tradicionalmente, um blend de leveduras da Noruega utilizadas para produção de farmhouse ales, normalmente disponíveis como uma cepa única. Não é um estilo de cerveja.

\textbf{Frutas de pomo} - maçã, pera, marmelo. A classificação botânica contém outras frutas, mas estas são as mais comuns.

\textbf{Drupas} - frutas carnudas com um único caroço, como por exemplo cereja, ameixa, pêssego, damasco, manga, etc.
\subsubsection*{Termos de Fermentação Mista}
\addcontentsline{toc}{subsubsection}{Termos de Fermentação Mista}
\textbf{Caráter acético} - parecido com vinagre, pungente, não é uma acidez limpa.

\textbf{Brett} - Abreviação para \textit{Brettanomyces}, um gênero de levedura de alta atenuação que é normalmente utilizado para produzir sabores complexos frutados (frutas de pomo, frutas tropicais, drupas), florais e frequentemente \textit{funky} e complexos (couro, suor, estábulo, pelego, \textit{funky}, etc) em bebidas fermentadas. Derivados de fenóis ou ácidos graxos produzidos durante a fermentação. Significa literalmente “fungo britânico” e é associado com características produzidas durante o envelhecimento em barris.

As variedades mais comuns utilizadas no processo fabril incluem \textit{B. bruxellensis} e \textit{B. anomalous}, apesar de também serem conhecidas por outros nomes. Existem diversas cepas com perfis bastante diferentes (assim como com o \textit{S. cerevisiae}).  Normalmente usadas como cepas de fermentação secundárias, apesar existirem algumas cepas que podem atenuar suficientemente o mosto a ponto de serem usadas como cepas primárias.\\\\
\textit{(Nota do tradutor: não existe uma boa palavra que traduza o termo funk/funky. Preferimos deixar sem tradução visto que é um termo largamente difundido no meio da cerveja artesanal brasileira.)}\\
\textbf{Acidez limpa} - um descritor da qualidade da acidez que implica que a acidez não possui vinagre, funk complexo ou notas excessivas. Normalmente utilizado para descrever uma acidez lática vivaz e de boa qualidade

\textbf{Acetato de etila} - um éster proveniente da fermentação formado por ácido acético e etanol. Produzido pela levedura em diversos níveis de acordo com a cepa e estresse da levedura. Em baixos níveis é frutado como pera, abacaxi ou bagas/berries, mas em altos níveis é definitivamente uma falha e possui aroma de solvente ou esmalte. Altos níveis de oxigênio e levedura selvagem podem gerar quantidade excessiva.

\textbf{Indole} - formado por contaminação por bactérias ‘coliformes’ durante a fermentação. É frequentemente associado à produção simultânea de DMS. Mais frequentemente encontrado em cervejas que têm um tempo de latência (\textit{lag time}) muito longo ou em cervejas de fermentação espontânea. Aroma de fezes, fazenda suja ou fazenda de porco. Em níveis mais baixos, pode ser jasmim ou floral. Sempre uma falha.

\textbf{LAB/BAL} - abreviação de Bactéria Ácido Lática, inclui \textit{Lactobacillus}, \textit{Pediococcus} e outros da família Lactobacillaceae. Um termo mais amplo para identificar a origem da acidez lática.

\textbf{Lacto} - abreviação de \textit{Lactobacillus}.

\textbf{Pedio} - abreviação de \textit{Pediococcus}.

\textbf{Ropiness/Ropey} - descreve uma sensação na boca onde a cerveja desenvolve um aumento em sua viscosidade e se apresenta densa e xaroposa. A causa mais comum é a ação de diversas bactérias, sendo a Pedio a mais comum e ocorre com o aumento na produção de polissacarídeos. É um estágio comum na fermentação com culturas mistas. A presença de Brett reduz esta viscosidade com o tempo.

\textbf{Sacch} - abreviação de \textit{Saccharomyces}.

\textbf{THP} - abreviação de Tetrahidropiridina. Normalmente produzida por Lacto ou brett. Em níveis baixos apresenta característica de cereal, lembrando aveia/cereais tostados (pense no cereal matinal ‘Cheerios’). Em altos níveis, pode ser percebido como gaiola de roedores, roedores e similar à urina (parecido com a falha encontrada em sidra e vinho). A THP aumenta com exposição ao oxigênio, mas a Brett ativa reduzirá ela com o tempo. Sempre uma falha.
\subsubsection*{Termos de Qualidade ou Off Flavors/ Defeitos Sensoriais}
\addcontentsline{toc}{subsubsection}{Termos de Qualidade ou Off Flavors/ Defeitos Sensoriais}
\textbf{Característica de adjunto} - uma característica presente no aroma, sabor e na sensação na boca que reflete o uso de porcentagens maiores de fermentescíveis não malteados.  Pode estar presente com um caráter de milho, um corpo mais leve do que um produto puro malte ou geralmente uma cerveja com sabor mais sutil. Não necessariamente implica o uso de um adjunto específico.

\textbf{Drinkability} - Pertinente a facilidade associada ao beber a cerveja, existem traduções utilizando “tomabilidade”, “facilidade em beber”, “bebabilidade” ou outros termos, porém, de forma geral existe um consenso em utilizar a palavra em sua forma original assim como “dry hopping” ou “funky”. \textit{(Nota do Tradutor: o termo Drinkability não aparece na versão original, mas pode gerar dúvidas)}.

\textbf{Equilibrada} - relativo a um estilo, equilibrada implica agradável, harmonioso, condizente, uma combinação complementar de todos elementos, não um mesmo nível de todos elementos presentes. Não implica uma quantia absoluta, seria uma medida da coordenação apropriada dos elementos que constituem o sabor.

\textbf{Limpa} - ausente de off flavors/defeitos sensoriais, um termo positivo.

\textbf{Crisp} - Uma mudança rápida e abrupta na sensação na boca de uma cerveja indo de suave para uma sensação aguda, levando a um final seco. Normalmente um termo positivo.

\textit{(Nota do tradutor: não existe a palavra crisp em português e nem um termo que defina bem. Pense em algo com final seco, limpo e bem definido. Normalmente se refere ao final das lagers leves)}\\
\textbf{DMS} - Dimetilsulfureto, que pode se apresentar com características bastante diversas. A maioria é inapropriada em qualquer estilo de cerveja, no entanto um sutil caráter de milho cozido pode estar presente e é aceitável em cervejas com alto nível de malte Pilsen. Quando o guia de estilos disser que qualquer nível de DMS é apropriado, é este suave sabor de milho cozido, não um caráter de outros vegetais cozidos ou outros sabores de DMS.

\textbf{Seco} - mesma utilização do vinho, ou seja, sem percepção de dulçor. Bem atenuado. Obviamente não quer dizer o “oposto de molhado” neste contexto.

\textbf{Elegante} - macio, saboroso, refinado, um caráter agradável que indica o uso de ingredientes de alta qualidade manuseados com cautela; sem elementos sobressalentes desagradáveis, sabores agressivos e sensações que agridem o palato.

\textbf{Harsh} - quando aplicado à cerveja, uma textura, sabor ou retrogosto que é desagradável, agressivo, intenso. Alguns sinônimos neste contexto são, áspero, acre, abrasivo, que não é agradável, mais sujo, menos refinado e menos puro. Um termo de qualidade indicando o oposto de macio, limpo e agradável. Pode implicar adstringência, mas pode também se aplicar a amargor, álcool e outras sensações. Negativo.

\textbf{Funky} - um termo positivo ou negativo, dependendo do contexto. Se esperado ou desejável, pode normalmente se manifestar com um caráter de celeiro, feno molhado, sutil terroso, pelego ou curral. Quando muito intenso, inesperado ou indesejável pode ter o caráter de silagem, fecal, fralda de bebê ou baia de cavalo.

\textbf{Rústico} - grosseiro, caráter robusto remanescente de ingredientes tradicionais, mais velhos. Uma experiência sensorial menos refinada de modo geral.
\subsubsection*{Termos de aparência}
\addcontentsline{toc}{subsubsection}{Termos de aparência}
\textbf{Renda belga (Belgian Lace), Rendado} - uma característica do padrão de treliça persistente formado pela espuma no interior do copo conforme a cerveja é consumida. O visual lembra o belo trabalho com rendas belga e é considerado um indicador desejável da qualidade da cerveja.

\textbf{Pernas/Lágrimas} - um padrão que uma bebida deixa no lado de dentro do copo após uma porção ter sido consumida. O termo se refere as gotas que escorrem vagarosamente das paredes do copo deixando rastros. Não é um indicador de qualidade, mas podem indicar um volume mais alto de álcool, açúcar ou glicerol.
