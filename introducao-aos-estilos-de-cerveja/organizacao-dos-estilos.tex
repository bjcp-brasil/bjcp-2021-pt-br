\subsection*{Organização dos Estilos}
\addcontentsline{toc}{subsection}{Organização dos Estilos}
Os estilos de cerveja descritos no guia foram categorizados para auxiliar na organização de competições de cervejeiros caseiros. Categorias (o agrupamento maior de estilos) são construções artificiais que representam uma coleção de subcategorias individuais (estilos de cerveja) que podem ou não ter alguma relação histórica, geográfica ou tradicional dentre si.

Não presuma que a filiação a uma categoria de estilo de alguma forma relacione estilos individuais de cerveja dentre eles. A única razão pela qual os estilos são agrupados juntos é auxiliar na administração de competições quanto a tamanho e complexidade. Nomes dados aos agrupamentos possuem finalidade apenas para competições e podem não ser encontrados em contextos mais amplos na indústria da cerveja e de produção de cerveja.

Competições não precisam julgar cada categoria de estilo separadamente; elas podem ser combinadas, divididas ou reorganizadas de outra forma para a finalidade da competição. Organizadores de competições são livres para combinar subcategorias de estilo em suas próprias categorias de competição. Desde que cada cerveja enviada seja julgada em uma subcategoria (estilo) declarada, qualquer agrupamento lógico é permitido.
