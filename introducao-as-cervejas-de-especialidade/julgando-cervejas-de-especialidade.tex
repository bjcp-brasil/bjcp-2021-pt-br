\subsection*{Julgando Cervejas de Especialidade}
\addcontentsline{toc}{subsection}{Julgando Cervejas de Especialidade}
\textit{Juízes devem ler e entender as instruções da sessão “Inscrevendo Cervejas de Especialidade” fornecidas aos participantes.}

Equilíbrio geral e drinkability são os fatores críticos para o sucesso de uma Cerveja de Especialidade. A inscrição deve ser uma fusão coerente da cerveja com seus ingredientes especiais, onde nenhum ofusca/domina o outro.

Ingredientes Especiais devem complementar e realçar a cerveja base e o produto final deve ser agradável de beber. A cerveja deve apresentar componentes reconhecíveis de acordo com os requerimentos de inscrição para o estilo, tendo em mente que algumas cervejas podem se enquadrar em diversos estilos.

Juízes devem estar cientes de que existe um elemento criativo ao produzir estes estilos e não devem ter preconceito de combinações que pareçam estranhas. Mantenha uma mente aberta porque algumas harmonizações de sabores estranhos podem ser surpreendentemente deliciosas. Incomum não necessáriamente quer dizer melhor, no entanto. O sabor sempre deve ser o fator de decisão final, não a criatividade percebida, a dificuldade de produção ou raridade dos ingredientes.

\subsubsection*{Avaliação Geral}
Juízes experientes vão normalmente provar as Cervejas de Especialidade buscando o prazer geral associado a ela antes de avaliar seus detalhes. Essa rápida avaliação é feita para detectar se a combinação funciona ou não. Se a cerveja possui sabores conflitantes, ela não será agradável independentemente de seu mérito técnico.

\textit{O velho provérbio que diz “not missing the forest for the trees – não perder a floresta pelas árvores” é aplicável. (Nota do Tradutor: no sentido de acabar perdendo o conjunto da obra analisando os pequenos detalhes). Não julgue estes estilos de forma tão rígida como os Estilos Clássicos, correndo o risco de perder uma sinergia bem sucedida de ingredientes}.

\subsubsection*{Estilo Base}
Juízes não devem ser pedantes demais ao procurar todas as características da cerveja do Estilo Base declarado. Afinal, a cerveja base normalmente não contém ingredientes especiais, então a característica sensorial não será a mesma da cerveja original. Existem interações de sabores que podem produzir efeitos sensoriais adicionais.

Juízes também devem entender que o processo de fermentação pode transformar alguns ingredientes (particularmente aqueles com açúcares fermentescíveis), e que a característica do ingrediente especial na cerveja pode não ser percebido da mesma forma que o ingrediente de especialidade sozinho. Portanto, juízes devem buscar a agradabilidade geral e o equilíbrio da combinação final, desde que a cerveja sugira tanto o Estilo Base quanto o Ingrediente de Especialidade ou processo especial.

\subsubsection*{Múltiplos Ingredientes}
Juízes não precisam sentir cada Ingrediente de Especialidade (como por exemplo especiarias) de forma individual quando diversos forem declarados. Normalmente é a combinação resultante que contribui para o caráter maior, então permita que estes ingredientes sejam usados em intensidades variadas para produzir uma experiência de degustação mais agradável.

Nem toda cerveja irá se enquadrar em um estilo perfeitamente. Algumas cervejas com diversos ingredientes podem ser inscritas em diversos estilos. Seja gentil ao julgar estas cervejas. Recompense aquelas cervejas que estão bem-feitas e são agradáveis de beber ao invés de repreender o participante sobre onde ele deveria tê-la inscrito.

Se um inscrito declarar um alergênico potencial na cerveja, não deduza pontos caso você não consiga notar sua presença.

\subsubsection*{Efeito do Ingrediente Especial no Equilíbrio}
A característica do Ingrediente de Especialidade/Especial deve ser agradável e prover suporte, não artificial ou inapropriadamente forte, levando em conta que alguns ingredientes têm uma característica inerente muito prominente. Lúpulos de aroma, subprodutos de fermentação e componentes de malte da cerveja base podem não ser identificáveis quando ingredientes adicionais estão presentes e também podem ser ofuscados de forma intencional para deixar que a característica do ingrediente adicionado seja mais claramente percebido na apresentação final.

Aroma de lúpulo pode estar ausente ou equilibrado com os ingredientes adicionados, dependendo do estilo. Os ingredientes adicionados devem adicionar uma complexidade extra à cerveja, mas não tão prominente a ponto de desequilibrar a apresentação resultante.
